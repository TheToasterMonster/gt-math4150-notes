\chapter{Sept.~17 --- Arithmetic Functions}

\section{Properties of Multiplicative Functions}

\begin{remark}
  We write $\sum_{d \mid n} f(d)$
  to denote a sum over the positive
  divisors of $n$. For instance,
  \[
    \sum_{d \mid 12} f(d)
    = f(1) + f(2) + f(3) + f(4) + f(6) + f(12).
  \]
\end{remark}

\begin{theorem}\label{thm:sum-mult}
  Let $f$ be an arithmetic function, and
  for $n \in \Z$, $n > 0$, define
  \[
    F(n) = \sum_{d \mid n} f(d).
  \]
  If $f$ is multiplicative, then so is
  $F$.
\end{theorem}

\begin{proof}
  Let $m, n$ be relative prime. We need
  to show that $F(mn) = F(m)F(n)$. We have
  \[
    F(mn) = \sum_{d \mid mn} f(d).
  \]
  We claim that every divisor $d$ of
  $mn$ can be written uniquely as
  $d = d_1 d_2$, where $d_1 \mid m$ and
  $d_2 \mid m$. Moreover, any such
  product $d_1 d_2$ is a divisor of $mn$.
  To see this, write
  $m = p_1^{a_1} \dots p_r^{a_r}$ and
  $n = q_1^{b_1} \dots q_s^{b_s}$, where
  all the $p_1, \dots, p_r, q_1, \dots, q_s$
  are distinct. Then if $d \mid mn$, then
  \[
    d = p_1^{e_1} \dots p_r^{e_r} q_1^{f_1} \dots q_s^{f_s},
    \quad 0 \le e_i \le q_i, 0 \le f_j \le b_j.
  \]
  Then we must choose $d_1 = p_1^{e_1} \dots p_r^{e_r}$
  and $d_2 = q_1^{f_1} \dots q_s^{f_s}$,
  which proves the claim.

  Using the claim, we can split the sum
  into
  \[
    F(mn) = \sum_{d \mid mn} f(d)
    = \sum_{d_1 \mid m} \sum_{d_2 \mid n} f(d_1 d_2)
    = \sum_{d_1 \mid m} \sum_{d_2 \mid n} f(d_1) f(d_2)
    = F(m) F(n),
  \]
  where we note that $(d_1, d_2) = 1$
  since $(m, n) = 1$.
\end{proof}

\begin{example}
  Let $m = 4$, $n = 3$. Then we can write
  \begin{align*}
    F(3 \cdot 4)
    = \sum_{d \mid 12}
    f(d)
    &= f(1) + f(2) + f(3) + f(4) + f(6) + f(12) \\
    &= f(1 \cdot 1) + f(1 \cdot 2) + f(3 \cdot 1) + f(1 \cdot 4) + f(3 \cdot 2) + f(3 \cdot 4) \\
    &= f(1) f(1) + f(1) f(2) + f(3) f(1) + f(1) f(4) + f(3) f(2) + f(3) f(4) \\
    &= (f(1) + f(3))(f(1) + f(2) + f(4))
    = F(3) F(4).
  \end{align*}
\end{example}

\section{Properties of the Euler Phi Function}

\begin{theorem}\label{thm:phi-multiplicative}
  The Euler $\varphi$ function is
  multiplicative.
\end{theorem}

\begin{proof}
  Let $m, n \in \Z$, $m, n > 0$ with
  $(m, n) = 1$. We need to show that
  $\varphi(mn) = \varphi(m) \varphi(n)$.
  Consider the array of positive integers
  $\le mn$ organized as follows:
  \[
  \begin{matrix}
    1 & m + 1 & 2m + 1 & \cdots & (n - 1)m + 1 \\
    2 & m + 2 & 2m + 2 & \cdots & (n - 1)m + 2 \\
    \vdots & \vdots & \vdots & \ddots & \vdots \\
    i & m + i & 2m + i & \cdots & (n - 1)m + i \\
    \vdots & \vdots & \vdots & \ddots & \vdots \\
    m & 2m & 3m & \cdots & nm
  \end{matrix}
  \]
  Consider the $i$th row. If $(i, m) > 1$,
  then no element on the $i$th row is
  relatively prime to $m$ (and hence
  cannot be relatively prime to $mn$). Thus
  we may restrict our attention to those
  $i$ that satisfy $(i, m) = 1$. There
  are, by definition, $\varphi(m)$
  such values of $i$. The entries in the
  $i$th row are
  \[
    i, \quad m + i, \quad 2m + i, \quad \dots, \quad (n - 1)m + i.
  \]
  We claim that this is a complete
  residue system modulo $n$. To see this,
  suppose that
  \[
    km + i \equiv jm + i \pmod{n},
    \quad 0 \le k, j \le n - 1.
  \]
  Then $km \equiv jm \Pmod{n}$. Since
  $(m, n) = 1$, this implies
  $k \equiv j \Pmod{n}$. Since
  $0 \le k, j \le n - 1$, we must have
  $k = j$. The claim follows since
  we have $n$ non-congruent elements
  (modulo $n$)
  in the list. Thus, there are
  $\varphi(n)$ elements in the $i$th row
  that are relatively prime to $n$. Also,
  $(km + i, m) = (i, m) = 1$
  by the Euclidean algorithm, so
  they are relatively prime to $m$ as well. Thus
  $\varphi(mn) = \varphi(m) \varphi(n)$.
\end{proof}

\begin{theorem}
  Let $p$ be prime, $a \in \Z$, $a > 0$.
  Then $\varphi(p^a) = p^a - p^{a - 1}$.
\end{theorem}

\begin{proof}
  The total number of integers
  not exceeding $p^a$ is $p^a$. The only
  integers not relatively prime to
  $p^a$ are the multiples of $p$:
  $p, 2p, 3p, \dots, (p^{a - 1}) p$.
  There are $p^{a - 1}$ such integers, so
  $\varphi(p^a) = p^a - p^{a - 1}$.
\end{proof}

\begin{theorem}\label{thm:phi-product}
  Let $n \in \Z$, $n > 0$. Then
  \[
    \varphi(n) = n \prod_{p \mid n} \left(1 - \frac{1}{p}\right).
  \]
\end{theorem}

\begin{proof}
  Write $n = p_1^{a_1} \dots p_r^{a_r}$.
  Then
  \begin{align*}
    \varphi(n)
    = \varphi(p_1^{a_1} \cdots p_r^{a_r})
    &= \varphi(p_1^{a_1}) \cdots \varphi(p_r^{a_r})
    = (p_1^{a_1} - p_1^{a_1 - 1}) \cdots (p_r^{a_r} - p_r^{a_r - 1}) \\
    &= p_1^{a_1} \cdots p_r^{a_r} \left(1 - \frac{1}{p_1}\right) \cdots \left(1 - \frac{1}{p_r}\right)
    = n \prod_{p \mid n} \left(1 - \frac{1}{p}\right).
  \end{align*}
  This proves the desired formula.
\end{proof}

\begin{remark}
  One can interpret
  Theorem \ref{thm:phi-product}
  probabilistically: It says that
  $\varphi(n)$ is $n$ times the ``probability''
  that an integer is not divisible by any
  of the primes dividing $n$.
\end{remark}

\begin{example}
  Consider $n = 504 = 2^3 \cdot 3^2 \cdot 7$.
  Then $\varphi(n)$ is given by
  \[
    \varphi(504)
    = 504 \left(1 - \frac{1}{2}\right)
    \left(1 - \frac{1}{3}\right)
    \left(1 - \frac{1}{7}\right)
    = 504 \cdot \frac{1}{2} \cdot \frac{2}{3} \cdot \frac{6}{7} = 144.
  \]
\end{example}

\begin{theorem}[Gauss]\label{thm:euler-phi-sum}
  Let $n \in \Z$, $n > 0$. Then
  \[
    \sum_{d \mid n} \varphi(d) = n.
  \]
\end{theorem}

\begin{proof}
  Let $d$ be a divisor of $n$. Define
  the set
  \[
    S_d = \{1 \le m \le n : (m, n) = d\}.
  \]
  Note that $(m, n) = d$ if and only if
  $(m / d, n / d) = 1$. Thus
  $|S_d| = \varphi(n / d)$. Note also
  that every integer less than or equal to
  $n$ belongs to exactly one of the
  $S_d$, so
  \[
    n = \sum_{d \mid n} |S_d| = \sum_{d \mid n} \varphi(n / d)
    = \sum_{d \mid n} \varphi(d),
  \]
  which the last equality follows since
  $\{d : d \mid n\} = \{n / d : d \mid n\}$.
\end{proof}

\begin{example}
  Let $n = 12$. We verify that
  $12 = \sum_{d \mid 12} \varphi(d)$.
  Write the table
  \begin{center}
  \begin{tabular}{c|c}
    $d$ & $S_d$ \\
    \hline
    $1$ & $\{1, 5, 7, 11\}$ \\
    $2$ & $\{2, 10\}$ \\
    $3$ & $\{3, 9\}$ \\
    $4$ & $\{4, 8\}$ \\
    $6$ & $\{6\}$ \\
    $12$ & $\{12\}$
  \end{tabular}
  \end{center}
  Summing the $|S_d| = \varphi(12 / d)$, we
  indeed get
  $12 = 4 + 2 + 2 + 2 + 1 + 1$.
\end{example}
