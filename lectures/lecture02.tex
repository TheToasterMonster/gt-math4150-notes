\chapter{Aug.~20 --- Prime Numbers}

\begin{quote}
  \emph{Two fish are in a tank. One says to the other, "Ha, how do you drive this thing?"}
\end{quote}

\section{Prime Numbers}

\begin{definition}
  Let $p \in \Z$ with $p > 1$. Then $p$
  is \emph{prime} if the only positive
  divisors of $p$ and $1$ and $p$.
  If $n \in \Z$, $n > 1$ and $n$ is not
  prime, then $n$ is \emph{composite}.
\end{definition}

\begin{remark}
  The number $1$ is neither prime
  nor composite.
\end{remark}

\begin{example}
  The following are prime numbers:
  $2, 3, 5, 7, 11, 13, 17, 19, 23, 29, 31, 37, 41, 43, 47, \dots$.
\end{example}

\begin{lemma}\label{lem:has-prime-divisor}
  Every integer greater than $1$ has a
  prime divisor.
\end{lemma}

\begin{proof}
  Assume to the contrary that there exists
  $n > 1$ that has no prime divisor.
  By the well-ordering principle,\footnote{The \emph{well-ordering principle} says that every nonempty subset of the positive integers contains a least element.}
  we may take $n$ to be the smallest
  such positive integer. Since $n$
  has no prime divisors, $n$ cannot be
  prime. Thus $n$ has a divisor $a$
  with $1 < a < n$. Since $1 < a < n$,
  $a$ must have a prime divisor $p$ by the
  minimality of $n$. But then $p \mid a$
  and $a \mid n$, so $p \mid n$ by
  transitivity,
  a contradiction.
\end{proof}

\begin{theorem}[Euclid]
  There are infinitely many prime numbers.
\end{theorem}

\begin{proof}
  Assume to the contrary that there are
  only finitely many primes
  $p_1, p_2, \dots, p_n$.
  Consider
  \[
    N = p_1 p_2 \cdots p_n + 1.
  \]
  By Lemma \ref{lem:has-prime-divisor},
  $N$ has a prime divisor $p = p_j$
  for some $1 \le j \le n$. Since
  $p$ divides $N$ and $p$ divides
  $p_1 p_2 \cdots p_n$,
  $p$ also divides
  $N - p_1 p_2 \cdots p_n = 1$,
  which is a contradiction.
\end{proof}

\begin{exercise}
  Modify the proof and construct
  infinitely many problematic $N$.
\end{exercise}

\section{Sieve of Eratosthenes}

\begin{prop}
  If $n$ is composite, then $n$ has a
  prime divisor that is less than or
  equal to $\sqrt{n}$.
\end{prop}

\begin{proof}
  Since $n$ is composite, $n = ab$ where
  $1 < a, b < n$. Without loss of
  generality, assume $a \le b$. We claim
  $a \le \sqrt{n}$. To see this,
  suppose to the contrary that
  $a > \sqrt{n}$.
  Then $n = ab \ge a^2 > n$, a
  contradiction.
  By Lemma
  \ref{lem:has-prime-divisor},
  $a$ has a prime divisor
  $p \le a \le \sqrt{n}$. But then
  $p \mid a$ and $a \mid n$, so $p \mid n$.
\end{proof}

\begin{remark}
  The proposition implies that if
  all the prime divisors of an integer
  $n$ are greater than $\sqrt{n}$,
  then $n$ is prime. So to check the
  primality of $n$,
  it suffices to check divisibility
  by primes $\le \sqrt{n}$.
\end{remark}

\begin{example}
  The \emph{sieve of Eratosthenes} proceeds
  as follows. To find primes $\le 50$,
  we can delete multiples of primes
  $\le \sqrt{50} \approx 7.07$. To start,
  we know that $2$ is prime. Then cross
  out all multiples of $2$. The smallest
  number remaining is $3$, which we now
  know must be prime. Then cross out all
  multiples of $3$. Continue this process
  until we cross out all multiples of
  $7$, and then all remaining numbers
  are prime.
\end{example}

\section{Gaps in Primes}
\begin{prop}
  For any positive integer $n$, there are
  at least $n$ consecutive composite
  positive integers.
\end{prop}

\begin{proof}
  Consider the following list of $n$
  consecutive numbers:
  \[
    (n + 1)! + 2,\quad (n + 1)! + 3,\quad
    (n + 1)! + 4,\quad
    \dots,\quad (n + 1)! + (n + 1).
  \]
  Note that for any $2 \le m \le n + 1$,
  we have $m \mid m$ and $m \mid (n + 1)!$,
  so $m$ divides $(n + 1)! + m$.
  Thus each number in the above list
  is composite, so we have at least
  $n$ consecutive composite integers.
\end{proof}

\begin{remark}
  With some modifications to this proof
  (namely a more ``efficient''
  construction), one can find asymptotic
  lower bounds for the length of long
  prime gaps.
\end{remark}

\begin{conjecture}
  There are infinitely many pairs
  of primes that differ by exactly $2$.
\end{conjecture}

\begin{remark}
  Zhang (2013) was able to show that
  there are infinitely many pairs of
  pairs of primes whose difference is
  $\le 70,000,000$. This has been
  lowered to $246$ by the Polymath project,
  which included Tao and Maynard. Assuming
  other strong conjectures
  (Elliot-Halberstam), we can get
  down to $6$.
\end{remark}

\begin{remark}
  In addition to long and short prime gaps,
  we can also consider the average length
  of prime gaps. Gauss conjectured that
  as $x \to \infty$, the number of primes
  $\le x$, denoted $\pi(x)$, satisfies
  \[
    \pi(x) \sim \frac{x}{\log x},
  \]
  i.e. $\pi(x)$ is asymptotic
  to $x / {\log x}$. Said
  differently, this says that the
  ``probability'' that an integer $\le x$
  is prime is $\pi(x) / x \sim 1 / {\log x}$.
  This conjecture was proved independently
  in 1896 by de la Vall\'e-Poussin and
  Hadamard, and is now known as the
  \emph{prime number theorem}.
\end{remark}

\begin{definition}
  Let $x \in \R$. Define
  $\pi(x) = |{\{p : \text{$p$ prime}, p \le x\}}|$.
\end{definition}

\begin{theorem}[Prime number theorem]
  As $x \to \infty$,
  $\pi(x)$ is asymptotic to $x / {\log x}$,
  i.e.
  \[
    \lim_{x \to \infty}
    \frac{\pi(x)}{x / {\log x}} = 1.
  \]
\end{theorem}

\section{Other Open Problems}

\begin{conjecture}[Goldbach]
  Every even integer $\ge 4$ is a sum
  of two primes.
\end{conjecture}

\begin{theorem}[Ternary Goldbach]
  Every odd integer $\ge 7$ is a sum of
  three primes.
\end{theorem}

\begin{remark}
  Goldbach's conjecture implies
  ternary Goldbach (subtract $3$), but not
  vice versa.
\end{remark}

\begin{definition}
  Primes of the form $p = 2^n - 1$
  are called \emph{Mersenne primes}, and
  primes of the form $p = 2^{2^n} + 1$
  are called \emph{Fermat primes}.
\end{definition}

\begin{conjecture}
  There are infinitely many Mersenne
  primes but only finitely many
  Fermat primes.
\end{conjecture}
