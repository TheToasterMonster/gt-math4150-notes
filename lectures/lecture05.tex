\chapter{Sept.~3 --- Congruences}

\begin{quote}
  \emph{No, Tony's the guy with no shins.}
\end{quote}

\section{Dirichlet's Theorem}

\begin{theorem}[Dirichlet's theorem on primes in arithmetic progressions]
  Let $a, b \in \Z$ with
  $a, b > 0$ and $(a, b) = 1$. Then
  the arithmetic progression
  $a, a + b, a + 2b, a + 3b, \dots$
  contains infinitely many primes.
\end{theorem}

\begin{remark}
  Setting $a = b = 1$ recovers the
  fact that there are infinitely many
  primes.
\end{remark}

\begin{remark}
  The general case of Dirichlet's theorem
  is difficult, but we can use the
  fundamental theorem of arithmetic
  to prove some special cases, e.g.
  when $a = 3$ and $b = 4$.
\end{remark}

\begin{lemma}\label{lem:4n+1-product}
  Let $a, b \in \Z$. If $a$ and $b$ are
  expressible as $4n + 1$, then so is
  their product $ab$.
\end{lemma}

\begin{proof}
  Let $a = 4m + 1$ and $b = 4n + 1$. Then
  \[
    ab = (4m + 1)(4n + 1)
    = 16mn + 4m + 4n + 1
    = 4(4mn + m + n) + 1,
  \]
  which proves the desired result.
\end{proof}

\begin{prop}
  There are infinitely many primes
  of the form $4n + 3$ with $n \ge 0$.
\end{prop}

\begin{proof}
  Assume to the contrary that there are
  finitely many primes of the form
  $4n + 3$, say $3, p_1, \dots, p_r$.
  Then consider the integer
  $N = 4p_1 \cdots p_r + 3$. The prime
  factorization of $N$ must contain a
  prime of the form $4n + 3$, since
  otherwise $N$ would be a product of
  primes of the form $4n + 1$, which
  must again be of the form $4n + 1$.
  Thus we have $3 \mid N$ or $p_i \mid N$
  for some $1 \le i \le r$.

  If $3 \mid N$, then $3 \mid N - 3 = 4p_1 \dots p_r$,
  which is a contradiction.
  Otherwise, $p_i \mid N$ for some
  $1 \le i \le r$, and we have
  $p_i \mid N - 4p_1 \cdots p_r = 3$,
  which is a contradiction as well.
\end{proof}

\begin{remark}
  The same proof does not work for
  primes of the form $4n + 1$, since a
  product of numbers of the form
  $4n + 3$ is not necessarily again
  of the form $4n + 3$.
\end{remark}

\section{Congruences}

\begin{definition}
  Let $a, b, m \in \Z$ with $m > 0$.
  Then we say that $a$ is \emph{congruent
  to $b$ modulo $m$}, and we write
  $a \equiv b \Pmod{m}$,
  if $m \mid (a - b)$. The integer
  $m$ is called the \emph{modulus}
  of the congruence. We write
  $a \not\equiv b \Pmod{m}$ if
  $a$ is not congruent to $b$ modulo $m$.
\end{definition}

\begin{example}
  We have $25 \equiv 1 \Pmod{4}$ and
  $25 \equiv 4 \Pmod{7}$.
\end{example}

\begin{prop}\label{prop:congruence-equiv}
  Congruence modulo $m$ is
  an equivalence relation on $\Z$.
\end{prop}

\begin{proof}
  Reflexivity is clear since
  $m \mid 0 = (a - a)$ any $a \in \Z$,
  so $a \equiv a \Pmod{m}$.
  For symmetry, suppose that
  $a \equiv b \Pmod{m}$. Then
  $m \mid a - b$. But then $m \mid (-1)(a - b) = b - a$, so
  $b \equiv a \Pmod{m}$ as well.

  Finally, for transitivity, suppose that
  $a \equiv b \Pmod{m}$ and
  $b \equiv c \Pmod{m}$. Then
  $m \mid a - b$ and $m \mid b - c$, so
  $m$ also divides their sum
  $m \mid (a - b) + (b - c) = a - c$, i.e.
  $a \equiv c \Pmod{m}$.
\end{proof}

\begin{remark}
  A consequence of Proposition
  \ref{prop:congruence-equiv}
  is that $\Z$ is partitioned into
  its equivalence classes under congruence
  modulo $m$. For $a \in \Z$, we
  write $[a]$ to denote the equivalence
  class of $a$ modulo $m$ (not to be
  confused with the floor function).
\end{remark}

\begin{example}
  The equivalence classes of $\Z$
  under congruence modulo $4$ are
  \begin{align*}
    [0]
    &= \{\dots, -8, -4, 0, 4, 8, \dots\}, \\
    [1]
    &= \{\dots, -7, -3, 1, 5, 9, \dots\}, \\
    [2]
    &= \{\dots, -6, -2, 2, 6, 10, \dots\}, \\
    [3]
    &= \{\dots, -5, -1, 3, 7, 11, \dots\}.
  \end{align*}
\end{example}

\begin{definition}
  A set of $m$ integers such that every
  integer is congruent modulo $m$ to
  exactly one integer of the set is
  called a \emph{complete residue system}
  modulo $m$.
\end{definition}

\begin{example}
  $\{0, 1, 2, 3\}$ is a complete
  residue system modulo $4$.
  So is $\{4, 5, -6, -1\}$.
\end{example}

\begin{prop}
  The set $\{0, 1, \dots, m - 1\}$
  is a complete residue system
  modulo $m$.
\end{prop}

\begin{proof}
  First we prove that every integer
  is congruent to one of
  $0, 1, \dots, m - 1$ modulo $m$.
  By the division algorithm, for any
  $a \in \Z$, there exist $q, r \in \Z$
  with $0 \le r \le m - 1$ such that
  $a = qm + r$. Thus we have
  $a - r = qm$, so $m \mid a - r$, i.e.
  $a \equiv r \Pmod{m}$. This proves
  existence since $r \in \{0, 1, \dots, m - 1\}$.

  Now we show uniqueness. Suppose
  $a \equiv r_1 \Pmod{m}$ and
  $a \equiv r_2 \Pmod{m}$ where
  $r_1, r_2 \in \{0, 1, \dots, m - 1\}$.
  By transitivity, we have
  $r_1 \equiv r_2 \Pmod{m}$, so
  $m \mid r_1 - r_2$. But
  $0 \le r_1, r_2 \le m - 1$, so
  \[
    -(m - 1) \le r_1 - r_2 \le m - 1,
  \]
  so we must have $r_1 - r_2 = 0$, i.e.
  $r_1 = r_2$. This proves uniqueness.
\end{proof}

\begin{definition}
  The set $\{0, 1, \dots, m - 1\}$ is
  called the set of \emph{least nonnegative residues}
  modulo $m$.
\end{definition}

\begin{prop}
  Let $a, b, c, d, m \in \Z$, $m > 0$
  such that $a \equiv b \Pmod{m}$
  and $c \equiv d \Pmod{m}$. Then
  \begin{enumerate}
    \item $a + c \equiv b + d \Pmod{m}$;
    \item $ac \equiv bd \Pmod{m}$.
  \end{enumerate}
\end{prop}

\begin{proof}
  Since $a \equiv b \Pmod{m}$ and
  $c \equiv d \Pmod{m}$, we have
  $m \mid b - a$ and $m \mid d - c$.
  Then $m$ divides
  \[
    (b - a) + (d - c)
    = (b + d) - (a + c),
  \]
  so we have $a + c \equiv b + d \Pmod{m}$.
  This proves (1).

  To prove (2), note that
  since $m \mid a - b$, we also have
  $m \mid c(a - b)$. Likewise,
  $m \mid d - c$ implies $m \mid b(d - c)$.
  Then $m$ divides the difference
  \[
    c(a - b) - b(d - c)
    = ac - bd,
  \]
  which shows that $ac \equiv bd \Pmod{m}$.
  This shows (2).
\end{proof}

\begin{remark}
  This shows that the congruence classes
  of $\Z$ modulo $m$ form a \emph{ring}.
\end{remark}

\begin{example}
  Consider the complete residue system
  $\{0, 1, 2, 3\}$ modulo $4$. Their
  squares mod $4$ are
  \[
    \{0^2, 1^2, 2^2, 3^2\}
    \equiv \{0, 1, 0, 1\}
    \equiv \{0, 1\} \pmod{4}.
  \]
\end{example}
