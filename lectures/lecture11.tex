\chapter{Sept.~29 --- Arithmetic Functions, Part 2}

\section{The Divisor Function}

\begin{definition}
  Let $n \in \Z$. The \emph{number of
  positive divisors} of $n$, denoted
  $\tau(n)$, is defined by
  \[
    \tau(n) =
    \#\{d \in \Z : d > 0, d \mid n\}.
  \]
\end{definition}

\begin{theorem}
  $\tau(n)$ is multiplicative.
\end{theorem}

\begin{proof}
  Observe that
  $\tau(n) = \sum_{d \mid n} 1$ and
  $1$ is multiplicative, so
  the result follows from
  Theorem \ref{thm:sum-mult}.
\end{proof}

\begin{remark}
  Since $\tau(n)$ is multiplicative,
  it is determined by its behavior on
  prime powers.
\end{remark}

\begin{theorem}\label{thm:tau-prime-power}
  Let $p$ be prime and let
  $a \in \Z$, $a > 0$. Then
  $\tau(p^a) = a + 1$.
\end{theorem}

\begin{proof}
  The divisors of $p^a$ are
  exactly the integers
  $1, p, p^2, \dots, p^a$. There
  are $a + 1$ of these.
\end{proof}

\begin{theorem}
  Let $n = p_1^{a_1} \cdots p_r^{a_r}$
  with $p_1, \dots, p_r$ distinct
  primes and $a_1, \dots, a_r$ positive
  integers. Then
  \[
    \tau(n) = \prod_{i = 1}^r (a_i + 1).
  \]
\end{theorem}

\begin{proof}
  This follows from
  $\tau$ being multiplicative and
  Theorem \ref{thm:tau-prime-power}.
\end{proof}

\begin{remark}
  For some interesting reading about $\tau(n)$,
  see \emph{Dirichlet's
  divisor problem}.
\end{remark}

\begin{example}
  Consider $504 = 2^3 \cdot 3^2 \cdot 7$.
  Then
  $\tau(504) = (3 + 1)(2 + 1)(1 + 1) = 24$.
\end{example}

\section{The Sum of Divisors Function}

\begin{definition}
  Let $n \in \Z$, $n > 0$. The
  \emph{sum of divisors function},
  denoted $\sigma(n)$, is defined by
  \[
    \sigma(n) = \sum_{d \mid n} d.
  \]
\end{definition}

\begin{theorem}
  $\sigma(n)$ is multiplicative.
\end{theorem}

\begin{proof}
  This follows from
  $f(d) = d$ being multiplicative
  and Theorem \ref{thm:sum-mult}.
\end{proof}

\begin{theorem}
  Let $p$ be a prime and $a \in \Z$,
  $a > 0$. Then
  $\sigma(p^a) = (p^{a + 1} - 1) / (p - 1)$.
\end{theorem}

\begin{proof}
  The divisors of $p^a$ are
  $1, p, p^2, \dots, p^a$, so
  \[
    \sigma(p^a)
    = 1 + p + p^2 + \cdots + p^a
    = \frac{p^{a + 1} - 1}{p - 1}
  \]
  by the formula for a (finite)
  geometric series.
\end{proof}

\begin{theorem}
  Let $n = p_1^{a_1} \cdots p_r^{a_r}$
  with $p_1, \dots, p_r$ distinct
  primes and $a_1, \dots, a_r$ positive
  integers. Then
  \[
    \sigma(n) = \prod_{i = 1}^r
    \frac{p_i^{a_i + 1} - 1}{p_i - 1}.
  \]
\end{theorem}

\begin{example}
  Consider $504 = 2^3 \cdot 3^2 \cdot 7$.
  Then
  \[
    \sigma(504)
    = \frac{2^{3 + 1} - 1}{2 - 1}
    \cdot \frac{3^{2 + 1} - 1}{3 - 1}
    \cdot \frac{7^{1 + 1} - 1}{7 - 1}
    = 15 \cdot 13 \cdot 8 = 1560.
  \]
\end{example}

\section{Perfect Numbers}

\begin{definition}
  Let $n \in \Z$, $n > 0$. Then
  $n$ is a \emph{perfect number} if
  $\sigma(n) = 2n$, or
  $\sigma(n) - n = n$.
\end{definition}

\begin{remark}
  Note that $\sigma(n) - n$ is the
  sum of \emph{proper} divisors of $n$, so
  perfect numbers are those that
  equal the sum of their proper
  divisors.
\end{remark}

\begin{example}
  $6$ and $28$ are perfect numbers.
\end{example}

\begin{conjecture}
  There are infinitely many perfect
  numbers.
\end{conjecture}

\begin{conjecture}
  All perfect numbers are even.
\end{conjecture}

\begin{theorem}\label{thm:even-perfect}
  Let $n \in \Z$, $n > 0$.
  Then $n$ is an even perfect number
  if and only if
  \[
    n = 2^{p - 1}(2^p - 1)
  \]
  for some prime $p$, and $2^p - 1$
  is prime (i.e. $2^p - 1$ is
  a Mersenne prime).
\end{theorem}

\begin{proof}
  $(\Rightarrow \text{Euler})$
  Assume $n$ is an even perfect number.
  Then we can write $n = 2^a b$ with $a, b \in \Z$,
  $a \ge 1$, and $b$ odd. Then we have
  \[
    \sigma(2^a b)
    = \sigma(2^a) \sigma(b)
    = (2^{a + 1} - 1) \sigma(b).
  \]
  Also, since $n$ is perfect, we
  can also write
  $\sigma(2^a b) = 2 \cdot 2^a b = 2^{a + 1} b$,
  thus
  \[
    (2^{a + 1} - 1) \sigma(b)
    = 2^{a + 1} b. \tag{$1$}
  \]
  This implies that $2^{a + 1} \mid (2^{a + 1} - 1) \sigma(b)$,
  so $2^{a + 1} \mid \sigma(b)$
  since $(2^{a + 1}, 2^{a + 1} - 1) = 1$.
  Then $\sigma(b) = 2^{a + 1} c$ (2) for
  some integer $c \ge 1$. Substituting
  this into $(1)$, we get
  \[
    (2^{a + 1} - 1) 2^{a + 1} c
    = 2^{a + 1} b,
  \]
  so we have $(2^{a + 1} - 1) c = b$
  (3).
  We now show that $c = 1$.
  Suppose to the contrary that $c > 1$.
  Then (3) implies that $b$ has at least
  $3$ distinct divisors, namely
  $1, b, c$. Then
  \[
    \sigma(b) \ge 1 + b + c.
  \]
  But (2) implies
  $\sigma(b) = 2^{a + 1} c = (2^{a + 1} - 1 + 1) c = (2^{a + 1} - 1)c + c = b + c$
  by (3), a contradiction.
  So $c = 1$, and by (3), $b = 2^{a + 1} - 1$.
  Also, (2) implies $\sigma(b) = b + 1$,
  so $b$ must be prime.
  One can show that $2^{a + 1} - 1$
  being prime implies $a + 1$ is prime,
  so $n = 2^a (2^{a + 1} - 1)$ with
  $2^{a + 1} - 1$ and $a + 1$ prime.

  $(\Leftarrow \text{Euclid})$
  Assume that $n = 2^{p - 1}(2^p - 1)$
  with $p$ and $2^p - 1$ both prime.
  Then
  \[
    \sigma(2^{p - 1}(2^p - 1))
    = \sigma(2^{p - 1}) \sigma(2^p - 1)
    = (2^p - 1)(2^p - 1 + 1)
    = (2^p - 1) 2^p
    = 2 \cdot 2^{p - 1}(2^p - 1),
  \]
  which shows that $\sigma(n) = 2n$.
  Thus $n$ is perfect.
\end{proof}

\begin{remark}
  Theorem \ref{thm:even-perfect} gives a characterization of
  even perfect numbers and a bijection
  between even perfect numbers and
  Mersenne primes.
\end{remark}

\begin{example}
  The first $5$ perfect numbers
  correspond to $p = 2, 3, 5, 7, 13$.
\end{example}

\section{The M\"obius Function}

\begin{definition}
  Let $n \in \Z$, $n > 0$. The
  \emph{M\"obius function}, denoted
  $\mu(n)$, is defined by
  \[
    \mu(n) =
    \begin{cases}
      1 & \text{if } n = 1, \\
      0 & \text{if } p^2 \mid n \text{ for some } p, \\
      (-1)^r & \text{if $n = p_1, \dots, p_r$ with $p_i$ distinct primes}.
    \end{cases}
  \]
\end{definition}

\begin{example}
  Since $504 = 2^3 \cdot 3^2 \cdot 7$,
  we have $\mu(504) = 0$. On the other
  hand,
  \[\mu(6) = (-1)^2 = 1 \quad \text{and}\quad
  \mu(30) = (-1)^3 = -1\]
\end{example}

\begin{theorem}
  $\mu(n)$ is multiplicative.
\end{theorem}

\begin{proof}
  Let $m, n$ be relatively prime
  positive integers. We need to show
  that $\mu(mn) = \mu(m) \mu(n)$.
  This is clear if $m = 1$ or $n = 1$,
  so we may assume $m, n > 1$.
  Note that $m$ or $n$ is divisible
  by a square if and only if $mn$
  is divisible by a square, since
  $(m, n) = 1$. In this case, both
  $\mu(m) \mu(n)$ and $\mu(mn)$ are $0$.

  Now suppose $m, n$ are products
  of distinct primes, say
  $m = p_1 \cdots p_r$ and
  $n = q_1 \cdots q_s$. Since
  $(m, n) = 1$, we have $p_i \ne q_j$
  for any $1 \le i \le r$ and
  $1 \le j \le s$. Thus
  \[
    \mu(mn)
    = \mu(p_1 \cdots p_r q_1 \cdots q_s)
    = (-1)^{r + s}
    = (-1)^r (-1)^s
    = \mu(p_1 \cdots p_r) \mu(q_1 \cdots q_s)
    = \mu(m) \mu(n),
  \]
  as desired. So $\mu$ is multiplicative.
\end{proof}

\begin{prop}
  Let $n \in \Z$, $n > 0$. Then
  \[
    \sum_{d \mid n} \mu(d)
    =
    \begin{cases}
      1 & \text{if } n = 1, \\
      0 & \text{if } n > 1.
    \end{cases}
  \]
\end{prop}

\begin{proof}
  Since $\mu$ is multiplicative,
  so is $F(n) = \sum_{d \mid n} \mu(d)$
  by Theorem \ref{thm:sum-mult}. Thus
  it suffices to show that
  $F(p^a) = 0$ for prime powers $p^a$.
  We have
  \[
    F(p^a)
    = \sum_{d \mid p^a} \mu(d)
    = \mu(1) + \mu(p) + \mu(p^2) + \cdots + \mu(p^a).
  \]
  Note that $p^2 \mid p^j$ for $j \ge 2$,
  so $\mu(p^j) = 0$ for $j \ge 2$.
  Thus
  \[
    F(p^a) = \mu(1) + \mu(p) = 1 - 1 = 0.
  \]
  It is clear that
  $F(1) = \sum_{d \mid 1} \mu(d) = \mu(1) = 1$, so
  the result follows.
\end{proof}

\begin{example}
  Let $n = 12$. Then we have
  \begin{align*}
    \sum_{d \mid 12} \mu(d)
    &= \mu(1) + \mu(2) + \mu(3) + \mu(4) + \mu(6) + \mu(12) \\
    &= 1 - 1 - 1 + 0 + 1 + 0
    = 0.
  \end{align*}
\end{example}
