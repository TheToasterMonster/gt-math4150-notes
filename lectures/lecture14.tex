\chapter{Oct.~13 --- Quadratic Reciprocity}

\begin{quote}
  \begin{emph}
    Why did the man bring his watch to the
    bank? He wanted to save time.
  \end{emph}
\end{quote}

\section{Applications of Gauss's Lemma}

\begin{exercise}
  Calculate the following:
  \[
    \left(\frac{-1}{13}\right), \quad
    \left(\frac{2}{17}\right), \quad
    \left(\frac{-14}{11}\right), \quad
    \left(\frac{18}{23}\right).
  \]
  For the first, since
  $13 \equiv 1 \Pmod{4}$, we have
  $\big(\frac{-1}{13}\big) = 1$.
  For the second, we compute
  \[
    \{2, 2(2), 3(2), 4(2), 5(2), 6(2), 7(2), 8(2)\}
    \equiv \{2, 4, 6, 8, 10, 12, 14, 16\}
    \pmod{17}.
  \]
  Four of these residues are
  greater that $17 / 2 = 8.5$, so
  $\big(\frac{2}{17}\big) = (-1)^4 = 1$.
  For the third, write
  \[
    \left(\frac{-14}{11}\right)
    = \left(\frac{-1}{11}\right)
    \left(\frac{14}{11}\right)
    = \left(\frac{-1}{11}\right)
    \left(\frac{3}{11}\right)
    = -\left(\frac{3}{11}\right)
  \]
  since $11 \equiv 3 \Pmod{4}$.
  To compute $\big(\frac{3}{11}\big)$,
  we list
  \[
    \{3, 2(3), 3(3), 4(3), 5(3)\}
    \equiv \{3, 6, 9, 1, 4\}
    \pmod{11}.
  \]
  Two of the above are greater than $11 / 2 = 5.5$,
  so $\big(\frac{3}{11}\big) = (-1)^2 = 1$.
  Thus $\big(\frac{-14}{11}\big) = -1$.
  For the last,
  \[
    \left(\frac{18}{23}\right)
    = \left(\frac{2}{23}\right)
    \left(\frac{9}{23}\right)
    = \left(\frac{2}{23}\right),
  \]
  which we can compute by enumerating
  \begin{align*}
    &\{2, 2(2), 3(2), 4(2), 5(2), 6(2), 7(2), 8(2), 9(2), 10(2), 11(2)\} \\
    &\quad \quad \equiv \{2, 4, 6, 8, 10, 12, 14, 16, 18, 20, 22\}
    \pmod{23}.
  \end{align*}
  Six of the above are greater than
  $23 / 2 = 11.5$, so
  $\big(\frac{18}{23}\big) = \big(\frac{2}{23}\big) = (-1)^6 = 1$.
\end{exercise}

\begin{theorem}
  Let $p$ be an odd prime. Then
  \[
    \left(\frac{2}{p}\right)
    = (-1)^{(p^2 - 1) / 8}
    =
    \begin{cases}
      1 & \text{if } p \equiv 1, 7 \Pmod{8}, \\
      -1 & \text{if } p \equiv 3, 5 \Pmod{8}.
    \end{cases}
  \]
\end{theorem}

\begin{proof}
  By Gauss's lemma, we have
  $\big(\frac{2}{p}\big) = (-1)^n$,
  where $n$ is the number of least positive
  residues of
  \[
    2, \quad 2(2), \quad 3(2), \quad 4(2), \quad \dots, \quad
    \left(\frac{p - 1}{2}\right) 2.
  \]
  Let $k \in \Z$ with $1 \le k \le (p - 1) / 2$.
  Note that $2k < p / 2$ if and only if
  $k < p / 4$ (we always have $2k < p$),
  so there are $\lfloor p / 4 \rfloor$
  values of $k$ for which $2k < p / 2$.
  Thus, there are $(p - 1) / 2 - \lfloor p / 4 \rfloor$
  values of $k$ for which
  $2k > p / 2$ (recall that $p$ is odd),
  so $n = (p - 1) / 2 - \lfloor p / 4 \rfloor$.
  To show that $(p^2 - 1) / 8$
  and $(p - 1) / 2 - \lfloor p / 4 \rfloor$
  always have the same parity, we can
  just check the four cases:
  \begin{itemize}
    \item $p \equiv 1 \Pmod{8}$.
      Then $p = 8m + 1$ for some $m \in \Z$.
      Then
      \[
        n = \frac{p - 1}{2} - \left\lfloor \frac{p}{4} \right\rfloor
        = \frac{8m + 1 - 1}{2} - \left\lfloor \frac{8m + 1}{4} \right\rfloor
        = 4m - 2m = 2m.
      \]
      Note that this is even.
      On the other hand, we can check that
      \[
        \frac{p^2 - 1}{8}
        = \frac{(8m + 1)^2 - 1}{8}
        = 8m^2 + 2m.
      \]
      This is also even, so
      the parity matches in this case.
    \item Check the cases
      $p \equiv 3, 5, 7 \Pmod{8}$ similarly
      as an exercise.
  \end{itemize}
  Since $(p^2 - 1) / 8$ and $n$ agree
  modulo $2$, we have
  $(-1)^{(p^2 - 1) / 8} = (-1)^n = \big(\frac{2}{p}\big)$.
\end{proof}

\begin{example}
  We have $\big(\frac{2}{23}\big) = 1$
  since $23 \equiv 7 \Pmod{8}$.
\end{example}

\section{Quadratic Reciprocity}

\begin{remark}
  We will now try to understand
  $\big(\frac{q}{p}\big)$ for distinct
  odd primes $p, q$.
\end{remark}

\begin{theorem}[Law of quadratic reciprocity]\label{thm:quadratic-reciprocity}
  Let $p, q$ be distinct odd primes. Then
  \[
    \left(\frac{p}{q}\right)
    \left(\frac{q}{p}\right)
    = (-1)^{(p - 1)(q - 1) / 4}
    =
    \begin{cases}
      1 & \text{if } p \equiv 1 \text{ or } q \equiv 1 \Pmod{4}, \\
      -1 & \text{if } p \equiv q \equiv 3 \Pmod{4}.
    \end{cases}
  \]
\end{theorem}

\begin{remark}
  Quadratic reciprocity allows us to
  simplify the calculation for
  $\big(\frac{p}{q}\big)$.
  For example, consider
  \begin{quote}
    Which primes are quadratic residues modulo
    $17$, i.e. evaluate $\big(\frac{p}{17}\big)$?
  \end{quote}
  This is a finite problem: We may just
  compute all squares modulo $17$.
  Now consider
  \begin{quote}
    For which primes $p$ is $17$ a
    quadratic residue, i.e. evaluate
    $\big(\frac{17}{p}\big)$?
  \end{quote}
  This is a priori an infinite problem,
  but we can convert it to the previous one
  by quadratic reciprocity.
\end{remark}

\begin{example}
  Compute $\big(\frac{7}{53}\big)$.
  We use quadratic reciprocity:
  $7 \equiv 3 \Pmod{4}$ and
  $53 \equiv 1 \Pmod{4}$, so
  \[
    \left(\frac{7}{53}\right)
    =
    \left(\frac{53}{7}\right)
    =
    \left(\frac{4}{7}\right)
    = 1
  \]
  since $4$ is always a square modulo any
  prime.
\end{example}

\begin{example}
  Calculate $\big(\frac{-158}{101}\big)$.
  We can first write
  \[
    \left(\frac{-158}{101}\right)
    = \left(\frac{-1}{101}\right)
    \left(\frac{158}{101}\right)
    = \left(\frac{158}{101}\right)
    = \left(\frac{57}{101}\right)
    = \left(\frac{3}{101}\right)
    \left(\frac{19}{101}\right)
  \]
  since $101 \equiv 1 \Pmod{4}$,
  $158 \equiv 57 \Pmod{101}$, and
  $57 = 3 \cdot 19$. We can now
  apply quadratic reciprocity (note that
  we could not have done this earlier,
  since $158, 57$ are not prime):
  \[
    \left(\frac{-158}{101}\right)
    = \left(\frac{101}{3}\right)
    \left(\frac{101}{19}\right)
    = \left(\frac{2}{3}\right)
    \left(\frac{6}{19}\right)
    = \left(\frac{2}{3}\right)\left(\frac{25}{19}\right)
    = -1 \cdot 1 = 1.
  \]
\end{example}

\begin{lemma}\label{lem:qr-lemma}
  Let $p$ be an odd prime number and
  let $a \in \Z$, $p \nmid a$, $a$ odd.
  Let
  \[
    N = \sum_{j = 1}^{(p - 1) / 2}
    \left\lfloor \frac{ja}{p} \right\rfloor.
  \]
  Then $\big(\frac{a}{p}\big) = (-1)^N$.
\end{lemma}

\begin{proof}
  Let $r_1, r_2, \dots, r_n$ be the least
  non-negative residues among
  $a, 2a, \dots, ((p - 1) / 2) a$
  that are $> p / 2$. Likewise, let
  $s_1, \dots, s_m$ be the remaining
  residues that are $< p / 2$. Note that
  \[
    r_1, \dots, r_n, s_1, \dots, s_m
  \]
  are all distinct modulo $p$ (they come
  from $a, 2a, 3a, \dots , ((p - 1) / 2)a$,
  which are distinct since $p \nmid a$).
  This means that the fractions
  $r_i / p$, $s_j / p$  are also all
  distinct. Then
  \[
    ja
    = p \cdot \frac{ja}{p}
    = p \left(\left\lfloor \frac{ja}{p} \right\rfloor + \frac{\text{remainder}}{p}\right)
    = p \left\lfloor \frac{ja}{p} \right\rfloor + \text{remainder depending on $j$},
  \]
  where the remainders are exactly the
  numbers $r_1, \dots, r_n, s_1, \dots, s_m$.
  Then
  \[
    \sum_{j = 1}^{(p - 1) / 2} ja
    = \sum_{j = 1}^{(p - 1) / 2}
    p \left\lfloor \frac{ja}{p} \right\rfloor
    + \sum_{i = 1}^n r_i + \sum_{j = 1}^m s_j.
    \tag{$1$}
  \]
  Note also that
  \[
    \sum_{j = 1}^{(p - 1) / 2} j
    = \sum_{i = 1}^n (p - r_i) + \sum_{j = 1}^m s_j
    = pn - \sum_{i = 1}^n r_i + \sum_{j = 1}^m s_j.
    \tag{$2$}
  \]
  Subtracting $(2)$ from $(1)$ gives the
  equation
  \[
    \sum_{j = 1}^{(p - 1) / 2} j(a - 1)
    = \sum_{j = 1}^{(p - 1) / 2}
    p \left\lfloor \frac{ja}{p} \right\rfloor
    - pn + 2\sum_{i = 1}^n r_i.
  \]
  Taking the above equation modulo $2$,
  since $a$ is odd, we get
  \[
    \sum_{j = 1}^{(p - 1) / 2} p \left\lfloor \frac{ja}{p} \right\rfloor
    - pn \equiv 0 \Pmod{2},
  \]
  so $pN \equiv pn \Pmod{2}$,
  so $N \equiv n \Pmod{2}$
  since $2 \nmid p$.
  So $(-1)^N = (-1)^n = \big(\frac{a}{p}\big)$
  by Gauss's lemma.
\end{proof}

\begin{example}
  We compute $\big(\frac{7}{11}\big)$
  using Lemma \ref{lem:qr-lemma}.
  We calculate
  \[
    N = \sum_{j = 1}^5 \left\lfloor \frac{7j}{11} \right\rfloor
    = \left\lfloor \frac{7}{11} \right\rfloor
    + \left\lfloor \frac{14}{11} \right\rfloor
    + \left\lfloor \frac{21}{11} \right\rfloor
    + \left\lfloor \frac{28}{11} \right\rfloor
    + \left\lfloor \frac{35}{11} \right\rfloor
    = 0 + 1 + 1 + 2 + 3 = 7,
  \]
  so $\big(\frac{7}{11}\big) = (-1)^7 = -1$
  by Lemma \ref{lem:qr-lemma}.
\end{example}
