\chapter{Sept.~15 --- Fermat's Little Theorem}

\begin{quote}
  \emph{What do you call it when you have
  your grandmother on speed dial?
  It's an insta gram.}
\end{quote}

\section{Fermat's Little Theorem}

\begin{theorem}[Fermat's little theorem]
  Let $p$ be a prime and $a \in \Z$
  such that $p \nmid a$. Then
  \[
    a^{p - 1} \equiv 1 \pmod{p}.
  \]
\end{theorem}

\begin{proof}
  Consider the $p - 1$ integers
  $a, 2a, 3a, \dots, (p - 1)a$. Note that
  $p \nmid a_i$ for any $1 \le i \le p - 1$.
  Note also that no two of these integers
  are congruent modulo $p$: If $a i \equiv a j \Pmod{p}$
  for some $i \ne j$, then we
  can multiply by the inverse $\overline{a}$
  of $a$ (which exists since $p \nmid a$)
  to get $i \equiv j \Pmod{p}$, which
  is impossible.
  Thus $\{a, 2a \dots, (p - 1)a\}$ is a
  complete nonzero residue system, so
  \[
    a(2a)(3a) \cdots ((p - 1)a)
    \equiv 1 \cdot 2 \cdot 3 \cdots (p - 1)
    \pmod{p}.
  \]
  Then $a^{p - 1}(p - 1)! \equiv (p - 1)! \Pmod{p}$, so
  $a^{p - 1} \equiv 1 \Pmod{p}$
  since $p \nmid (p - 1)!$.
\end{proof}

\begin{corollary}
  Let $p$ be prime and $a \in \Z$ with
  $p \nmid a$. Then
  $a^{p - 2}$ is the inverse of $a$
  modulo $p$.
\end{corollary}

\begin{proof}
  By Fermat's little theorem,
  $a \cdot a^{p - 2} = a^{p - 1} \equiv 1 \Pmod{p}$.
\end{proof}

\begin{corollary}
  Let $p$ be prime and $a \in \Z$.
  Then $a^p \equiv a \Pmod{p}$.
\end{corollary}

\begin{proof}
  If $p \mid a$, then both
  sides are congruent to
  $0$ modulo $p$. Otherwise, if
  $p \nmid a$, then we can write
  $a^p = a \cdot a^{p - 1} \equiv a \cdot 1 = a \Pmod{p}$
  by Fermat's little theorem.
\end{proof}

\begin{corollary}
  Let $p$ be prime. Then
  $2^p \equiv 2 \Pmod{p}$.
\end{corollary}

\begin{definition}
  If $n \in \Z$ is composite and
  $2^n \equiv 2 \Pmod{n}$,
  then $n$ is called a \emph{pseudoprime}.
\end{definition}

\begin{remark}
  It is known that there are infinitely
  many (even and odd) pseudoprimes.
\end{remark}

\begin{example}
  Consider $n = 341 = 11 \cdot 31$.
  To prove that $2^{341} \equiv 2 \Pmod{341}$,
  it suffices to show that
  $2^{341} \equiv 2 \Pmod{11}$
  and $2^{341} \equiv 2 \Pmod{31}$
  by the Chinese remainder theorem.
  Note that
  \begin{align*}
    2^{341}
    = (2^{10})^{34} \cdot 2 \equiv
    1^{34} \cdot 2 &= 2 \pmod{11} \\
    2^{341} = (2^{30})^{11} \cdot 2^{11}
    \equiv 1^{11} \cdot (2^5)^2 \cdot 2
    \equiv 1^2 \cdot 2 &= 2 \pmod{31}
  \end{align*}
  by Fermat's little theorem, so
  $341$ is a pseudoprime.
\end{example}

\section{Euler's Theorem}

\begin{definition}
  Let $n \in \Z$, $n > 0$.
  \emph{Euler's phi function}, denoted
  $\varphi(n)$, is the number of
  positive integers $\le n$ that are
  relatively prime to $n$. In other words,
  \[
    \varphi(n)
    = \#\{m \in \Z : 1 \le m \le n, (m, n) = 1\}.
  \]
\end{definition}

\begin{example}
  We have $\varphi(4) = 2$,
  $\varphi(14) = 6$, and
  $\varphi(p) = p - 1$ for any prime $p$.
\end{example}

\begin{theorem}[Euler's theorem]
  Let $a, m \in \Z$ with $m > 0$. If
  $(a, m) = 1$, then
  \[
    a^{\varphi(m)} \equiv 1 \pmod{m}.
  \]
\end{theorem}

\begin{proof}
  Let $r_1, r_2, \dots, r_{\varphi(m)}$
  be the distinct positive integers
  not exceeding $m$ such that
  $(r_i, m) = 1$. Then consider the
  integers $a r_1, a r_2, \dots, a r_{\varphi(m)}$.
  Note first that
  $(a r_i, m) = 1$ since
  $(r_i, m) = 1$ and $(a, m) = 1$
  by assumption. Note also that
  $a r_i \not\equiv a r_j \Pmod{m}$
  for $i \ne j$ since
  $\overline{a}$ exists (since
  $(a, m) = 1$), and multiplying
  by $\overline{a}$ implies
  $r_i \equiv r_j \Pmod{m}$, which is
  impossible. Thus the least
  nonnegative residues of
  $\{a r_1, \dots, a r_{\varphi(m)}\}$
  coincide with
  $\{r_1, \dots, r_{\varphi(m)}\}$, so
  we have
  \[
    (a r_1)(a r_2) \cdots (a r_{\varphi(m)})
    = r_1 r_2 \cdots r_{\varphi(m)} \pmod{m},
  \]
  thus $a^{\varphi(m)} (r_1 \cdots r_{\varphi(m)}) \equiv (r_1 \cdots r_{\varphi(m)}) \Pmod{m}$.
  Since $(r_1 \cdots r_{\varphi(m)}, m) = 1$,
  the inverse of $r_1 \cdots r_{\varphi(m)}$
  modulo $m$ exists, and
  multiplying by the inverse
  gives $a^{\varphi(m)} \equiv 1 \Pmod{m}$.
\end{proof}

\begin{remark}
  Taking $m = p$ recovers Fermat's little theorem
  since $\varphi(p) = p - 1$ for prime $p$.
\end{remark}

\begin{definition}
  Let $m$ be a positive integer.
  A set of $\varphi(m)$ integers
  such that each integer is relatively
  prime to $m$ and no two are
  congruent modulo $m$ is called
  a \emph{reduced residue system}
  modulo $m$.
\end{definition}

\begin{example}
  $\{1, 5, 7, 11\}$ is a reduced
  residue system modulo $12$.
  So is
  \[
    \{5(1), 5(5), 5(7), 5(11)\}
    = \{5, 25, 35, 55\}.
  \]
  For a prime $p$, the set
  $\{1, 2, \dots, p - 1\}$ is always a
  reduced residue system modulo $p$.
\end{example}

\begin{corollary}
  Let $a, m \in \Z$ with $m > 0$
  and $(a, m) = 1$. Then
  $\overline{a} \equiv a^{\varphi(m) - 1} \Pmod{m}$.
\end{corollary}

\section{Arithmetic Functions and Multiplicativity}

\begin{definition}
  An \emph{arithmetic function} is a
  function whose domain is the set of
  positive integers.
\end{definition}

\begin{example}\label{ex:arith-funcs}
  The following are examples of
  arithmetic functions:
  \begin{enumerate}
    \item Euler's $\varphi$ function;
    \item $v(n)$, the number of
      positive divisors of $n$;
    \item $\sigma(n)$, the sum of
      the positive divisors of $n$;
    \item $\omega(n)$, the number of
      distinct prime factors of $n$;
    \item $p(n)$, the number of
      integer partitions of $n$;
    \item $\Omega(n)$, the number of
      total prime factors (counted with
      multiplicity) of $n$.
  \end{enumerate}
\end{example}

\begin{definition}
  An arithmetic function
  $f$ is \emph{multiplicative} if
  $f(mn) = f(m) f(n)$ whenever
  $(m, n) = 1$.
  We say that $f$ is \emph{completely multiplicative}
  if $f(mn) = f(m) f(n)$ for all
  $m, n$.
\end{definition}

\begin{remark}
  Note that if $n > 1$, then we can write
  $n = p_1^{a_1} \cdots p_r^{a_r}$.
  If $f$ is multiplicative, then
  \[
    f(n) = f(p_1^{a_1} \cdots p_r^{a_r})
    = f(p_1^{a_1}) \cdots f(p_r^{a_r}).
  \]
  So multiplicative functions are
  determined by their values at
  prime powers. If
  $f$ is completely multiplicative,
  then $f(n) = f(p_1)^{a_1} \cdots f(p_r)^{a_r}$
  and $f$ is determined by its values
  at primes.
\end{remark}

\begin{example}
  The functions $\varphi, v, \sigma$
  from Example \ref{ex:arith-funcs}
  are multiplicative, while
  $\omega, p, \Omega$ are not.
\end{example}

\begin{example}
  The functions $f(n) = 1$ and $f(n) = 0$
  are completely multiplicative. The
  function $f$ defined by
  $f(1) = 1$ and $f(n) > 0$ if $n > 1$
  is also completely multiplicative.
\end{example}

\begin{remark}
  If $f$ is multiplicative and not
  identically zero, then $f(1) = 1$.
  To see this, take $n$ such
  that $f(n) \ne 0$ (since $f$ is not
  identically zero). Then
  $f(n) = f(n \cdot 1) = f(n) f(1)$, so
  $f(1) = 1$ since $f(n) \ne 0$.
\end{remark}
