\chapter{Aug.~18 --- Divisibility}

\begin{quote}
  \emph{Something something pair a' docks.} (I forgot to write it down oops.)
\end{quote}

\section{Basic Properties of Divisibility}

\begin{definition}
  Let $a, b \in \Z$. We say that
  $a$ \emph{divides} $b$, and we write
  $a \mid b$, if there exists $c \in \Z$
  such that $b = ac$. We also say that
  $a$ is a \emph{divisor} (or \emph{factor})
  of $b$. We write
  $a \nmid b$ if $a$ does not divide $b$.
\end{definition}

\begin{example} We have the following:
  \begin{enumerate}
    \item We have $3 \mid 6$ since
      $6 = 3 \cdot 2$, and $3 \mid -6$ since
      $-6 = 3 \cdot (-2)$.
    \item For any $a \in \Z$, we have $a \mid 0$
      since $0 = a \cdot 0$.
    \item Technically, we have
      $0 \mid 0$, but do not confuse this
      with the indeterminate form $0 / 0$.
  \end{enumerate}
\end{example}

\begin{prop}
  Let $a, b, c \in \Z$. If $a \mid b$ and
  $b \mid c$, then $a \mid c$. In particular,
  divisibility is transitive.
\end{prop}

\begin{proof}
  Since $a \mid b$ and $b \mid c$, there
  exist integers $e, f$ such that
  $b = ae$ and $c = bf$. We can write
  \[
    c = bf = (ae)f = a(ef),
  \]
  so that $a$ divides $c$ by definition.
\end{proof}

\begin{prop}
  Let $a, b, c, m, n \in \Z$. If $c \mid a$
  and $c \mid b$, then $c \mid (am + bn)$. In
  other words, $c$ divides any integral
  linear combination of $a$ and $b$.
\end{prop}

\begin{proof}
  Since $c \mid a$ and $c \mid b$, we have
  $a = ce$ and $b = cf$ for some $e, f \in \Z$.
  Then
  \[
    am + bn = (ce)m + (cf)n
    = c(em + fn),
  \]
  so that $c$ divides $am + bn$ by definition.
\end{proof}

\section{The Division Algorithm}

\begin{definition}
  Let $x \in \R$. The
  \emph{greatest integer function} (or
  \emph{floor function}) of
  $x$, denoted $[x]$ (or
  $\lfloor x \rfloor$),
  is the greatest integer less than or equal
  to $x$.
\end{definition}

\begin{example}
  We have the following:
  \begin{enumerate}
    \item If $a \in \Z$, then $[a] = a$.
      The converse is also true:
      If $[a] = a$ for $a \in \R$, then
      $a \in \Z$.
    \item We have
      $[\pi] = 3$, $[e] = 2$, $[-1.5] = -2$,
      and $[-\pi] = -4$.
  \end{enumerate}
\end{example}

\begin{lemma}\label{lem:floor-bound}
  Let $x \in \R$. Then $x - 1 < [x] \le x$.
\end{lemma}

\begin{proof}
  The upper bound is obvious. To show the
  lower bound, suppose to the contrary that
  $[x] \le x - 1$. Then $[x] < [x] + 1 \le x$,
  which contradicts the maximality of $[x]$
  as $[x] + 1$ is an integer.
\end{proof}

\begin{example}
  We can write $5 = 3 \cdot 1 + 2$ and
  $26 = 6 \cdot 4 + 2$; this is
  the \emph{division algorithm}.
\end{example}

\begin{theorem}[Division algorithm]
  Let $a, b \in \Z$ with $b > 0$. Then there
  exist unique $q, r \in \Z$ such
  that
  \[
    a = bq + r, \quad 0 \le r < b.
  \]
  Call $q$ the \emph{quotient} and
  $r$ the \emph{remainder} of the division.
\end{theorem}

\begin{proof}
  First we show existence. Let $q = [a / b]$
  and $r = a - b[a / b]$. By construction,
  $a = bq + r$. To check that
  $0 \le r < b$, note that by Lemma
  \ref{lem:floor-bound}, we have
  $a / b - 1 < [a / b] \le a / b$. Multiplying
  by $-b$ gives
  \[
    -a \le -b[a / b] < b - a,
  \]
  and adding $a$ gives the desired inequality
  $0 \le a - b[a / b] = r < b$.

  Now we prove uniqueness. Assume there are
  $q_1, q_2, r_1, r_2 \in \Z$ such that
  \[
    a = bq_1 + r_1 = bq_2 + r_2, \quad
    0 \le r_1, r_2 < b.
  \]
  Then $0 = (bq_1 + r_1) - (bq_2 + r_2) = b(q_1 - q_2) + (r_1 - r_2)$, so we find that
  \[
    r_2 - r_1 = b(q_1 - q_2).
  \]
  So $b \mid r_2 - r_1$. But
  $0 \le r_1, r_2 < b$
  implies $-b < r_2 - r_1 < b$, so
  we must have $r_2 - r_1 = 0$, i.e.
  $r_1 = r_2$. This then implies
  $0 = b(q_1 - q_2)$, which gives
  $q_1 - q_2 = 0$ since $b > 0$, so
  $q_1 = q_2$ as well.
\end{proof}

\begin{remark}
  In the division algorithm, we have $r = 0$
  if and only if $b \mid a$.
\end{remark}

\begin{example}
  Suppose $a = -5$, $b = 3$. Then
  $q = [a / b] = -2$ and
  $r = a - b[a / b] = 1$, i.e.
  \[
    -5 = 3 \cdot (-2) + 1.
  \]
  Note that $-5 = 3 \cdot (-1) + (-2)$ also, but
  this does not contradict uniqueness since
  $-2 \notin [0, 3)$.
\end{example}

\begin{definition}
  Let $n \in \Z$. Then $n$ is \emph{even}
  if $2 \mid n$, and $\emph{odd}$ otherwise.
\end{definition}
