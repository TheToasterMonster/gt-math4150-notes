\chapter{Oct.~27 --- Primitive Roots, Part 3}

\section{Primitive Roots for Primes, Continued}

\begin{theorem}[Legendre]\label{thm:legendre}
  Let $p$ be a prime and let $d \in \Z$ 
  with $d > 0$ and $d \mid p - 1$. Then
  there are exactly $\varphi(d)$ incongruent
  integers having order $d$ modulo $p$.
\end{theorem}

\begin{proof}
  Given $d \mid p - 1$, let $f(d)$ be the
  number of integers among
  $1, 2, \dots, p - 1$ that have order $d$ 
  modulo $p$. We wish to show that
  $f(d) = \varphi(d)$. We will first show
  that if $f(d) \ne 0$, then $f(d) = \varphi(d)$.
  Then we will show that $f(d) \ne 0$ for
  all $d \mid p - 1$.

  Suppose first that $f(d) > 0$. Then
  there exists an integer $a$ with order
  $d$. Note that the integers
  $a^1, a^2, \dots, a^d$ are incongruent
  modulo $p$. To see this, suppose
  otherwise that $a^i \equiv a^j \Pmod{p}$
  for some $i > j$. Then
  $a^{i - j} \equiv 1 \Pmod{p}$
  with $i - j < d$, contradicting
  $\ord_p a = d$. Note also that
  \[
    (a^k)^d \equiv (a^d)^k \equiv 1 \pmod{p},
  \]
  so each is a solution to
  $x^d - 1 \equiv 0 \Pmod{p}$. Since this
  congruence has exactly $d$ solutions,
  $a^1, \dots, a^d$ must be all of them.
  So any integer of order $d$ modulo $p$ 
  must be congruent to one of these.
  In particular, any element
  of order $d$ must be a power of $a$.
  Recall that
  \[
    \ord_p (a^i)
    = \frac{\ord_p a}{(\ord_p a, i)},
  \]
  so $\ord_p (a^i) = d$ if and only if
  $\ord_p(a) / (\ord_p a, i) = d$,
  which holds only when $(d, i) = 1$
  since $\ord_p(a) = d$. Thus there are
  exactly $\varphi(d)$ values of
  $i$, so $f(d) = \varphi(d)$.

  We now show that $f(d)$ cannot be $0$.
  Note that any integer $b$ with
  $1 \le b \le p - 1$ must have order
  dividing $p - 1$, so any such $b$ is
  counted by exactly one $f(d)$. Thus
  \[
    \sum_{d \mid p - 1} f(d)
    = p - 1,
  \]
  so we have $\sum_{d \mid p - 1} f(d) = p - 1 = \sum_{d \mid p - 1} \varphi(d)$.
  Rearranging, we get
  \[
    \sum_{d \mid p - 1} (\varphi(d) - f(d)) = 0.
  \]
  If $f(d) \ne 0$, then $f(d) = \varphi(d)$,
  so the corresponding summand is $0$.
  Then
  \[
    0 = \sum_{\substack{d \mid p - 1 \\ f(d) = 0}} (\varphi(d) - f(d))
    = \sum_{\substack{d \mid p - 1 \\ f(d) = 0}} \varphi(d),
  \]
  which is a sum of positive integers, so
  this can only happen if
  $f(d) \ne 0$ for all $d \mid p - 1$.
\end{proof}

\begin{corollary}
  Let $p$ be a prime. Then there are
  exactly $\varphi(p - 1)$ primitive
  roots modulo $p$.
\end{corollary}

\begin{remark}
  Note that Theorem \ref{thm:legendre}
  gives no way to construct a primitive
  root.
\end{remark}

\begin{example}
  Let $p = 7$. Theorem \ref{thm:legendre}
  implies that there exist residues
  of orders $1, 2, 3, 6$ since
  $\varphi(7) = 6$. Further, we can
  compute the following table:
  \begin{center}
    \begin{tabular}{c|c|c}
      order & \# residues & residues \\
      \hline
      1 & $\varphi(1) = 1$ & $1$\\
      2 & $\varphi(2) = 1$ & $6$\\
      3 & $\varphi(3) = 2$ & $2, 4$ \\
      6 & $\varphi(6) = 2$ & $3, 5$
    \end{tabular}
  \end{center}
\end{example}

\begin{exercise}
  Construct the analogous table for $p = 13$.
  The solution is:
  \begin{center}
    \begin{tabular}{c|c|c}
      order & \# residues & residues \\
      \hline
      1 & $\varphi(1) = 1$ & $1$\\
      2 & $\varphi(2) = 1$ & $12$\\
      3 & $\varphi(3) = 2$ & $3, 9$ \\
      4 & $\varphi(4) = 2$ & $8, 5$ \\
      6 & $\varphi(6) = 2$ & $4, 10$ \\
      12 & $\varphi(12) = 4$ & $2, 6, 11, 7$
    \end{tabular}
  \end{center}
\end{exercise}

\begin{example}
  Find all incongruent integers having
  orders $6$ and $7$ modulo $19$.
  Immediately we know that there are
  $0$ integers of order $7$ modulo
  $19$ since $7 \nmid \varphi(19) = 18$.
  To find the elements of order $6$,
  we would like to have a primitive root.
  We check that $2$ is a primitive
  root:
  \[
    2^1 = 2, \quad
    2^2 = 4, \quad 2^3 = 8, \quad
    2^6 = 7, \quad 2^9
    = -1, \quad 2^{18} = 1.
  \]
  Now to find the integers of order $6$,
  we solve the equation
  \[
    6 = \ord_{19}(2^a)
    = \frac{\ord_{19} 2}{(\ord_{19} 2, a)}
    = \frac{18}{(18, a)},
  \]
  so $(18, a) = 18 / 6 = 3$.
  Thus $a = 3, 15$,
  which corresponds to
  $2^3 = 8$ and $2^{15} = (2^7)^2 \cdot 2 = 3$ modulo $19$.
\end{example}

\begin{remark}
  The frequency with which $2$ appears
  as a primitive root motivates the
  following conjecture:
  \begin{quote}
    \vspace{-2em}
    \begin{conjecture}
      There are infinitely many primes $p$
      for which $2$ is a primitive root
      modulo $p$.
    \end{conjecture}
  \end{quote}
  This conjecture is still open.
  A generalization of the above is
  the following:
  \begin{quote}
    \vspace{-2em}
    \begin{conjecture}[Artin]
      If $r$ is any non-square integer
      other than $-1$, then there are
      infinitely many primes $p$ for which
      $r$ is a primitive root modulo $p$.
    \end{conjecture}
  \end{quote}
  In this direction, Heath-Brown proved in
  1986 that there are at most two integers
  $r$ for which the conjecture is false.
\end{remark}

\section{The Primitive Root Theorem}

\begin{remark}
  We will prove this over the course
  of the next few lectures. The following
  two propositions limit the cases we need
  to consider.
\end{remark}

\begin{prop}
  There are no primitive roots modulo
  $2^n$ where $n \in \Z$, $n \ge 3$.
\end{prop}

\begin{proof}
  Note that any primitive root modulo
  $2^n$ must be odd and have order
  \[
    \varphi(2^n)
    = 2^n - 2^{n - 1} = 2^{n - 1}.
  \]
  Let $a$ be an odd integer. To prove that
  there are no primitive roots, it suffices
  to show that
  \[
    a^{2^{n - 2}} \equiv 1 \pmod{2^n}.
  \]
  We prove this by induction on $n$.
  When $n = 3$, one can check that
  $1^2 = 3^2 = 5^2 = 7^2 \equiv 1 \Pmod{8}$,
  so the desired condition is satisfied.
  Now suppose $a^{2^{n - 2}} \equiv 1 \Pmod{2^n}$
  for some $n \ge 3$. Then
  \[
    a^{2^{n - 2}}
    = b 2^n + 1, \quad b \in \Z.
  \]
  Note that
  $(x^{2^{n - 2}})^2 = x^{2^{n - 1}}$,
  so squaring both sides yields
  \[
    a^{2^{n - 1}}
    = (a^{2^{n - 2}})^2
    = (b 2^n + 1)^2
    = b^2 2^{2n} + 2^{n + 1} b + 1
    \equiv 1 \pmod{2^{n + 1}},
  \]
  which proves thee inductive step.
  So there are no primitive roots modulo
  $2^n$ for $n \ge 3$.
\end{proof}
