\chapter{Aug.~25 --- Greatest Common Divisors}

\begin{quote}
  \emph{What do you call a root vegetable, fresh off the oven, and a pig that you throw off the balcony? One is a heated yam, and the other is a yeeted ham.}
\end{quote}

\section{Greatest Common Divisors}

\begin{remark}
  Given $a, b \in \Z$, not both zero,
  we can
  consider the set
  \[S = \{c \in \Z : c \mid a \text{ and } c \mid b\},\]
  of common divisors of both $a$ and $b$.
  Note that $\pm 1 \in S$, so $S$ is
  nonempty, and $S$ is also finite as
  at least one of $a, b$ is nonzero.
  Thus $S$ has a maximal element.
\end{remark}

\begin{definition}
  Let $a, b \in \Z$, not both zero. Then
  the \emph{greatest common divisor}
  of $a$ and $b$, denoted $(a, b)$,
  is the largest integer $d$ such that
  $d \mid a$ and $d \mid b$.
  If $(a, b) = 1$, then we say that
  $a, b$ are \emph{relatively prime} (or
  \emph{coprime}).
\end{definition}

\begin{remark}
  Note that $(0, 0)$ is not defined.
  Also note that if $(a, b) = d$, then
  \[
    (a, b) = (-a, b) = (a, -b)
    = (-a, -b) = d.
  \]
\end{remark}

\begin{example}
  We will compute $(24, 60)$. The
  list of positive divisors of $24$ and
  $60$ are
  \begin{align*}
    24 &: 1, 2, 3, 4, 6, 8, 12, 24; \\
    60 &: 1, 2, 3, 4, 5, 6, 10, 12, 15, 20, 30, 60.
  \end{align*}
  We can then see that $(24, 60) = 12$.
\end{example}

\begin{remark}
  In general, we have $(a, 0) = |a|$.
\end{remark}

\begin{prop}
  Let $(a, b) = d$. Then
  $(a / d, b / d) = 1$.
\end{prop}

\begin{proof}
  Let $d' = (a / d, b / d) > 0$. Then
  $d' \mid (a / d)$ and $d' \mid (b / d)$,
  so there exist $e, f$ such that
  $a / d = ed'$ and $b / d = fd'$.
  We can write this as $a = e d' d$ and
  $b = f d' d$. Thus
  $d' d$ is a common divisor of
  $a$ and $b$, so
  we must have $d' = 1$ by the maximality
  of $d$.
\end{proof}

\begin{prop}
  Let $a, b \in \Z$, not both zero,
and let \[T = \{ma + nb : m, n \in \Z, ma + nb > 0\}.\] Then
  $\min T$ exists and is equal to
  $(a, b)$.
\end{prop}

\begin{proof}
  Without loss of generality, we can
  assume $a \ne 0$. Note that
  $|a| \in T$, so $T$ is nonempty. Thus
  by the well-ordering principle, $T$
  has a minimal element $d$.
  Then $d = m' a + n' b$ for some
  $m', n' \in \Z$. We will show that
  $d \mid a$, a similar argument
  shows that $d \mid b$. By the
  division algorithm, we may write
  \[
    a = dq + r, \quad 0 \le r < d.
  \]
  It suffices to show that $r = 0$.
  We can rewrite the above as
  \[
    r
    = a - dq
    = a - (m' a + n' b)q
    = a(1 - m' q) - b(n' q).
  \]
  So $r$ is an integral linear combination
  of $a, b$. Since
  $d$ is the smallest positive
  integral linear combination of
  $a, b$ and $0 \le r < d$, we must
  have $r = 0$.
  So $d$ is a common divisor of $a, b$.

  Now suppose $c \mid a$ and $c \mid b$,
  then $c \mid (ma + nb)$, so
  $c$ divides $d = m' a + n' b$.
  Thus $c \le d$, so $d = (a, b)$.
\end{proof}

\begin{remark}
  If $(a, b) = d$, then $d = ma + nb$
  for some $m, n \in \Z$. If $d = 1$, then
  the converse also holds: If
  \[
    1 = ma + nb,
  \]
  and $d'$ is a common divisor of $a, b$,
  then $d' \mid 1$, so $d' = 1$.
\end{remark}

\begin{remark}
  Along the way, we showed that any
  common divisor of $a, b$ divides
  $(a, b)$.
\end{remark}

\begin{definition}
  Let $a_1, \dots, a_n \in \Z$, with at
  least one nonzero. Then the
  \emph{greatest common divisor}
  of $a_1, \dots, a_n$, denoted
  $(a_1, \dots, a_n)$, is the
  largest integer $d$ such that
  $d \mid a_i$ for $1 \le i \le n$.
  If $(a_1, \dots, a_n) = 1$, then we say
  that $a_1, \dots, a_n$ are
  \emph{relatively prime}, and if
  $(a_i, a_j) = 1$ for all $1\le i \ne j \le n$,
  then we say that $a_1, \dots, a_n$
  are \emph{pairwise relatively prime}.
\end{definition}

\begin{remark}
  Pairwise relatively prime implies
  relatively prime, but the converse
  is not true (e.g. $\{2, 4, 3\}$).
\end{remark}

\section{The Euclidean Algorithm}

\begin{lemma}
  If $a, b \in \Z$ with $0 < b \le a$ and
  $a = bq + r$ with $q, r \in \Z$, then
  $(a, b) = (r, b)$.
\end{lemma}

\begin{proof}
  It suffices to show that the two sets of
  common divisors (of $a, b$ and of $r, b$)
  are the same. Denote by $S_1$ and $S_2$
  these two sets, respectively. First let
  $c \in S_1$, so $c \mid a$ and $c \mid b$.
  We can write
  \[
    r = a - bq,
  \]
  so we have $c \mid r$. Thus
  $c \in S_2$, so $S_1 \subseteq S_2$.
  Now let $c \in S_2$, so $c \mid r$ and
  $c \mid b$. We have
  \[
    a = bq + r
  \]
  by hypothesis, so $c \mid a$, i.e.
  $c \in S_1$.
  Thus $S_1 = S_2$, so
  $(a, b) = \max S_1 = \max S_2 = (r, b)$.
\end{proof}

\begin{example}
  The above lemma allows us to compute
  greatest common divisors more efficiently.
  We will compute $(803, 154)$. We
  can write
  $803 = 5 \cdot 154 + 33$, so
  $(803, 154) = (154, 33)$.
  Continuing, we get
  \[
    (803, 154)
    = (154, 33)
    = (33, 22)
    = (22, 11)
    = (11, 0) = 11.
  \]
\end{example}

\begin{theorem}[Euclidean algorithm]
  Let $a, b \in \Z$ with $0 < b \le a$.
  Set $r_{-1} = a$, $r_0 = b$, and inductively
  write
  $r_{i - 1} = q_i r_i + r_{i + 1}$
  by the division algorithm for
  $n \ge 1$.
  Then $r_n = 0$ for some $n \ge 1$
  and $(a, b) = r_{n - 1}$.
\end{theorem}

\begin{proof}
  Note that $r_1 > r_2 > r_3 > \cdots$.
  If $r_n \ne 0$ for all $n \ge 1$, then
  this is a strictly decreasing infinite
  sequence of positive integers, which
  is not possible. So $r_n = 0$ for some
  $n \ge 1$. The conclusion
  $(a, b) = r_{n - 1}$ follows by repeatedly
  applying the lemma since
  $(a, b) = (r_i, r_{i + 1}) = (r_{n - 1}, 0) = r_{n - 1}$.
\end{proof}

\begin{example}
  By reversing this process, we can
  write $(a, b)$ explicitly as an integer
  linear combination of $a, b$.
  Using the previous example of computing
  $(803, 154)$, we can see that
  \begin{align*}
    (803, 154)
    &= 11
    = 33 - 1 \cdot 22 \\
    &= 33 - 1 \cdot (154 - 4 \cdot 33)
    = 5 \cdot 33 - 1 \cdot 154 \\
    &= 5 \cdot (803 - 5 \cdot 154) - 1 \cdot 154
    = 5 \cdot 803 - 26 \cdot 154.
  \end{align*}
  Thus we have found that
  $(803, 154) = 5 \cdot 803 - 26 \cdot 154$.
  Note that this representation
  is not unique, e.g.
  we can also write
  $11 = 19 \cdot 803 - 99 \cdot 154$.
  In fact, there are infinitely many such
  representations.
\end{example}
