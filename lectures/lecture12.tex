\chapter{Oct.~1 --- Quadratic Residues}

\begin{quote}
  \emph{What do you call a bear with no
  ear? A bee.}

  \emph{What do you call a magician
  who loses his magic? Ian.}
\end{quote}

\section{M\"obius Inversion}

\begin{theorem}[M\"obius inversion]
  Let $f, g$ be arithmetic functions.
  Then
  \[
    f(n) = \sum_{d \mid n} g(d)
    \quad \text{if and only if} \quad
    g(n) = \sum_{d \mid n} \mu(d) f(n / d)
    = \sum_{d \mid n} \mu(n / d) f(d).
  \]
\end{theorem}

\begin{proof}
  $(\Rightarrow)$ Assume
  $f(n) = \sum_{d \mid n} g(d)$.
  Then we can write
  \[
    \sum_{d \mid n} \mu(d) f(n / d)
    = \sum_{d \mid n} \mu(d)
    \sum_{a \mid (n / d)} g(a).
  \]
  Note that $a \mid (n / d)$ if and only
  if $d \mid (n / a)$, so we can
  switch the order of summation to get
  \[
    \sum_{d \mid n} \mu(d) f(n / d)
    = \sum_{a \mid n} g(a)
    \sum_{d \mid (n / a)} \mu(d)
    = \sum_{a \mid n} g(a)
    \begin{cases}
      1 & \text{if } n = a \\
      0 & \text{otherwise}
    \end{cases}
    = g(n),
  \]
  where the second equality is by
  by Proposition \ref{prop:mobius-sum}.
  This proves the forward direction.

  $(\Leftarrow)$ Assume
  that $g(n) = \sum_{d \mid n} \mu(n / d) f(d)$.
  Then we have
  \[
    \sum_{d \mid n} g(d)
    = \sum_{d \mid n} \sum_{a \mid d}
    \mu(d / a) f(a)
    = \sum_{a \mid n} f(a)
    \sum_{\substack{d \mid n \\ a \mid d}} \mu(d / a)
    = \sum_{a \mid n} f(a)
    \sum_{b \mid (n / a)} \mu(b)
  \]
  where we let
  $d = ab$ and noted that
  $ab \mid n$ if and only $ab \mid n$
  if and only if $b \mid (n / a)$.
  Then
  \[
    \sum_{d \mid n} g(d)
    = \sum_{a \mid n} f(a)
    \begin{cases}
      1 & \text{if } n = a \\
      0 & \text{otherwise}
    \end{cases}
    = f(n)
  \]
  by Proposition \ref{prop:mobius-sum},
  which proves the reverse direction.
\end{proof}

\begin{example}
  Recall that $\sum_{d \mid n} \varphi(d) = n$ by
  Theorem \ref{thm:euler-phi-sum}.
  By M\"obius inversion,
  \[
    \varphi(n)
    = \sum_{d \mid n} \mu(d) \frac{n}{d}
    = n \sum_{d \mid n} \frac{\mu(d)}{d}
    = n \prod_{p^a \mid n} \sum_{d \mid p^a} \frac{\mu(d)}{d}
    = n \prod_{p \mid n} \left(1 - \frac{1}{p}\right),
  \]
  where $\mu(d) / d$ is multiplicative
  since $\mu(d)$ and $1 / d$ both are.
  This recovers the product formula
  for $\varphi$.
\end{example}

\begin{example}
  We have $\tau(n) = \sum_{d \mid n} 1$.
  So by M\"obius inversion,
  \[
    1 = \sum_{d \mid n} \tau(n / d) \mu(d).
  \]
\end{example}

\begin{example}
  We have $\sigma(n) = \sum_{d \mid n} d$.
  So by M\"obius inversion,
  \[
    n = \sum_{d \mid n} \mu(d) \sigma(n / d).
  \]
\end{example}

\section{Quadratic Residues}

\begin{remark}
  So far, we have only studied
  linear congruences, which take
  the form $ax \equiv b \Pmod{m}$.
  Now we will be interested in
  \emph{quadratic} congruences, i.e.
  congruences of the form
  $ax^2 + bx \equiv c \Pmod{m}$.
  We will primarily restrict to the
  case $x^2 \equiv a \Pmod{p}$
  for $p$ an odd prime (the question
  is easy for $p = 2$).
\end{remark}

\begin{definition}
  Let $a, m \in \Z$, $m > 0$, and
  $(a, m) = 1$. Then $a$ is
  a \emph{quadratic residue modulo $m$}
  if the congruence
  $x^2 \equiv a \Pmod{m}$ has a
  solution. Otherwise, $a$ is a
  \emph{quadratic non-residue modulo $m$}.
\end{definition}

\begin{example}
  The quadratic residues modulo $11$
  are
  \[
    \{
      1^2, 2^2, 3^2, 4^2, 5^2, 6^2, 7^2, 8^2, 9^2, 10^2
    \}
    \equiv \{1, 4, 9, 5, 3, 3, 5, 9, 4, 1\}
    \equiv \{1, 3, 4, 5, 9\} 
    \pmod{11}.
  \]
  The quadratic non-residues are
  $\{2, 6, 7, 8, 10\}$. Note that
  the sizes of these sets are the same.
\end{example}

\begin{prop}
  Let $p$ be an odd prime and
  $a \in \Z$, $p \nmid a$.
  Then $x^2 \equiv a \Pmod{p}$ has
  either $0$ or $2$ incongruent
  solutions modulo $p$.
\end{prop}

\begin{proof}
  Assume $x^2 \equiv a \Pmod{p}$ has
  a solution $x_0$. Then
  $-x_0$ is also a solution. It
  is also incongruent to $p$, since
  if $x_0 \equiv -x_0 \Pmod{p}$,
  then $2x_0 \equiv 0 \Pmod{p}$,
  which implies $p \mid 2 x_0$.
  Since $p$ is odd, we must have
  $p \mid x_0$, so
  $x_0 \equiv 0 \Pmod{p}$. But then
  $a \equiv x_0^2 \equiv 0 \Pmod{p}$,
  a contradiction.
  Thus $x^2 \equiv a \Pmod{p}$ has at least
  two incongruent solutions modulo
  $p$ if it has a solution at all.
  
  We now show $x^2 \equiv a \Pmod{p}$
  has at most $2$ incongruent solutions.
  Suppose $x_0, x_1$ are
  solutions. Then
  \[
    x_0^2 \equiv x_1^2 \equiv a \pmod{p}.
  \]
  Then $x_0^2 - x_1^2 \equiv 0 \Pmod{p}$,
  so $p \mid x_0^2 - x_1^2 = (x_0 - x_1)(x_0 + x_1)$.
  Thus $p \mid x_0 - x_1$ or
  $p \mid x_0 + x_1$. In the first
  case, $x_0 \equiv x_1 \Pmod{p}$,
  and in the second case,
  $x_0 \equiv -x_1 \Pmod{p}$. So any
  solution is congruent to either
  $x_0$ or $-x_0$, which means
  that $x^2 \equiv a \Pmod{p}$
  has at most $2$ incongruent
  solutions.
\end{proof}

\begin{corollary}
  Let $p$ be an odd prime and
  $a \in \Z$, $p \nmid a$.
  If $x^2 \equiv a \Pmod{p}$ is
  solvable with $x = x_0$, then
  the two solutions are given by
  $x_0$ and $p - x_0$.
\end{corollary}

\begin{prop}\label{prop:quadratic-residues-count}
  Let $p$ be an odd prime. Then there
  are exactly $(p - 1) / 2$ quadratic
  residues and $(p - 1) / 2$ quadratic
  non-residues modulo $p$.
\end{prop}

\begin{proof}
  For each $1 \le x \le p - 1$, if
  $x^2 \equiv a \Pmod{p}$, then
  $(p - x)^2 \equiv a \Pmod{p}$ as well,
  and these are the only two such
  residues which square to $a$.
  That is, for each pair
  \[
    (1, p - 1), \quad (2, p - 2), \quad
    \dots, \quad (i, p - i)
  \]
  for $1 \le i \le (p - 1) / 2$,
  we get a unique quadratic residue,
  namely $i^2$. Since there are
  $(p - 1) / 2$ pairs of residues
  modulo $p$ formed in this way,
  there are exactly $(p - 1) / 2$
  quadratic residues modulo $p$.
  These are given by
  $1^2, 2^2, \dots, ((p - 1) / 2)^2$.
  The remaining $(p - 1) / 2$
  elements are quadratic
  non-residues.
\end{proof}

\section{The Legendre Symbol}

\begin{definition}
  Let $p$ be an odd prime and
  $a \in \Z$, $p \nmid a$. The
  \emph{Legendre symbol} is
  \[
    \left(\frac{a}{p}\right)
    =
    \begin{cases}
      1 & \text{if $a$ is a quadratic residue modulo $p$,} \\
      -1 & \text{if $a$ is a quadratic non-residue modulo $p$.}
    \end{cases}
  \]
\end{definition}

\begin{example}
  Recall that $1, 3, 4, 5, 9$ are
  quadratic residues modulo $11$, so
  \[
    \left(\frac{1}{11}\right)
    = \left(\frac{3}{11}\right)
    = \left(\frac{4}{11}\right)
    = \left(\frac{5}{11}\right)
    = \left(\frac{9}{11}\right)
    = 1.
  \]
  The quadratic non-residues modulo
  $11$ were $2, 6, 7, 8, 10$, so
  \[
    \left(\frac{2}{11}\right)
    = \left(\frac{6}{11}\right)
    = \left(\frac{7}{11}\right)
    = \left(\frac{8}{11}\right)
    = \left(\frac{10}{11}\right)
    = -1.
  \]
\end{example}

\begin{example}
  Evaluate $\left(\frac{3}{7}\right)$.
  This asks whether $3$ is a quadratic
  residue modulo $7$. That is, whether
  there is a solution to the
  quadratic congruence
  $x^2 \equiv 3 \Pmod{7}$. One can check
  that
  \[
    \{
      1^2, 2^2, 3^2
    \}
    \equiv \{1, 4, 2\} \pmod{7},
  \]
  and these are all of the quadratic
  residues by Proposition
  \ref{prop:quadratic-residues-count}.
  Thus $\left(\frac{3}{7}\right) = -1$.
\end{example}
