\chapter{Sept.~22 --- Exam 1 Review}

\begin{quote}
  \emph{Why are Saturday and Sunday the
  strongest days? The other
  are week days.}
\end{quote}

\section{Practice Problems}

\begin{exercise}
  Show that $\varphi$ is multiplicative
  but not completely multiplicative.
\end{exercise}

\begin{proof}
  The idea for the first part is to
  draw an $m \times n$ table of the
  first $mn$ integers,
  see the proof of Theorem
  \ref{thm:phi-multiplicative} for
  the details.
  For the second part,
  note that
  $\varphi(2) = 1$, $\varphi(2)^2 = 1$,
  but $\varphi(4) = 2$.
\end{proof}

\begin{proof}[Alternative proof]
  Let $R_k$ denote the set of residue
  classes modulo $k$ that are coprime
  to $k$.
  Note that $|R_k| = \varphi(k)$, so
  it suffices to show there is a
  bijection $\psi : R_{mn} \to R_m \times R_n$
  for $(m, n) = 1$.
  Define
  \[\psi(a) = (a \Mod{m}, a \Mod{n}).\]
  To see that $\psi$ is surjective,
  let $(b, c) \in R_m \times R_n$. Since
  $(m, n) = 1$, by
  the Chinese remainder theorem there
  exists $a \in \Z$, defined modulo
  $mn$, such that
  $a \equiv b \Pmod{m}$ and
  $a \equiv c \Pmod{n}$.
  So $\psi(a) = (b, c)$. Note that
  $(a, mn) = 1$ since
  $(a, m) = (b, m) = 1$
  and $(a, n) = (c, n) = 1$, so
  $a \in R_{mn}$.
  Injectivity follows
  since the choice of
  $a$ is unique modulo $mn$ by
  the Chinese remainder theorem.
\end{proof}

\begin{exercise}
  Compute $(163, 67)$ by the Euclidean
  algorithm.
\end{exercise}

\begin{proof}
  We compute that
  \begin{align*}
  (163, 67)
  &= (163 - 134, 67) = (29, 67) \\
  &= (29, 67 - 2 \cdot 29) = (29, 9) \\
  &= (29 - 3 \cdot 9, 9) = (2, 9) \\
  &= (2, 9 - 4 \cdot 2) = (2, 1),
  \end{align*}
  so we have $(163, 67) = 1$.
\end{proof}

\begin{exercise}
  State and prove Wilson's theorem.
\end{exercise}

\begin{proof}
  Wilson's theorem states
  that $(p - 1)! \equiv -1 \Pmod{p}$
  for prime $p$ (the converse also holds).
  The idea behind the
  proof is to note that each residue
  modulo $p$
  other than $\pm 1$ can be paired with
  its (distinct) additive
  inverse modulo $p$.
  For the details, see the proof of
  Theorem \ref{thm:wilson}.
\end{proof}

\begin{exercise}
  Find the least positive solution
  $x$ to the congruence
  $x \equiv 20^{110} \Pmod{17}$.
\end{exercise}

\begin{proof}
  Use Fermat's little theorem: The division
  algorithm gives
  $110 = 6 \cdot 16 + 14$, so
  \begin{align*}
    x
    &\equiv 20^{6 \cdot 16 + 14}
    \equiv (20^{16})^6 \cdot 20^{14}
    \pmod{17} \\
    &= 1^6 \cdot 20^{14}
    \equiv 20^{14}
    = 3^{14} \pmod{17}.
  \end{align*}
  Multiplying both sides by $3^2$ gives
  $9x \equiv 3^2 x \equiv 3^{16} \equiv 1 \Pmod{17}$, so
  it suffices to find the inverse
  of $9$ modulo $17$. Using the
  Euclidean algorithm, we have
  \[
    (17, 9)
    = (17 - 9, 9) = (8, 9)
    = (8, 9 - 8) = (8, 1)
    = 1,
  \]
  so $1 = 9 - 8 = 9 - (17 - 9) = 2 \cdot 9 - 17$.
  Thus $\overline{9} \equiv 2 \Pmod{17}$,
  so we can take $x = 2$.
\end{proof}

\begin{exercise}
  Find the least positive solution
  $x$ to the congruence
  $x \equiv 38^{110} \Pmod{21}$.
\end{exercise}

\begin{proof}
  First we compute that
  $\varphi(21) = \varphi(7) \varphi(3) = 6 \cdot 2 = 12$.
  By Euler's theorem,
  \begin{align*}
    x \equiv 38^{110}
    \equiv 17^{110}
    \equiv 17{9 \cdot 12 + 2}
    \equiv (17^{12})^9 \cdot 17^2
    \equiv 17^2 \pmod{21}.
  \end{align*}
  Now we notice that
  $17^2 \equiv (-4)^2 \equiv 16 \Pmod{21}$, so
  we can take $x = 16$.
\end{proof}

\begin{exercise}
  Let $a, m \in \Z$ and $m > 1$.
  If $(a, m) = 1$, show that
  $a^{\varphi(m) - 1}$ is the multiplicative
  inverse of $a$ modulo $m$.
\end{exercise}

\begin{proof}
  By Euler's theorem,
  $a \cdot a^{\varphi(m) - 1} = a^{\varphi(m)} \equiv 1 \Pmod{m}$,
  so $\overline{a} \equiv a^{\varphi(m) - 1} \Pmod{m}$.
\end{proof}

\begin{exercise}
  Prove that for odd primes $p$, we
  have $2(p - 3)! \equiv -1 \Pmod{p}$.
\end{exercise}

\begin{proof}
  By Wilson's theorem, 
  $(p - 1)! \equiv -1 \Pmod{p}$. Then we
  have
  \[
    (p - 3)! (p - 2)(p - 1)
    \equiv -1 \Pmod{p},
  \]
  so $2(p - 3)! \equiv -1 \Pmod{p}$ since
  $(p - 2)(p - 1) \equiv 2 \Pmod{p}$.
\end{proof}

\begin{exercise}
  Find integers $a, b$ such that
  $(a, b) = 3$ and $a + b = 66$.
\end{exercise}

\begin{proof}
  It suffices to write
  $a = 3a_1$, $b = 3b_1$, where
  $(a_1, b_1) = 1$. One way to do this
  is $(a_1, b_1) = (1, 21)$:
  \[
    3(a_1 + b_1) = 3(1 + 21)
    = 3 \cdot 22 = 66.
  \]
  Thus we may take
  $a = 3$, $b = 63$.
\end{proof}

\begin{remark}
  Recall that a \emph{reduced} 
  residue system modulo $m$
  is a set $\{r_1, \dots, r_{\varphi(m)}\}$
  of integers coprime to $m$ and
  pairwise incongruent modulo $m$.
  Note that the $r_i$ themselves need
  not be coprime, in fact they may
  share arbitrarily large common factors:
  Take any $r_\ell \ne 1$ and consider
  \[
    \{r_\ell r_1, \dots, r_{\ell} r_{\varphi(m)}\}.
  \]
  By repeating this, we can get
  arbitrarily large powers of $r_\ell$
  as a common factor.
\end{remark}
