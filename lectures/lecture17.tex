\chapter{Oct.~22 --- Primitive Roots, Part 2}

\section{More on Primitive Roots}

\begin{prop}
  Let $a, m \in \Z$ with $m > 0$ and
  $(a, m) = 1$. If $i$ is a positive
  integer, then
  \[
    \ord_m(a^i)
    = \frac{\ord_m a}{(\ord_m a, i)}.
  \]
\end{prop}

\begin{proof}
  Let $d = (\ord_m a, 1)$. Then there exist
  $b, c \in \Z$ such that
  $\ord_m a = db$, $i = dc$ and
  $(b, c) = 1$. Note
  \[
    (a^i)^b
    = (a^{dc})^{(\ord_m a)/d}
    = (a^c)^{\ord_m a}
    = (a^{\ord_m a})^c
    = 1 \pmod{m}.
  \]
  By Proposition \ref{prop:order-divides},
  this implies $\ord_m(a^i) \mid b$. Also,
  \[
    1 \equiv (a^i)^{\ord_m(a^i)}
    \equiv a^{i \ord_m (a^i)} \pmod{m},
  \]
  so by Proposition \ref{prop:order-divides},
  $\ord_m a \mid i \ord_m(a^i)$. Thus
  $db \mid dc \ord_m(a^i)$, so
  $b \mid c \ord_m(a^i)$. Since
  $(b, c) = 1$, we must have
  $b \mid \ord_m(a^i)$. Thus we see that
  $\ord_m(a^i) = b = (\ord_m a) / d = (\ord_m a) / (\ord_m a, i)$.
\end{proof}

\begin{corollary}\label{cor:order-power-coprime}
  Let $a, m \in \Z$ with $m > 0$ and
  $(a, m) = 1$. If $i$ is a positive
  integer, then
  \[
    \ord_m(a^i) = \ord_m a
  \]
  if and only if $(\ord_m a, i) = 1$.
\end{corollary}

\begin{corollary}\label{cor:primitive-root-count}
  If a primitive root modulo $m$ exists,
  then there are exactly $\varphi(\varphi(m))$
  incongruent primitive roots modulo $m$.
\end{corollary}

\begin{proof}
  Let $r$ be a primitive root. Then
  the $\ord_m r = \varphi(m)$. By
  Proposition \ref{prop:primitive-root-generate},
  the set
  \[
    \{r^1, r^2, \dots, r^{\varphi(m)}\}
  \]
  is a reduced residue system modulo $m$.
  If $1 \le i \le \varphi(m)$, then by
  Corollary \ref{cor:order-power-coprime},
  $\ord_m(r^i) = \ord_m r = \varphi(m)$
  if and only if $(i, \varphi(m)) = 1$.
  There are $\varphi(\varphi(m))$
  such $i$, and each gives a distinct primitive root.
\end{proof}

\begin{example}
  We showed previously that $3$ is a
  primitive root modulo $7$. There are
  exactly
  \[\varphi(\varphi(7)) = \varphi(6) = 2\]
  primitive roots modulo $7$. In particular,
  we must have $\ord_m (3^i) = \varphi(7)$
  if and only if $(i, \varphi(7)) = (1, 6) = 1$.
  Thus $i = 1, 5$, so $3^1 = 3$ and
  $3^5 \equiv 5 \pmod{7}$ are the
  two primitive roots modulo $7$.
\end{example}

\begin{example}
  Recall that
  $2$ is a primitive root modulo $13$.
  Thus there are $\varphi(\varphi(13)) = \varphi(2) = 4$
  primitive roots. Find the other three
  primitive roots as an exercise.
\end{example}

\begin{remark}
  Note that $\varphi(\varphi(8)) = \varphi(4) = 2$,
  but this does not imply that $8$ has
  $2$ primitive roots. We need to know that
  a primitive root exists first for Corollary
  \ref{cor:primitive-root-count}
  to apply.
\end{remark}

\section{Primitive Roots for Primes}

\begin{theorem}[Lagrange]\label{thm:lagrange}
  Let $p$ be a prime and let
  \[
    f(x) = a_n x^n + a_{n-1} x^{n-1} + \cdots + a_1 x + a_0
  \]
  be a polynomial with degree $n$ and
  integer coefficients $a_0, a_1, \dots, a_n$,
  such that $p \nmid a_n$. Then the
  congruence
  \[
    f(x) \equiv 0 \pmod{p}
  \]
  has at most $n$ incongruent solutions.
\end{theorem}

\begin{proof}
  We proceed by induction on $n$. Suppose
  $n = 1$. Then $f(x) = a_1 x + a_0$
  where $p \nmid a_1$. Then
  \[
    a_1 x + a_0 \equiv 0 \pmod{p},
  \]
  which is equivalent to
  $a_1 x = -a_0 \Pmod{p}$. Now since
  $p \nmid a_1$, we can multiply both
  sides by $\overline{a}_1$ to get
  $x = -a_0 \overline{a}_1$.
  This proves the base case.

  Suppose $k \ge 1$ and that the theorem
  holds for polynomials of degree
  $k$. Let $n = k + 1$, then we can write
  \[
    f(x) = a_{k + 1} x^{k + 1} + \cdots + a_1 x + a_0
  \]
  where $p \nmid a_{k + 1}$. If
  $f(x) \equiv 0 \Pmod{p}$ has no solutions,
  then we are done. Now suppose $x_0$ is a
  solution. By polynomial long division,
  there exists a polynomial $q(x)$ with
  integer coefficients such that
  \[
    f(x) = (x - x_0) q(x) + r
  \]
  for some integer $r$, where $q(x)$
  has degree $k$. Note that
  \[
    0 \equiv f(x_0) \equiv (x_0 - x_0) q(x_0) + r
    \equiv r \pmod{p},
  \]
  so $r \equiv 0 \Pmod{p}$, and we have
  $f(x) \equiv (x - x_0) q(x) \Pmod{p}$. If
  \[
    0 \equiv f(x_1) \equiv (x_1 - x_0) q(x)
    \pmod{p},
  \]
  then $x_1 - x_0 \equiv 0 \Pmod{p}$
  or $q(x_1) \equiv 0 \Pmod{p}$.
  If $x_1 \not\equiv x_0 \Pmod{p}$,
  then $q(x_1) \equiv 0 \Pmod{p}$, and
  $q(x)$ has at most $k$ roots by the
  induction hypothesis. Thus
  $f(x)$ has at most $k + 1$ roots.
\end{proof}

\begin{prop}\label{prop:d-roots-mod-p}
  Let $p$ be a prime and $d \in \Z$ with
  $d > 0$ and $d \mid p - 1$. Then the
  congruence
  \[
    x^d - 1 \equiv 0 \pmod{p}
  \]
  has exactly $d$ incongruent solutions
  modulo $p$.
\end{prop}

\begin{proof}
  Since $d \mid p - 1$, there exists
  $e \in \Z$ such that $p - 1 = de$.
  Note that if $p \nmid x$, then
  \[
    0 \equiv x^{p - 1} - 1
    \equiv x^{de - 1}
    \equiv (x^d - 1)(x^{d(e - 1)} + x^{d(e - 2)} + \cdots + x^d + 1)
    \pmod{p}.
  \]
  Thus $x^d - 1 \equiv 0 \Pmod{p}$
  (call this $(1)$) or
  $x^{d(e - 1)} + x^{d(e - 2)} + \cdots + x^d + 1 \equiv 0 \Pmod{p}$
  (call this $(2)$). By
  Theorem \ref{thm:lagrange},
  $(2)$ has at most $d(e - 1) = p - 1 - d$
  solutions. Also $(1)$ has at most $d$
  solutions. By Fermat's little theorem,
  $x^{p - 1} - 1 \equiv \Pmod{p}$
  has exactly $p - 1$ solutions, so
  $x^d - 1 \equiv 0 \Pmod{p}$ has least
  $d$ solutions. Therefore, it has
  exactly $d$ solutions.
\end{proof}

\begin{remark}
  Proposition \ref{prop:d-roots-mod-p}
  is a generalization of the fact that
  $x^2 \equiv 1 \Pmod{p}$ has exactly
  $2$ solutions for odd primes $p$.
\end{remark}

\begin{example}
  Prove that $3$ is a primitive root
  modulo $43$, and then use this to
  calculate all elements of order $14$.
  To show that $3$ is a primitive root,
  we need to check $3^i$ for
  $i \mid \varphi(43) = 42$, so for
  \[
    i = 1, 2, 3, 6, 7, 14, 21, 42.
  \]
  We can compute that
  \begin{align*}
    3^1 &\equiv 3 \pmod{43},\\
    3^2 &\equiv 9 \pmod{43},\\
    3^3 &\equiv 27 \pmod{43},\\
    3^6 &\equiv 3^4 \cdot 3^2 \equiv (-5) \cdot 9 \equiv -2 \pmod{43},\\
    3^7 &\equiv -6 \pmod{43},\\
    3^{14} &\equiv 36 \equiv -7 \pmod{43},\\
    3^{21} &\equiv 42 \equiv -1 \pmod{43},\\
    3^{42} &\equiv 1 \pmod{43}.
  \end{align*}
  This confirms that $3$ is a primitive root modulo
  $43$. To find elements of order $14$, we
  want $i$ such that
  \[
    14 = \ord_{43}(3^i)
    = \frac{\ord_{43}(3)}{(\ord_{43}(3), i)}
    = \frac{42}{(42, i)},
  \]
  so we want $(42, i) = 42 / 14 = 3$.
  This works for
  $i = 3, 9, 15, 27, 33, 39$. Thus the
  elements of order $14$ are represented
  by $3^3, 3^9, 3^{15}, 3^{27}, 3^{33}, 3^{39}$ modulo $43$.
\end{example}
