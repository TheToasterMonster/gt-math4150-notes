\chapter{Aug.~27 --- Fundamental Theorem of Arithmetic}

\begin{quote}
  \emph{What's the difference between a mediocre clown and a rabbit in the gym? One's a bit funny, the other's a fit bunny.}
\end{quote}

\section{The Fundamental Theorem of Arithmetic}

\begin{lemma}[Euclid]\label{lem:prime-divides-product}
  Let $a, b \in \Z$ and let $p$ be a prime.
  If $p \mid ab$, then $p \mid a$
  or $p \mid b$.
\end{lemma}

\begin{proof}
  If $p \mid a$, then we are done, so
  assume $p \nmid a$. Then
  $(p, a) = 1$. Thus we can write
  $1 = ma + np$ for some $m, n \in \Z$.
  Since $p \mid ab$, we can write
  $ab = pc$ for some $c \in \Z$.
  Multiplying by $b$, we have
  \[
    b = bma + bnp
    = m(cp) + nb p
    = p(mc + nb).
  \]
  Thus we see that
  $p \mid b$, as desired.
\end{proof}

\begin{remark}
  This fails if $p$ is composite:
  Take $p = 6$, $a = 2$, and $b = 3$.
\end{remark}

\begin{exercise}
  Determine where the proof fails
  if $p$ is composite.
\end{exercise}

\begin{corollary}
  Let $a_1, \dots, a_n \in \Z$ and $p$
  a prime. If $p \mid a_1 \cdots a_n$,
  then $p \mid a_i$ for some $1 \le i \le n$.
\end{corollary}

\begin{proof}
  Induct on $n$. The base case $n = 1$
  is trivial. If $n = 2$, then this is
  just Lemma \ref{lem:prime-divides-product}.
  Now suppose $n \ge 2$, and we show the
  result for $n + 1$. Specifically,
  assume that if $p \mid a_1 \cdots a_n$,
  then $p \mid a_i$ for some
  $1 \le i \le n$. Suppose
  $p \mid a_1 \cdots a_n a_{n+1}$.
  Then $p \mid (a_1 \cdots a_n) a_{n+1}$.
  So by Lemma \ref{lem:prime-divides-product},
  we have $p \mid a_1 \cdots a_n$
  or $p \mid a_{n + 1}$. If $p \mid a_{n + 1}$, then we are done.
  Otherwise, $p \mid a_1 \cdots a_n$,
  so $p \mid a_i$ for some $1 \le i \le n$
  by the induction hypothesis.
  In particular, $p \mid a_i$ for some
  $1 \le i \le n + 1$, as desired.
\end{proof}

\begin{theorem}[Fundamental theorem of arithmetic]
  Every integer $m > 1$ may be
  expressed in the form
  $m = p_1^{a_1} \cdots p_n^{a_n}$
  where $p_1, \dots, p_n$ are distinct
  primes and $a_1, \dots, a_n$ are
  positive integers. This form is
  called the \emph{prime factorization}
  of the integer $m$. Moreover, this
  factorization is essentially unique, i.e.
  unique up to permutations
  of the factors $p_i^{a_i}$.
\end{theorem}

\begin{proof}
  We first prove existence.
  Assume to the contrary that there exists
  $m > 1$ that does not have a prime
  factorization. Without loss of generality,
  we can assume $m$ is the smallest such
  integer by the well-ordering principle.
  In particular, $m$ cannot be prime.
  So $m = ab$ for some $1 < a, b < m$. Then
  $a, b$ have prime factorizations. Thus
  so too does $m$, a contradiction.

  Now we prove uniqueness. Assume
  that $m = p_1^{a_1} \cdots p_n^{a_n} = q_1^{b_1} \cdots q_r^{b_r}$.
  Without loss of generality, we
  can assume $p_1 < p_2 < \cdots < p_n$
  and $q_1 < q_2 < \cdots < q_r$.
  We need to show that
  $n = r$, $p_i = q_i$ for each $i$, and
  $a_i = b_i$ for each $i$. Let $p_i \mid m$.
  Then $p_i \mid q_1^{b_1} \cdots q_r^{b_r}$,
  so $p_i \mid q_j$ for some $1 \le j \le r$.
  Thus $p_i = q_j$ since both are prime.
  Similarly, given $q_i$, we have
  $q_i = p_j$ for some $j$. Thus the
  primes in the two factorizations (as
  sets) are the same. Thus $n = r$, and
  by the ordering assumption, we have
  $p_i = q_i$ for each $1 \le i \le n$.
  So
  \[
    m = p_1^{a_1} \cdots p_n^{a_n}
    = p_1^{b_1} \cdots p_n^{b_n}.
  \]
  Suppose to the contrary that
  $a_i \ne b_i$ for some $i$. Without
  loss of generality, assume $a_i < b_i$.
  We have $p_i^{b_i} \mid m$, so
  $p_i^{b_i} \mid p_1^{a_1} \cdots p_{i - 1}^{a_{i - 1}} p_i^{a_i} p_{i + 1}^{a_{i + 1}} \cdots p_n^{a_n}$.
  Thus $p_i^{b_i - a_i} \mid p_1^{a_1} \cdots p_{i - 1}^{a_{i - 1}} p_{i + 1}^{a_{i + 1}} \cdots p_n^{a_n}$.
  Since $a_i < b_i$, we have
  $b_i - a_i > 0$, so
  $p_i \mid p_1^{a_1} \cdots p_{i - 1}^{a_{i - 1}} p_{i + 1}^{a_{i + 1}} \cdots p_n^{a_n}$
  by the transitivity of divisibility.
  Then $p_i \mid p_j$ for some $j \ne i$,
  so $p_i = p_j$, which is a contradiction
  since the $p_i$ are all distinct primes.
  This proves uniqueness.
\end{proof}

\begin{remark}
  This is one reason why we do not
  consider $1$ to be a prime, as we would
  lose uniqueness.
\end{remark}

\begin{example}
  We can write $60 = 2^2 \cdot 3 \cdot 5$
  and $756 = 2^2 \cdot 3^3 \cdot 7$.
\end{example}

\section{Least Common Multiples}

\begin{definition}
  Let $a, b \in \Z$ with $a, b > 0$.
  The \emph{least common multiple}
  of $a$ and $b$, denoted $[a, b]$, is the
  least positive integer $m$ such that
  $a \mid m$ and $b \mid m$.
\end{definition}

\begin{remark}
  Since $ab$ is a common multiple
  of $a$ and $b$, $[a, b]$ always
  exists by the well-ordering principle.
\end{remark}

\begin{example}
  We will compute $[6, 7]$. The multiples
  of $6$ and $7$ include:
  \begin{align*}
    6 &: 6, 12, 18, 24, 30, 36, 42, 48, \dots; \\
    7 &: 7, 14, 21, 28, 35, 42, 49, \dots.
  \end{align*}
  So we can see that $[6, 7] = 42 = 6 \cdot 7$.
  On the other hand, $[6, 8] = 24 \ne 6 \cdot 8$.
\end{example}

\begin{remark}
  The fundamental theorem of arithmetic
  can be used to calculate both GCDs and
  LCMs.
\end{remark}

\begin{prop}\label{prop:gcd-lcm-prime-factorization}
  Let $a, b \in \Z$ with $a, b > 1$.
  Write $a = p_1^{a_1} \cdots p_n^{a_n}$
  and $b = p_1^{b_1} \cdots p_n^{b_n}$,
  where the $p_i$ are distinct primes,
  and $a_i, b_i \ge 0$. Then we have
  \[
    (a, b) = p_1^{\min\{a_1, b_1\}} \cdots p_n^{\min\{a_n, b_n\}} \quad \text{and} \quad
    [a, b] = p_1^{\max\{a_1, b_1\}} \cdots p_n^{\max\{a_n, b_n\}}.
  \]
\end{prop}

\begin{proof}
  Left as an exercise.
\end{proof}

\begin{example}
  Calculate $(756, 2205)$ and
  $[756, 2205]$. We can write
  \[
    756 = 2^2 \cdot 3^3 \cdot 5^0 \cdot 7^1
    \quad \text{and} \quad
    2205 = 2^0 \cdot 3^2 \cdot 5^1 \cdot 7^2.
  \]
  So we have
  $(756, 2205) = 2^0 \cdot 3^2 \cdot 5^0 \cdot 7^1 = 63$ and
  $[756, 2205] = 2^2 \cdot 3^3 \cdot 5 \cdot 7^2 = 26460$.
\end{example}

\begin{lemma}\label{lem:min-max-sum}
  Given $x, y \in \R$, we have
  $\min\{x, y\} + \max\{x, y\} = x + y$.
\end{lemma}

\begin{proof}
  The result is obvious if $x = y$.
  Otherwise, one is the minimum and the
  other is the maximum.
\end{proof}

\begin{theorem}
  Let $a, b \in \Z$ with $a, b > 1$. Then
  $(a, b) [a, b] = ab$.
\end{theorem}

\begin{proof}
  Write $a = p_1^{a_1} \cdots p_n^{a_n}$
  and $b = p_1^{b_1} \cdots p_n^{b_n}$
  with $a_i, b_i \ge 0$ and $p_i$
  distinct. By Proposition
  \ref{prop:gcd-lcm-prime-factorization},
  \begin{align*}
    (a, b) [a, b]
    &= p_1^{\min\{a_1, b_1\}} \cdots p_n^{\min\{a_n, b_n\}}
      p_1^{\max\{a_1, b_1\}} \cdots p_n^{\max\{a_n, b_n\}} \\
    &= p_1^{\min\{a_1, b_1\} + \max\{a_1, b_1\}} \cdots
      p_n^{\min\{a_n, b_n\} + \max\{a_n, b_n\}}
    = p_1^{a_1 + b_1} \cdots p_n^{a_n + b_n}
    = ab,
  \end{align*}
  where the third equality follows
  from Lemma \ref{lem:min-max-sum}.
\end{proof}
