\chapter{Oct.~15 --- Quadratic Reciprocity, Part 2}

\begin{quote}
  \emph{What do you call a place in America that receives shipments of pollinators? A U.S. bee port.}
\end{quote}

\section{Proof of Quadratic Reciprocity}

\begin{proof}[Proof of Theorem \ref{thm:quadratic-reciprocity}]
  Without loss of generality, assume
  $p > q$.
  Consider a $q \times p$ grid on $\R^2$.
  Let $L$ be the line from
  $(0, 0)$ to $N = (q, p)$. Let
  $A = ((p - 1) / 2, 0)$,
  $B = (0, (q - 1) / 2)$. Let
  $M$ be the intersection of
  $L$ with the line
  $x = (p - 1) / 2$ and $D$ be the
  intersection of $L$ 
  with the line $y = (q - 1) / 2$.
  Also let $C = ((p - 1) / 2, (q - 1) / 2)$.
  We count the number of lattice points in
  the rectangle $OABC$, not including the
  axes.
  This number is clearly
  $(p - 1)(q - 1) / 4$. Now observe that:
  \begin{enumerate}
    \item The line $ON$ has slope
      $q / p$. In particular, $ON$
      contains no lattice points.
    \item The $y$-coordinate of $M$ is
      $((p - 1) / 2) (q / p) = q / 2 - q / 2p$.
      This lies between the consecutive
      integers
      $(q - 1) / 2$ and $(q + 1) / 2$:
      \[
        \frac{q - 1}{2} = \frac{q}{2} - \frac{1}{2}
        < \frac{q}{2} - \frac{q}{2p}
        < \frac{q}{2} < \frac{q + 1}{2}.
      \]
  \end{enumerate}
  So the number of lattice points
  in $OABC$ excluding the axes, and
  below the line $ON$ is
  \[
    N_1 = \sum_{j = 1}^{(p - 1) / 2}
    \left\lfloor \frac{jq}{p} \right\rfloor.
  \]
  Likewise, the number of lattice points
  above the line $ON$ is
  \[
    N_2 = \sum_{j = 1}^{(q - 1) / 2} \left\lfloor \frac{jp}{q} \right\rfloor.
  \]
  Thus the total number of lattice points
  in question is
  $N_1 + N_2 = (p - 1)(q - 1) / 4$. Then
  \[
    \left(\frac{p}{q}\right)
    \left(\frac{q}{p}\right)
    = (-1)^{N_2} (-1)^{N_1}
    = (-1)^{N_1 + N_2}
    = (-1)^{(p - 1)(q - 1) / 4}
  \]
  by Lemma \ref{lem:qr-lemma}, which
  proves the claim.
\end{proof}

\section{Applications of Quadratic Reciprocity}

\begin{remark}
  Note that we have characterized the
  primes for which $-1$ and $2$ are
  quadratic residues.
\end{remark}

\pagebreak
\begin{example}
  For what primes $p$ is
  $3$ a quadratic residue?
  It suffices to compute when $\big(\frac{3}{p}\big) = 1$.
  By quadratic reciprocity, we have
  \[
    \left(\frac{3}{p}\right)
    =
    \begin{cases}
      \big(\frac{p}{3}\big) & p \equiv 1 \Pmod{4},\\
      -\big(\frac{p}{3}\big) & p \equiv 3 \Pmod{4}.
    \end{cases}
  \]
  Note that the only (non-zero) quadratic
  residue modulo $3$ is $1$.
  If $p \equiv 1 \Pmod{4}$, then
  $\big(\frac{p}{3}\big) = 1$
  if and only if $p \equiv 1 \Pmod{3}$.
  If $p \equiv 3 \Pmod{4}$,
  then $\big(\frac{p}{3}\big) = -1$
  if and only if $p \equiv 2 \Pmod{3}$.
  By the Chinese remainder theorem,
  we can rewrite the first condition
  as $p \equiv 1 \Pmod{12}$ and the
  second condition as
  $p \equiv -1 \Pmod{12}$. Thus
  we see that $\big(\frac{3}{p}\big) = 1$
  if and only if
  $p \equiv \pm 1 \Pmod{12}$.
\end{example}

\begin{example}
  Characterize the primes $p$ for
  which both $2$ and $3$ are
  quadratic residues modulo $p$.
  We want $p$ such that
  $\big(\frac{2}{p}\big) = \big(\frac{3}{p}\big) = 1$.
  We already know that
  $\big(\frac{2}{p}\big) = 1$
  if and only if $p \equiv \pm 1 \Pmod{8}$
  and $\big(\frac{3}{p}\big) = 1$
  if and only if $p \equiv \pm 1 \Pmod{12}$.
  So by the Chinese remainder theorem,
  we have
  $\big(\frac{2}{p}\big) = \big(\frac{3}{p}\big) = 1$
  if and only if
  $p \equiv \pm 1 \Pmod{24}$.
\end{example}

\begin{example}
  Characterize the primes $p$ for which
  $13$ is a quadratic residue modulo $p$.
  By quadratic reciprocity,
  we have
  $\big(\frac{13}{p}\big) = \big(\frac{p}{13}\big)$.
  The non-zero
  quadratic residues modulo $13$ are
  \[
    \{1, 3, 4, 9, 10, 12\},
  \]
  So $\big(\frac{13}{p}\big) = \big(\frac{p}{13}\big) = 1$
  if and only if
  $p \equiv 1, 3, 4, 9, 10, 12 \equiv \pm 1, \pm 3, \pm 4 \Pmod{13}$.
\end{example}

\begin{example}
  Characterize the primes $p$ for
  which $11$ is a quadratic residue
  modulo $p$. By quadratic reciprocity,
  we have that
  \[
    \left(\frac{11}{p}\right)
    =
    \begin{cases}
      \big(\frac{p}{11}\big) & p \equiv 1 \Pmod{4},\\
      -\big(\frac{p}{11}\big) & p \equiv 3 \Pmod{4}.
    \end{cases}
  \]
  The quadratic residues modulo $11$ are
  $1, 3, 4, 5, 9$, and the
  quadratic non-residues are
  $2, 6, 7, 8, 10$. If
  $p \equiv 1 \Pmod{4}$, then
  $\big(\frac{p}{11}\big) = 1$
  if and only if $p \equiv 1, 3, 4, 5, 9 \Pmod{11}$,
  and if $p \equiv 3 \Pmod{4}$, then
  $\big(\frac{p}{11}\big) = -1$
  if and only if
  $p \equiv 2, 6, 7, 8, 10 \Pmod{11}$.
  One can check each of these
  cases by the Chinese
  remainder theorem, and one gets
  $\big(\frac{11}{p}\big) = 1$
  if and only if
  $p \equiv 1, 5, 7, 9, 19, 25, 35, 37, 39, 43 \Pmod{44}$, or
  \[
    p \equiv \pm 1, \pm 5, \pm 7, \pm 9, \pm 19 \pmod{44}.
  \]
\end{example}

\begin{example}
  Characterize the primes $p$ for which
  $-1$ and $2$ are both quadratic
  residues modulo $p$. Recall that
  $\big(\frac{-1}{p}\big) = 1$
  if and only if $p \equiv 1 \Pmod{4}$
  and $\big(\frac{2}{p}\big) = 1$
  if and only if $p \equiv \pm 1 \Pmod{8}$,
  so we see that
  $\big(\frac{-1}{p}\big) = \big(\frac{2}{p}\big) = 1$
  if and only if $p \equiv 1 \Pmod{8}$.
\end{example}

\begin{example}
  Characterize the primes $p$ for which
  both $-1$ and $3$ are quadratic
  residues modulo $p$.
  Recall that $\big(\frac{-1}{p}\big) = 1$
  if and only if $p \equiv 1 \Pmod{4}$
  and $\big(\frac{3}{p}\big) = 1$
  if and only if
  $p \equiv \pm 1 \Pmod{12}$.
  So we have
  $\big(\frac{-1}{p}\big) = \big(\frac{3}{p}\big) = 1$
  if and only if $p \equiv 1 \Pmod{12}$.
\end{example}

\begin{exercise}
  Characterize the primes $p$ for which
  $3$ and $5$ are quadratic residues
  modulo $p$.
\end{exercise}
