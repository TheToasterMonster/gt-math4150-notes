\chapter{Oct.~20 --- Primitive Roots}

\begin{quote}
  \emph{
  What happens when you step on a grape?
Nothing, it lets out a little whine.}
\end{quote}

\section{Orders}

\begin{remark}
  Let $m$ be a positive integer and
  $(a, m) = 1$. By Euler's theorem, we know
  that
  \[
    a^{\varphi(m)} \equiv 1 \pmod{m}.
  \]
  However, it may happen that
  $a^g \equiv 1 \pmod{m}$ for some smaller
  $g$.
\end{remark}

\begin{definition}
  Let $a, m \in \Z$ with $m > 0$,
  $(a, m) = 1$. Then the \emph{order of $a$ modulo $m$},
  denoted $\ord_m a$, is the least
  positive integer $n$ such that
  $a^n \equiv 1 \pmod{m}$.
\end{definition}

\begin{example}
  We compute $\ord_7 2$. We can compute that
  \begin{align*}
    2^1 &\equiv 2 \pmod{7},\\
    2^2 &\equiv 4 \pmod{7},\\
    2^3 &\equiv 1 \pmod{7},
  \end{align*}
  so we see that $\ord_7 2 = 3$. Note
  that Euler's theorem only guarantees
  $\ord_7 2 \le \phi(7) = 6$.
\end{example}

\begin{example}\label{ex:order-7-3}
  We compute $\ord_7 3$. We can compute
  that
  \begin{align*}
    3^1 &\equiv 3 \pmod{7},\\
    3^2 &\equiv 2 \pmod{7},\\
    3^3 &\equiv 6 \pmod{7},\\
    3^4 &\equiv 4 \pmod{7},\\
    3^5 &\equiv 5 \pmod{7},\\
    3^6 &\equiv 1 \pmod{7},
  \end{align*}
  so we see that $\ord_7 3 = 6$.
\end{example}

\begin{prop}\label{prop:order-divides}
  Let $a, m \in \Z$ with $m > 0$ and
  $(a, m) = 1$. Then $a^n \equiv 1 \Pmod{m}$
  for some positive integer $n$ if and only
  if $\ord_m a \mid n$. In particular,
  $\ord_m a \mid \varphi(m)$.
\end{prop}

\begin{proof}
  $(\Rightarrow)$ Suppose that
  $a^n \equiv 1 \Pmod{m}$. By the
  division algorithm, there exists
  $q, r \in \Z$ such that
  \[
    n = q (\ord_m a) + r, \quad
    0 \le r < \ord_m a.
  \]
  Then $1 = a^n \equiv a^{q (\ord_m a) + r} \equiv (a^{\ord_m a})^q a^r \equiv a^r \Pmod{m}$,
  which can only happen if $r = 0$
  by the definition of $\ord_m a$
  and $0 \le r < \ord_m a$. Therefore,
  $\ord_m a \mid n$.

  $(\Leftarrow)$ Suppose that
  $\ord_m a \mid n$. Then $n = q (\ord_m a)$,
  so $a^n = a^{q (\ord_m a)} \equiv (a^{\ord_m a})^q \equiv 1 \Pmod{m}$.
\end{proof}

\begin{example}
  We compute $\ord_{13} 2$.
  By Proposition \ref{prop:order-divides},
  it suffices to check divisors
  of $\varphi(13) = 12$:
  \begin{align*}
    2^1 &\equiv 2 \pmod{13},\\
    2^2 &\equiv 4 \pmod{13},\\
    2^3 &\equiv 8 \pmod{13},\\
    2^4 &\equiv 3 \pmod{13},\\
    2^6 &\equiv 12 \pmod{13},\\
    2^{12} &\equiv 1 \pmod{13},
  \end{align*}
  thus $\ord_{13} 2 = 12$. Note that we
  did not need to compute $2^7$, $2^8$, etc.
  to verify this.
\end{example}

\begin{prop}\label{prop:order-congruence}
  Let $a, m \in \Z$ with $m > 0$ and
  $(a, m) = 1$. If $i, j$ are
  non-negative integers, then
  $a^i \equiv a^j \Pmod{m}$ if and only
  if $i \equiv j \Pmod{{\ord_m a}}$.
\end{prop}

\begin{proof}
  Without loss of generality, suppose
  $i \ge j$.

  $(\Rightarrow)$ Assume
  $a^i \equiv a^j \Pmod{m}$. Then we can
  write
  \[
    a^j \equiv a^i \equiv a^j a^{i - j}
    \pmod{m}.
  \]
  Since $(a, m) = 1$, we can cancel
  $a^j$ to get
  $1 \equiv a^{i - j} \Pmod{m}$.
  Then by Proposition \ref{prop:order-divides},
  $\ord_m a \mid i - j$.

  $(\Leftarrow)$ Assume
  $i \equiv j \Pmod{{\ord_m a}}$. Then
  $\ord_m a \mid i - j$, so there exists
  $n \in \Z$ such that $i - j = n (\ord_m a)$.
  Thus $i = j + n (\ord_m a)$, and we have
  $a^i \equiv a^{j + n (\ord_m a)} \equiv a^j (a^{\ord_m a})^n \equiv a^j \Pmod{m}$.
\end{proof}

\begin{example}
  We have seen previously that $\ord_7 2 = 3$.
  So if $i, j$ are non-negative integers,
  then $2^i \equiv 2^j \Pmod{7}$
  if and only if $i \equiv j \Pmod{3}$
  by Proposition \ref{prop:order-congruence}.
  Note that
  \[
    2000 \equiv 2 \pmod{3},
  \]
  so we can calculate
  $2^{2000} \equiv 2^2 \equiv 4 \Pmod{7}$.
\end{example}

\section{Primitive Roots}

\begin{definition}
  Let $r, m \in \Z$ with $m > 0$ and
  $(r, m) = 1$. Then $r$ is called a
  \emph{primitive root modulo $m$} if
  $\ord_m r = \varphi(m)$.
\end{definition}

\begin{remark}
  See \emph{primitive root diffusers}
  for an interesting application
  (also \emph{quadratic residue diffusers}).
\end{remark}

\begin{example}
  We have seen that
  $3$ is a primitive root modulo $7$ and
  $2$ is a primitive root modulo $13$.
  On the other hand, $2$ is not a primitive
  root modulo $7$.
\end{example}

\begin{example}
  We prove that there are no primitive
  roots modulo $8$. The reduced residues
  modulo $8$ are $1, 3, 5, 7$, and
  $\varphi(8) = 4$. But
  $1^2 \equiv 3^2 \equiv 5^2 \equiv 7^2 \equiv 1$,
  so none of these are primitive roots
  modulo $8$.

  In particular, not all
  integers $m$ possess a primitive root.
  The \emph{primitive root theorem} (later)
  tells us that $m$ has a primitive root
  if and only if $m = 1, 2, 4, p^k, 2p^k$,
  where $p$ is an odd prime.
\end{example}

\begin{prop}\label{prop:primitive-root-generate}
  Let $r$ be a primitive root modulo $m$.
  Then
  $\{r, r^2, r^3, \dots, r^{\varphi(m)}\}$
  is a complete set of reduced residues
  modulo $m$.
\end{prop}

\begin{proof}
  Since $r$ is a primitive root modulo $m$,
  we have $(r, m) = 1$, and so
  $(r^n, m) = 1$ for any $n \ge 1$.
  Also, there are $\varphi(m)$ elements
  in the list, so it remains to show that
  they are distinct modulo $m$.

  To do this, suppose that $r^i \equiv r^j \Pmod{m}$
  for some $1 \le i, j \le \varphi(m)$.
  Then Proposition \ref{prop:order-congruence}
  implies that
  $i \equiv j \Pmod{{\varphi(m)}}$, so
  $i = j$. Thus the $r^i$ are distinct
  modulo $m$.
\end{proof}

\begin{remark}
  Proposition \ref{prop:primitive-root-generate}
  says that a primitive root (when it exists)
  generates the reduced residues modulo
  $m$.
\end{remark}

\begin{example}
  Recall that $3$ is a primitive root
  modulo $7$. We saw in Example
  \ref{ex:order-7-3} that
  \[
    \{3^1, 3^2, 3^3, 3^4, 3^5, 3^6\}
    \equiv \{3, 2, 6, 4, 5, 1\}
    \pmod{7},
  \]
  in particular this is a complete set of
  reduced residues modulo $7$.
\end{example}

\begin{example}
  Recall that $2$ is a primitive root
  modulo $13$. We can compute
  \[
    \{2^1, 2^2, 2^3, 2^4, 2^5, 2^6,
    2^7, 2^8, 2^9, 2^{10}, 2^{11}, 2^{12}\}
    \equiv
    \{2, 4, 8, 3, 6, 12, 11, 9, 5, 10, 7, 1\}
    \pmod{13},
  \]
  which is a complete set of reduced residues
  modulo $13$.
\end{example}

\begin{remark}
  If a primitive root exists, it is
  in general not unique. We will determine
  how many there are next lecture
  (we will see that there are
  $\varphi(\varphi(m))$ of them).
\end{remark}

\begin{exercise}
  Show there are no primitive roots
  modulo $12$.

  To do this, write
  $\Z / 12\Z \cong \Z / 3\Z \times \Z / 4\Z$,
  then we have
  \[(\Z / 12\Z)^\times \cong (\Z / 3\Z)^\times \times (\Z / 4\Z)^\times = \Z / 2\Z \times \Z / 2\Z\]
  which is not cyclic.
  Alternatively, one can just compute
  directly
  for $(\Z / 12\Z)^\times = \{1, 5, 7, 11\}$
  that
  \[
    1^2 \equiv 5^2 \equiv 7^2 \equiv 11^2 \equiv 1 \pmod{12},
  \]
  so none of these can be primitive
  roots modulo $12$.
\end{exercise}
