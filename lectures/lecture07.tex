\chapter{Sept.~10 --- Chinese Remainder Theorem}

\section{More on Linear Congruences}

\begin{corollary}
  Consider the linear congruence
  $ax \equiv b \Pmod{m}$ and let
  $d = (a, m)$. If $d \mid b$, then
  there are exactly $d$ incongruent
  solutions modulo $m$, given by
  \[
    x = x_0 + \frac{m}{d} \cdot n,
    \quad n = 0, 1, \dots, d - 1
  \]
  where $x_0$ is any particular solution.
\end{corollary}

\begin{example}
  We solve $16 x \equiv 8 \Pmod{28}$. We
  compute
  $d = (16, 28)$ by the Euclidean algorithm:
  \begin{align*}
    28 &= 1 \cdot 16 + 12 \\
    16 &= 1 \cdot 12 + 4 \\
    12 &= 3 \cdot 4 + 0.
  \end{align*}
  So $d = 4$. Since $4 \mid 8$, the
  congruence has $4$ incongruent solutions.
  Working backwards, we have
  \[
    4 = 2 \cdot 16 + (-1) \cdot 28.
  \]
  Multiplying by $2$, we get that
  $8 = 4 \cdot 16 + (-2) \cdot 28$.
  Taking this equation modulo $28$, we get
  \[
    16 \cdot 4 \equiv 8 \Pmod{28},
  \]
  so $x_0 = 4$ is a particular solution.
  Thus all the incongruent solutions
  are given by
  $x = 4 + (28 / 4) n$ for
  $n = 0, 1, 2, 3$, that is
  $x = 4, 11, 18, 25$.
\end{example}

\begin{definition}
  Any solution of $ax \equiv 1 \Pmod{m}$
  is called the \emph{multiplicative inverse}
  of $a$ modulo $m$. The multiplicative
  inverse of $a$ is often denoted
  $\overline{a}$.
\end{definition}

\begin{corollary}
  The congruence $ax \equiv 1 \Pmod{m}$
  has a solution if and only if
  $(a, m) = 1$. In this case, the
  congruence has a unique solution.
  In particular, the multiplicative
  inverse, if it exists, is unique.
\end{corollary}

\section{The Chinese Remainder Theorem}

\begin{example}\label{ex:crt-example}
  Consider the following problem:
  Find a positive integer having
  remainder $2$ when divided by $3$,
  remainder $1$ when divided by $4$,
  and remainder $3$ when divided by $5$.
  The problem can be rephrased as
  asking for a solution to the system
  of congruences:
  \[
    \begin{cases}
      x \equiv 2 \Pmod{3} \\
      x \equiv 1 \Pmod{4} \\
      x \equiv 3 \Pmod{5}.
    \end{cases}
  \]
\end{example}

\begin{theorem}[Chinese remainder theorem]
  Let $m_1, \dots, m_n$ be pairwise
  relatively prime positive integers,
  and let $b_1, \dots, b_n \in \Z$.
  Then the system of congruences
  \[
    \begin{cases}
      x \equiv b_1 \Pmod{m_1} \\
      x \equiv b_2 \Pmod{m_2} \\
      \quad \vdots \\
      x \equiv b_n \Pmod{m_n}
    \end{cases}
  \]
  has a unique solution modulo
  $M = m_1 \cdots m_n$.
\end{theorem}

\begin{proof}
  Let $M = m_1 \cdots m_n$ and
  $M_i = M / m_i$. Then
  $(M_i, m_i) = 1$, so there are solutions
  to each system
  $M_i x_i \equiv 1 \Pmod{m_i}$ given
  by $x_i = \overline{M}_i$. Consider
  \[
    x = b_1 M_1 \overline{M}_1
    + b_2 M_2 \overline{M}_2
    + \cdots
    + b_n M_n \overline{M}_n.
  \]
  Note that $m_i \mid M_j$ for
  $i \ne j$, so
  $x \equiv b_i M_i \overline{M}_i \equiv b_i \Pmod{m_i}$,
  so $x$ is a solution to the system.

  For uniqueness modulo $M$, let
  $x'$ be another solution. Then
  $x' \equiv b_i \Pmod{m_i}$ for
  each $1 \le i \le n$. Then
  \[
    x \equiv x' \Pmod{m_i}, \quad
    1 \le i \le n.
  \]
  Thus $m_i \mid x - x'$, so
  $M \mid x - x'$ since the $m_i$
  are pairwise relatively prime,
  so $x \equiv x' \Pmod{M}$.
\end{proof}

\begin{example}
  We now solve Example
  \ref{ex:crt-example}. Using the
  notation in the proof, we have
  \[
    (m_1, m_2, m_3) = (3, 4, 5), \quad
    (b_1, b_2, b_3) = (2, 1, 3), \quad
    M = 60,
    \quad
    (M_1, M_2, M_3) = (20, 15, 12).
  \]
  We still need to compute
  $\overline{M}_i$. In general, this
  can be done via the Euclidean algorithm.
  In this case,
  \[
    (\overline{M}_1, \overline{M}_2, \overline{M}_3)
    = (2, 3, 3).
  \]
  Now we can calculate the solution
  using
  \[
    x = b_1 M_1 \overline{M}_1
    + b_2 M_2 \overline{M}_2
    + b_3 M_3 \overline{M}_3
    = (2 \cdot 20 \cdot 2)
    + (1 \cdot 15 \cdot 3)
    + (3 \cdot 12 \cdot 3)
    = 233.
  \]
  Reducing modulo $60$, we get that the
  unique solution is given by
  $x \equiv 53 \Pmod{60}$.
\end{example}

\section{Wilson's Theorem}

\begin{lemma}\label{lem:self-inverse}
  Let $p$ be a prime and let
  $a \in \Z$. Then
  $a$ is its own inverse modulo $p$
  (i.e., $a \equiv \overline{a} \Pmod{p}$)
  if and only if $a \equiv \pm 1 \Pmod{p}$.
\end{lemma}

\begin{proof}
  $(\Rightarrow)$ Suppose
  $a \equiv \overline{a} \Pmod{p}$.
  Then $a^2 \equiv a \overline{a} \equiv 1 \Pmod{p}$, so
  $p \mid a^2 - 1 = (a - 1)(a + 1)$.
  Since $p$ is prime, we have
  $p \mid a - 1$ or $p \mid a + 1$, so
  $a \equiv \pm 1 \Pmod{p}$.

  $(\Leftarrow)$ This is obvious since
  $(\pm 1)^2 = 1$ in $\Z$, so they
  are also equal after reducing modulo $p$.
\end{proof}

\begin{theorem}[Wilson's theorem]\label{thm:wilson}
  Let $p$ be a prime. Then
  $(p - 1)! \equiv -1 \Pmod{p}$.
\end{theorem}

\begin{example}
  The idea behind the proof is the
  following: Concretely, if $p = 11$, we
  have
  \[
    (11 - 1)! =
    10 \cdot 9 \cdot 8 \cdot 7 \cdot 6 \cdot
    5 \cdot 4 \cdot 3 \cdot 2 \cdot 1
    \pmod{11}
  \]
  By Lemma \ref{lem:self-inverse},
  $10$ and $1$ are their own inverses
  modulo $11$.
  For each other integer $2 \le n \le 9$,
  we can pair them with their inverses:
  $(2, 6)$, $(3, 4)$, $(5, 9)$, $(7, 8)$.
  Then we can write
  \[
    (11 - 1)!
    \equiv (9 \cdot 5) \cdot (8 \cdot 7)
    \cdot (6 \cdot 2) \cdot (4 \cdot 3)
    \cdot 10 \cdot 1
    \equiv 10 \cdot 1 \equiv -1 \pmod{11}.
  \]
\end{example}

\begin{proof}[Proof of Theorem \ref{thm:wilson}]
  We can easily check the theorem
  for $p = 2, 3$, so suppose
  $p > 3$ is a prime. Then each $a$
  with $1 \le a \le p - 1$ has a unique
  inverse modulo $p$, and this inverse
  is distinct from $a$ if
  $2 \le a \le p - 2$. Pair each such
  integer with its inverse modulo $p$,
  say $a$ and $a'$. The product of
  all of these pairs is
  $(p - 2)!$, so
  $(p - 2)! \equiv 1 \Pmod{p}$.
  Thus $(p - 1)! \equiv p - 1 \equiv -1 \Pmod{p}$.
\end{proof}

\begin{prop}[Converse of Wilson's theorem]
  Let $n \in \Z$ with $n > 1$. If
  $(n - 1)! \equiv -1 \Pmod{n}$, then
  $n$ is prime.
\end{prop}

\begin{proof}
  Suppose $n = ab$ with $1 \le a < n$.
  It suffices to show that $a = 1$.
  Since $a < n$, we have
  $a \mid (n - 1)!$. Also,
  $n \mid (n - 1)! + 1$ by assumption,
  so $a \mid (n - 1)! + 1$ also since
  $a \mid n$. Thus
  \[
    a \mid ((n - 1)! + 1) - (n - 1)! = 1,
  \]
  so we must have $a = 1$.
\end{proof}

\begin{definition}
  A prime $p$ is a \emph{Wilson prime}
  if $(p - 1)! \equiv -1 \Pmod{p^2}$.
\end{definition}

\begin{example}
  The first few Wilson primes are
  $5, 13, 563$.
  In fact, these are the only known
  ones.
\end{example}
