\chapter{Oct.~8 --- The Legendre Symbol}

\section{More on the Legendre Symbol}

\begin{exercise}
  Find all the quadratic residues
  modulo $23$.

  We known that the quadratic residues modulo $23$ are the squares of
  $1, \dots, 11$, so
  \begin{align*}
    \{1^2, 2^2, 3^2, 4^2, 5^2, 6^2, 7^2, 8^2, 9^2, 10^2, 11^2\}
    &\equiv
    \{1, 4, 9, 16, 2, 13, 3, 18, 12, 8, 6\} \pmod{23} \\
    &= \{1, 2, 3, 4, 6, 8, 9, 12, 13, 16, 18\}
    \pmod{23}
  \end{align*}
  is the set of quadratic residues modulo $23$.
\end{exercise}

\begin{theorem}[Euler's criterion]
  Let $p$ be an odd prime with
  $a \in \Z$ and $p \nmid a$. Then
  \[
    \left( \frac{a}{p} \right) \equiv a^{(p - 1) / 2} \pmod{p}.
  \]
\end{theorem}

\begin{proof}
  Suppose first that
  $\big(\frac{a}{p}\big) = 1$.
  Then $x^2 \equiv a \pmod{p}$ has a
  solution, say $x = x_0$. Then
  \[
    a^{(p - 1) / 2}
    \equiv (x_0^2)^{(p - 1) / 2}
    \equiv x_0^{p - 1}
    \equiv 1
    \equiv \left( \frac{a}{p} \right)
    \pmod{p}
  \]
  by Fermat's little theorem. Now
  suppose that
  $\big(\frac{a}{p}\big) = -1$. Since
  $p \nmid a$, for each
  $1 \le i \le p - 1$,
  the linear congruence
  $i j \equiv a \Pmod{p}$ has a unique
  solution
  $j$ with $1 \le j \le p - 1$.
  Note that $i \ne j$ since
  $a$ is not a quadratic residue modulo $p$.
  Thus we can pair the residues
  $1, 2, \dots, p - 1$ into
  $(p - 1) / 2$ pairs $(i, j)$ such
  that $i j \equiv a \Pmod{p}$.
  Multiplying these pairs together,
  we get
  \[
    (p - 1)!
    \equiv 1 \cdot 2 \cdot 3 \cdots (p - 1)
    \equiv a^{(p - 1) / 2} \pmod{p}.
  \]
  The left-hand side is congruent to
  $-1 = \big(\frac{a}{p}\big)$
  by Wilson's theorem, which completes
  the proof.
\end{proof}

\begin{example}
  We compute
  $\big(\frac{3}{7}\big)$. Using
  Euler's criterion,
  \[
    \left( \frac{3}{7} \right)
    \equiv 3^{(7 - 1) / 2}
    \equiv 3^3
    \equiv 27
    \equiv 6
    \equiv -1 \pmod{7},
  \]
  so we get that
  $\big(\frac{3}{7}\big) = -1$.
\end{example}

\begin{prop}
  Let $p$ be an odd prime and
  $a, b \in \Z$ with
  $p \nmid a$ and $p \nmid b$. Then
  \begin{enumerate}
    \item $\big(\frac{a^2}{p}\big) = 1$;
    \item if $b \equiv a \Pmod{p}$, then
      $\big(\frac{b}{p}\big) = \big(\frac{a}{p}\big)$;
    \item $\big(\frac{ab}{p}\big)
      = \big(\frac{a}{p}\big)
      \big(\frac{b}{p}\big)$.
  \end{enumerate}
\end{prop}

\begin{proof}
  (1) The congruence
  $x^2 \equiv a^2 \Pmod{p}$
  has a solution $x = a$.

  (2) The congruence
  $x^2 \equiv a \Pmod{p}$
  is equivalent to the congruence
  $x^2 \equiv b \equiv a \Pmod{p}$.

  (3) By Euler's criterion,
  $\big(\frac{ab}{p}\big) \equiv (ab)^{(p - 1) / 2} \equiv a^{(p - 1) / 2} b^{(p - 1) / 2} \equiv \big(\frac{a}{p}\big) \big(\frac{b}{p}\big) \Pmod{p}$.
  Since $\big(\frac{ab}{p}\big)$,
  $\big(\frac{a}{p}\big)$, and
  $\big(\frac{b}{p}\big)$ are each
  $\pm 1$, congruence modulo $p$
  is equivalent to equality
  (since $1 \ne -1$ for $p \ge 3$).
\end{proof}

\begin{example}
  Calculate $\big(\frac{-11}{7}\big)$.
  Using the above properties and
  Euler's criterion, we have
  \[
    \left( \frac{-11}{7} \right)
    =
    \left( \frac{-1}{7} \right)
    \left( \frac{11}{7} \right)
    =
    \left( \frac{-1}{7} \right)
    \left( \frac{4}{7} \right)
    =
    \left( \frac{-1}{7} \right)
    \equiv (-1)^3 \equiv -1 \pmod{7}
  \]
  since $4$ is a quadratic residue
  modulo $7$. So
  $\big(\frac{-11}{7}\big) = -1$.
\end{example}

\section{Particular Cases of the Legendre Symbol}
\begin{remark}
  If $a = \pm 2^{a_0} p_1^{a_1} \cdots p_r^{a_r}$,
  then we have
  \[
    \left( \frac{a}{p} \right)
    =
    \left( \frac{\pm 1}{p} \right)
    \left( \frac{2}{p} \right)^{a_1}
    \left( \frac{p_1}{p} \right)^{a_1}
    \cdots
    \left( \frac{p_r}{p} \right)^{a_r}.
  \]
  Thus to evaluate $\big(\frac{a}{p}\big)$,
  it suffices to understand
  $\big(\frac{-1}{p}\big)$,
  $\big(\frac{2}{p}\big)$, and
  $\big(\frac{q}{p}\big)$ for odd primes $q$.
\end{remark}

\begin{theorem}
  Let $p$ be an odd prime. Then
  \[
    \left( \frac{-1}{p} \right)
    = (-1)^{(p - 1) / 2}
    = \begin{cases}
      1 & \text{if $p \equiv 1 \Pmod{4}$,} \\
      -1 & \text{if $p \equiv 3 \Pmod{4}$.}
    \end{cases}
  \]
\end{theorem}

\begin{proof}
  The first equality follows
  from Euler's criterion. The
  second is a direct computation:
  Note that $p$ can only be
  congruent to $1$ or $3$ modulo $4$.
  If $p \equiv 1 \Pmod{4}$, then
  $p = 1 + 4k$ for some $k \in \Z$. Then
  \[
    (-1)^{(p - 1) / 2}
    = (-1)^{(1 + 4k - 1) / 2}
    = (-1)^{2k}
    = 1.
  \]
  Similarly, if $p \equiv 3 \Pmod{4}$,
  then $p = 3 + 4k$ for some
  $k \in \Z$, and
  \[
    (-1)^{(p - 1) / 2}
    = (-1)^{(3 + 4k - 1) / 2}
    = (-1)^{1 + 2k}
    = -1.
  \]
  This proves the second equality.
\end{proof}

\begin{lemma}[Gauss's lemma]
  Let $p$ be an odd prime and let
  $a \in \Z$ with $p \nmid a$. Let
  $n$ be the number of least positive
  residues of the integers
  \[
    a, \quad 2a, \quad 3a, \quad \dots, \quad
    \left(\frac{p - 1}{2}\right) a
  \]
  that are greater than $p / 2$. Then
  $\big(\frac{a}{p}\big) = (-1)^n$.
\end{lemma}

\begin{proof}
  Let $r_1, \dots, r_n$ be the least
  positive residues among
  $a, 2a, \dots, ((p - 1) / 2)a$
  that are greater than $p / 2$, and
  let $s_1, \dots, s_m$ be the residues
  which are less than $p / 2$.
  Note that none of the $r_i, s_j$
  are congruent to $0$ modulo $p$
  since $p \nmid a$. Now consider the
  $(p - 1) / 2$ integers given by
  \[
    p - r_1, \quad p - r_2, \quad,
    \dots, \quad p - r_n, \quad
    s_1, \quad s_2, \quad \dots, \quad s_m.
  \]
  We claim that this is the set of
  residues $1, 2, \dots, (p - 1) / 2$
  in some order. All elements
  are $\ge 1$, and are $\le (p - 1) / 2$ 
  since they are $< p / 2$ and
  are integers. So it suffices to
  show that there are no duplicates.

  If $p - r_i \equiv p - r_j \Pmod{p}$,
  then $r_i \equiv r_j \Pmod{p}$, so
  $k_i a \equiv k_j a \Pmod{p}$
  for some $k_i \ne k_j$. Since
  $(a, p) = 1$, we can multiply by its
  inverse $\overline{a}$ to get
  $k_i \equiv k_j \Pmod{p}$, which is
  a contradiction.
  By a similar argument, the
  $s_j$ are all distinct.
  It only remains to consider
  $p - r_i \equiv s_j \Pmod{p}$. Then
  \[
    -k_i a \equiv k_j a \pmod{p}
  \]
  for some $1 \le k_i, k_j \le (p - 1) / 2$.
  The congruence then implies
  $-k_i \equiv k_j \Pmod{p}$. But
  $p - k_i > p / 2 \ge (p - 1) \ge k_j$,
  so this congruence is impossible.
  This proves the claim.

  Thus, multiplying all the numbers
  together gives
  \begin{align*}
    \left(\frac{p - 1}{2}\right)!
    \equiv (p - r_1) \cdots (p - r_n)
    s_1 \cdots s_m
    &\equiv (-1)^n a \cdot 2a \cdots \left(\frac{p - 1}{2}\right) a \\
    &\equiv (-1)^n a^{(p - 1) / 2} \left(\frac{p - 1}{2}\right)!
    \pmod{p}.
  \end{align*}
  Since $p \nmid ((p - 1) / 2)!$,
  we can multiply by its inverse to
  get $a^{(p - 1) / 2} \equiv (-1)^n \Pmod{p}$.
  The result then follows since the
  left-hand side is congruent to
  $\big(\frac{a}{p}\big)$ by Euler's
  criterion and $\big(\frac{a}{p}\big),
  (-1)^n$
  are $\pm 1$.
\end{proof}

\begin{example}
  We use Gauss's lemma to calculate
  $\big(\frac{6}{11}\big)$. We have
  $\big(\frac{6}{11}\big) = (-1)^n$,
  where $n$ is the number of least
  positive residues among
  \[
    6, \quad 2 \cdot 6, \quad
    3 \cdot 6, \quad 4 \cdot 6, \quad
    5 \cdot 6
  \]
  that are larger than $11 / 2 = 5.5$.
  Reducing the above gives
  $\{6, 1, 7, 2, 8\}$, so
  $n = 3$. Thus
  $\big(\frac{6}{11}\big) = -1$.
\end{example}
