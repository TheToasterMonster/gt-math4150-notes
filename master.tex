\documentclass[12pt, letterpaper, oneside]{book}
\usepackage[margin={0.6in, 0.75in}]{geometry}
\usepackage{microtype}
% \usepackage{kpfonts}
\usepackage{amsmath, amssymb, amsthm}
\usepackage{parskip}
\usepackage[many]{tcolorbox}
\usepackage{footnote}
\usepackage{cancel}
\usepackage{titlesec}
\usepackage{pgffor}
\usepackage[shortlabels, inline]{enumitem}
\usepackage{hyperref}
\usepackage{tikz-cd}

\usepackage[overload]{textcase}

\renewcommand{\chaptername}{Lecture}
\newtheorem{axiom}{Axiom}[chapter]
\newtheorem{theorem}{Theorem}[chapter]
\newtheorem{prop}{Proposition}[chapter]
\newtheorem{corollary}{Corollary}[theorem]
\newtheorem{lemma}{Lemma}[chapter]
\newtheorem{conjecture}{Conjecture}[theorem]
\theoremstyle{definition}
\newtheorem{definition}{Definition}[chapter]
\newtheorem{exercise}{Exercise}[chapter]
\newtheorem{example}{Example}[definition]
\newtheorem*{remark}{Remark}

\tcbset{sharp corners, breakable, enhanced, parbox=false}
\newtcolorbox{mybox}[3][]
{
  colframe = #2!150,
  colback  = #2!5,
  coltitle = #2!0!white,  
  title    = {#3},
  #1,
}

\titleformat{\chapter}[display]
    {\normalfont\huge\bfseries}{\chaptertitlename\ \thechapter}{20pt}{\Huge}
\titlespacing*{\chapter}{0pt}{0pt}{40pt}

\newcommand{\R}{\mathbb{R}}
\newcommand{\N}{\mathbb{N}}
\newcommand{\Z}{\mathbb{Z}}
\newcommand{\C}{\mathbb{C}}
\newcommand{\Q}{\mathbb{Q}}
\newcommand{\F}{\mathbb{F}}
\newcommand{\Mod}[1]{\ {\mathrm{mod}\ #1}}
\newcommand{\Pmod}[1]{\ (\mathrm{mod}\ #1)}

\DeclareMathOperator{\ord}{ord}
\DeclareMathOperator{\lcm}{lcm}
\DeclareMathOperator{\re}{Re}
\DeclareMathOperator{\im}{Im}

\title{MATH 4150: Introduction to Number Theory}
\author{Frank Qiang\\Instructor: Joshua Stucky}
\date{Georgia Institute of Technology\\Fall 2025}

\begin{document}
  \maketitle

  \begingroup
  \let\cleardoublepage\clearpage
  \tableofcontents
  \endgroup

  % \foreach \i in {00, 01, 02, 03, 04, ..., 50} {%
  %   \edef\FileName{lectures/lecture\i.tex}%     The % here are necessary to eliminate any
  %   \IfFileExists{\FileName}{%  spurious spaces that may get inserted
  %      \input{\FileName}%       at these points
  %   }
  % }
  \chapter{Aug.~18 --- Divisibility}

\begin{quote}
  \emph{Something something pair a' docks.} (I forgot to write it down oops.)
\end{quote}

\section{Basic Properties of Divisibility}

\begin{definition}
  Let $a, b \in \Z$. We say that
  $a$ \emph{divides} $b$, and we write
  $a \mid b$, if there exists $c \in \Z$
  such that $b = ac$. We also say that
  $a$ is a \emph{divisor} (or \emph{factor})
  of $b$. We write
  $a \nmid b$ if $a$ does not divide $b$.
\end{definition}

\begin{example} We have the following:
  \begin{enumerate}
    \item We have $3 \mid 6$ since
      $6 = 3 \cdot 2$, and $3 \mid -6$ since
      $-6 = 3 \cdot (-2)$.
    \item For any $a \in \Z$, we have $a \mid 0$
      since $0 = a \cdot 0$.
    \item Technically, we have
      $0 \mid 0$, but do not confuse this
      with the indeterminate form $0 / 0$.
  \end{enumerate}
\end{example}

\begin{prop}
  Let $a, b, c \in \Z$. If $a \mid b$ and
  $b \mid c$, then $a \mid c$. In particular,
  divisibility is transitive.
\end{prop}

\begin{proof}
  Since $a \mid b$ and $b \mid c$, there
  exist integers $e, f$ such that
  $b = ae$ and $c = bf$. We can write
  \[
    c = bf = (ae)f = a(ef),
  \]
  so that $a$ divides $c$ by definition.
\end{proof}

\begin{prop}
  Let $a, b, c, m, n \in \Z$. If $c \mid a$
  and $c \mid b$, then $c \mid (am + bn)$. In
  other words, $c$ divides any integral
  linear combination of $a$ and $b$.
\end{prop}

\begin{proof}
  Since $c \mid a$ and $c \mid b$, we have
  $a = ce$ and $b = cf$ for some $e, f \in \Z$.
  Then
  \[
    am + bn = (ce)m + (cf)n
    = c(em + fn),
  \]
  so that $c$ divides $am + bn$ by definition.
\end{proof}

\section{The Division Algorithm}

\begin{definition}
  Let $x \in \R$. The
  \emph{greatest integer function} (or
  \emph{floor function}) of
  $x$, denoted $[x]$ (or
  $\lfloor x \rfloor$),
  is the greatest integer less than or equal
  to $x$.
\end{definition}

\begin{example}
  We have the following:
  \begin{enumerate}
    \item If $a \in \Z$, then $[a] = a$.
      The converse is also true:
      If $[a] = a$ for $a \in \R$, then
      $a \in \Z$.
    \item We have
      $[\pi] = 3$, $[e] = 2$, $[-1.5] = -2$,
      and $[-\pi] = -4$.
  \end{enumerate}
\end{example}

\begin{lemma}\label{lem:floor-bound}
  Let $x \in \R$. Then $x - 1 < [x] \le x$.
\end{lemma}

\begin{proof}
  The upper bound is obvious. To show the
  lower bound, suppose to the contrary that
  $[x] \le x - 1$. Then $[x] < [x] + 1 \le x$,
  which contradicts the maximality of $[x]$
  as $[x] + 1$ is an integer.
\end{proof}

\begin{example}
  We can write $5 = 3 \cdot 1 + 2$ and
  $26 = 6 \cdot 4 + 2$; this is
  the \emph{division algorithm}.
\end{example}

\begin{theorem}[Division algorithm]
  Let $a, b \in \Z$ with $b > 0$. Then there
  exist unique $q, r \in \Z$ such
  that
  \[
    a = bq + r, \quad 0 \le r < b.
  \]
  Call $q$ the \emph{quotient} and
  $r$ the \emph{remainder} of the division.
\end{theorem}

\begin{proof}
  First we show existence. Let $q = [a / b]$
  and $r = a - b[a / b]$. By construction,
  $a = bq + r$. To check that
  $0 \le r < b$, note that by Lemma
  \ref{lem:floor-bound}, we have
  $a / b - 1 < [a / b] \le a / b$. Multiplying
  by $-b$ gives
  \[
    -a \le -b[a / b] < b - a,
  \]
  and adding $a$ gives the desired inequality
  $0 \le a - b[a / b] = r < b$.

  Now we prove uniqueness. Assume there are
  $q_1, q_2, r_1, r_2 \in \Z$ such that
  \[
    a = bq_1 + r_1 = bq_2 + r_2, \quad
    0 \le r_1, r_2 < b.
  \]
  Then $0 = (bq_1 + r_1) - (bq_2 + r_2) = b(q_1 - q_2) + (r_1 - r_2)$, so we find that
  \[
    r_2 - r_1 = b(q_1 - q_2).
  \]
  So $b \mid r_2 - r_1$. But
  $0 \le r_1, r_2 < b$
  implies $-b < r_2 - r_1 < b$, so
  we must have $r_2 - r_1 = 0$, i.e.
  $r_1 = r_2$. This then implies
  $0 = b(q_1 - q_2)$, which gives
  $q_1 - q_2 = 0$ since $b > 0$, so
  $q_1 = q_2$ as well.
\end{proof}

\begin{remark}
  In the division algorithm, we have $r = 0$
  if and only if $b \mid a$.
\end{remark}

\begin{example}
  Suppose $a = -5$, $b = 3$. Then
  $q = [a / b] = -2$ and
  $r = a - b[a / b] = 1$, i.e.
  \[
    -5 = 3 \cdot (-2) + 1.
  \]
  Note that $-5 = 3 \cdot (-1) + (-2)$ also, but
  this does not contradict uniqueness since
  $-2 \notin [0, 3)$.
\end{example}

\begin{definition}
  Let $n \in \Z$. Then $n$ is \emph{even}
  if $2 \mid n$, and $\emph{odd}$ otherwise.
\end{definition}

  \chapter{Aug.~20 --- Prime Numbers}

\begin{quote}
  \emph{Two fish are in a tank. One says to the other, "Ha, how do you drive this thing?"}
\end{quote}

\section{Prime Numbers}

\begin{definition}
  Let $p \in \Z$ with $p > 1$. Then $p$
  is \emph{prime} if the only positive
  divisors of $p$ and $1$ and $p$.
  If $n \in \Z$, $n > 1$ and $n$ is not
  prime, then $n$ is \emph{composite}.
\end{definition}

\begin{remark}
  The number $1$ is neither prime
  nor composite.
\end{remark}

\begin{example}
  The following are prime numbers:
  $2, 3, 5, 7, 11, 13, 17, 19, 23, 29, 31, 37, 41, 43, 47, \dots$.
\end{example}

\begin{lemma}\label{lem:has-prime-divisor}
  Every integer greater than $1$ has a
  prime divisor.
\end{lemma}

\begin{proof}
  Assume to the contrary that there exists
  $n > 1$ that has no prime divisor.
  By the well-ordering principle,\footnote{The \emph{well-ordering principle} says that every nonempty subset of the positive integers contains a least element.}
  we may take $n$ to be the smallest
  such positive integer. Since $n$
  has no prime divisors, $n$ cannot be
  prime. Thus $n$ has a divisor $a$
  with $1 < a < n$. Since $1 < a < n$,
  $a$ must have a prime divisor $p$ by the
  minimality of $n$. But then $p \mid a$
  and $a \mid n$, so $p \mid n$ by
  transitivity,
  a contradiction.
\end{proof}

\begin{theorem}[Euclid]
  There are infinitely many prime numbers.
\end{theorem}

\begin{proof}
  Assume to the contrary that there are
  only finitely many primes
  $p_1, p_2, \dots, p_n$.
  Consider
  \[
    N = p_1 p_2 \cdots p_n + 1.
  \]
  By Lemma \ref{lem:has-prime-divisor},
  $N$ has a prime divisor $p = p_j$
  for some $1 \le j \le n$. Since
  $p$ divides $N$ and $p$ divides
  $p_1 p_2 \cdots p_n$,
  $p$ also divides
  $N - p_1 p_2 \cdots p_n = 1$,
  which is a contradiction.
\end{proof}

\begin{exercise}
  Modify the proof and construct
  infinitely many problematic $N$.
\end{exercise}

\section{Sieve of Eratosthenes}

\begin{prop}
  If $n$ is composite, then $n$ has a
  prime divisor that is less than or
  equal to $\sqrt{n}$.
\end{prop}

\begin{proof}
  Since $n$ is composite, $n = ab$ where
  $1 < a, b < n$. Without loss of
  generality, assume $a \le b$. We claim
  $a \le \sqrt{n}$. To see this,
  suppose to the contrary that
  $a > \sqrt{n}$.
  Then $n = ab \ge a^2 > n$, a
  contradiction.
  By Lemma
  \ref{lem:has-prime-divisor},
  $a$ has a prime divisor
  $p \le a \le \sqrt{n}$. But then
  $p \mid a$ and $a \mid n$, so $p \mid n$.
\end{proof}

\begin{remark}
  The proposition implies that if
  all the prime divisors of an integer
  $n$ are greater than $\sqrt{n}$,
  then $n$ is prime. So to check the
  primality of $n$,
  it suffices to check divisibility
  by primes $\le \sqrt{n}$.
\end{remark}

\begin{example}
  The \emph{sieve of Eratosthenes} proceeds
  as follows. To find primes $\le 50$,
  we can delete multiples of primes
  $\le \sqrt{50} \approx 7.07$. To start,
  we know that $2$ is prime. Then cross
  out all multiples of $2$. The smallest
  number remaining is $3$, which we now
  know must be prime. Then cross out all
  multiples of $3$. Continue this process
  until we cross out all multiples of
  $7$, and then all remaining numbers
  are prime.
\end{example}

\section{Gaps in Primes}
\begin{prop}
  For any positive integer $n$, there are
  at least $n$ consecutive composite
  positive integers.
\end{prop}

\begin{proof}
  Consider the following list of $n$
  consecutive numbers:
  \[
    (n + 1)! + 2,\quad (n + 1)! + 3,\quad
    (n + 1)! + 4,\quad
    \dots,\quad (n + 1)! + (n + 1).
  \]
  Note that for any $2 \le m \le n + 1$,
  we have $m \mid m$ and $m \mid (n + 1)!$,
  so $m$ divides $(n + 1)! + m$.
  Thus each number in the above list
  is composite, so we have at least
  $n$ consecutive composite integers.
\end{proof}

\begin{remark}
  With some modifications to this proof
  (namely a more ``efficient''
  construction), one can find asymptotic
  lower bounds for the length of long
  prime gaps.
\end{remark}

\begin{conjecture}
  There are infinitely many pairs
  of primes that differ by exactly $2$.
\end{conjecture}

\begin{remark}
  Zhang (2013) was able to show that
  there are infinitely many pairs of
  pairs of primes whose difference is
  $\le 70,000,000$. This has been
  lowered to $246$ by the Polymath project,
  which included Tao and Maynard. Assuming
  other strong conjectures
  (Elliot-Halberstam), we can get
  down to $6$.
\end{remark}

\begin{remark}
  In addition to long and short prime gaps,
  we can also consider the average length
  of prime gaps. Gauss conjectured that
  as $x \to \infty$, the number of primes
  $\le x$, denoted $\pi(x)$, satisfies
  \[
    \pi(x) \sim \frac{x}{\log x},
  \]
  i.e. $\pi(x)$ is asymptotic
  to $x / {\log x}$. Said
  differently, this says that the
  ``probability'' that an integer $\le x$
  is prime is $\pi(x) / x \sim 1 / {\log x}$.
  This conjecture was proved independently
  in 1896 by de la Vall\'e-Poussin and
  Hadamard, and is now known as the
  \emph{prime number theorem}.
\end{remark}

\begin{definition}
  Let $x \in \R$. Define
  $\pi(x) = |{\{p : \text{$p$ prime}, p \le x\}}|$.
\end{definition}

\begin{theorem}[Prime number theorem]
  As $x \to \infty$,
  $\pi(x)$ is asymptotic to $x / {\log x}$,
  i.e.
  \[
    \lim_{x \to \infty}
    \frac{\pi(x)}{x / {\log x}} = 1.
  \]
\end{theorem}

\section{Other Open Problems}

\begin{conjecture}[Goldbach]
  Every even integer $\ge 4$ is a sum
  of two primes.
\end{conjecture}

\begin{theorem}[Ternary Goldbach]
  Every odd integer $\ge 7$ is a sum of
  three primes.
\end{theorem}

\begin{remark}
  Goldbach's conjecture implies
  ternary Goldbach (subtract $3$), but not
  vice versa.
\end{remark}

\begin{definition}
  Primes of the form $p = 2^n - 1$
  are called \emph{Mersenne primes}, and
  primes of the form $p = 2^{2^n} + 1$
  are called \emph{Fermat primes}.
\end{definition}

\begin{conjecture}
  There are infinitely many Mersenne
  primes but only finitely many
  Fermat primes.
\end{conjecture}

  \chapter{Aug.~25 --- Greatest Common Divisors}

\begin{quote}
  \emph{What do you call a root vegetable, fresh off the oven, and a pig that you throw off the balcony? One is a heated yam, and the other is a yeeted ham.}
\end{quote}

\section{Greatest Common Divisors}

\begin{remark}
  Given $a, b \in \Z$, not both zero,
  we can
  consider the set
  \[S = \{c \in \Z : c \mid a \text{ and } c \mid b\},\]
  of common divisors of both $a$ and $b$.
  Note that $\pm 1 \in S$, so $S$ is
  nonempty, and $S$ is also finite as
  at least one of $a, b$ is nonzero.
  Thus $S$ has a maximal element.
\end{remark}

\begin{definition}
  Let $a, b \in \Z$, not both zero. Then
  the \emph{greatest common divisor}
  of $a$ and $b$, denoted $(a, b)$,
  is the largest integer $d$ such that
  $d \mid a$ and $d \mid b$.
  If $(a, b) = 1$, then we say that
  $a, b$ are \emph{relatively prime} (or
  \emph{coprime}).
\end{definition}

\begin{remark}
  Note that $(0, 0)$ is not defined.
  Also note that if $(a, b) = d$, then
  \[
    (a, b) = (-a, b) = (a, -b)
    = (-a, -b) = d.
  \]
\end{remark}

\begin{example}
  We will compute $(24, 60)$. The
  list of positive divisors of $24$ and
  $60$ are
  \begin{align*}
    24 &: 1, 2, 3, 4, 6, 8, 12, 24; \\
    60 &: 1, 2, 3, 4, 5, 6, 10, 12, 15, 20, 30, 60.
  \end{align*}
  We can then see that $(24, 60) = 12$.
\end{example}

\begin{remark}
  In general, we have $(a, 0) = |a|$.
\end{remark}

\begin{prop}
  Let $(a, b) = d$. Then
  $(a / d, b / d) = 1$.
\end{prop}

\begin{proof}
  Let $d' = (a / d, b / d) > 0$. Then
  $d' \mid (a / d)$ and $d' \mid (b / d)$,
  so there exist $e, f$ such that
  $a / d = ed'$ and $b / d = fd'$.
  We can write this as $a = e d' d$ and
  $b = f d' d$. Thus
  $d' d$ is a common divisor of
  $a$ and $b$, so
  we must have $d' = 1$ by the maximality
  of $d$.
\end{proof}

\begin{prop}
  Let $a, b \in \Z$, not both zero,
and let \[T = \{ma + nb : m, n \in \Z, ma + nb > 0\}.\] Then
  $\min T$ exists and is equal to
  $(a, b)$.
\end{prop}

\begin{proof}
  Without loss of generality, we can
  assume $a \ne 0$. Note that
  $|a| \in T$, so $T$ is nonempty. Thus
  by the well-ordering principle, $T$
  has a minimal element $d$.
  Then $d = m' a + n' b$ for some
  $m', n' \in \Z$. We will show that
  $d \mid a$, a similar argument
  shows that $d \mid b$. By the
  division algorithm, we may write
  \[
    a = dq + r, \quad 0 \le r < d.
  \]
  It suffices to show that $r = 0$.
  We can rewrite the above as
  \[
    r
    = a - dq
    = a - (m' a + n' b)q
    = a(1 - m' q) - b(n' q).
  \]
  So $r$ is an integral linear combination
  of $a, b$. Since
  $d$ is the smallest positive
  integral linear combination of
  $a, b$ and $0 \le r < d$, we must
  have $r = 0$.
  So $d$ is a common divisor of $a, b$.

  Now suppose $c \mid a$ and $c \mid b$,
  then $c \mid (ma + nb)$, so
  $c$ divides $d = m' a + n' b$.
  Thus $c \le d$, so $d = (a, b)$.
\end{proof}

\begin{remark}
  If $(a, b) = d$, then $d = ma + nb$
  for some $m, n \in \Z$. If $d = 1$, then
  the converse also holds: If
  \[
    1 = ma + nb,
  \]
  and $d'$ is a common divisor of $a, b$,
  then $d' \mid 1$, so $d' = 1$.
\end{remark}

\begin{remark}
  Along the way, we showed that any
  common divisor of $a, b$ divides
  $(a, b)$.
\end{remark}

\begin{definition}
  Let $a_1, \dots, a_n \in \Z$, with at
  least one nonzero. Then the
  \emph{greatest common divisor}
  of $a_1, \dots, a_n$, denoted
  $(a_1, \dots, a_n)$, is the
  largest integer $d$ such that
  $d \mid a_i$ for $1 \le i \le n$.
  If $(a_1, \dots, a_n) = 1$, then we say
  that $a_1, \dots, a_n$ are
  \emph{relatively prime}, and if
  $(a_i, a_j) = 1$ for all $1\le i \ne j \le n$,
  then we say that $a_1, \dots, a_n$
  are \emph{pairwise relatively prime}.
\end{definition}

\begin{remark}
  Pairwise relatively prime implies
  relatively prime, but the converse
  is not true (e.g. $\{2, 4, 3\}$).
\end{remark}

\section{The Euclidean Algorithm}

\begin{lemma}
  If $a, b \in \Z$ with $0 < b \le a$ and
  $a = bq + r$ with $q, r \in \Z$, then
  $(a, b) = (r, b)$.
\end{lemma}

\begin{proof}
  It suffices to show that the two sets of
  common divisors (of $a, b$ and of $r, b$)
  are the same. Denote by $S_1$ and $S_2$
  these two sets, respectively. First let
  $c \in S_1$, so $c \mid a$ and $c \mid b$.
  We can write
  \[
    r = a - bq,
  \]
  so we have $c \mid r$. Thus
  $c \in S_2$, so $S_1 \subseteq S_2$.
  Now let $c \in S_2$, so $c \mid r$ and
  $c \mid b$. We have
  \[
    a = bq + r
  \]
  by hypothesis, so $c \mid a$, i.e.
  $c \in S_1$.
  Thus $S_1 = S_2$, so
  $(a, b) = \max S_1 = \max S_2 = (r, b)$.
\end{proof}

\begin{example}
  The above lemma allows us to compute
  greatest common divisors more efficiently.
  We will compute $(803, 154)$. We
  can write
  $803 = 5 \cdot 154 + 33$, so
  $(803, 154) = (154, 33)$.
  Continuing, we get
  \[
    (803, 154)
    = (154, 33)
    = (33, 22)
    = (22, 11)
    = (11, 0) = 11.
  \]
\end{example}

\begin{theorem}[Euclidean algorithm]
  Let $a, b \in \Z$ with $0 < b \le a$.
  Set $r_{-1} = a$, $r_0 = b$, and inductively
  write
  $r_{i - 1} = q_i r_i + r_{i + 1}$
  by the division algorithm for
  $n \ge 1$.
  Then $r_n = 0$ for some $n \ge 1$
  and $(a, b) = r_{n - 1}$.
\end{theorem}

\begin{proof}
  Note that $r_1 > r_2 > r_3 > \cdots$.
  If $r_n \ne 0$ for all $n \ge 1$, then
  this is a strictly decreasing infinite
  sequence of positive integers, which
  is not possible. So $r_n = 0$ for some
  $n \ge 1$. The conclusion
  $(a, b) = r_{n - 1}$ follows by repeatedly
  applying the lemma since
  $(a, b) = (r_i, r_{i + 1}) = (r_{n - 1}, 0) = r_{n - 1}$.
\end{proof}

\begin{example}
  By reversing this process, we can
  write $(a, b)$ explicitly as an integer
  linear combination of $a, b$.
  Using the previous example of computing
  $(803, 154)$, we can see that
  \begin{align*}
    (803, 154)
    &= 11
    = 33 - 1 \cdot 22 \\
    &= 33 - 1 \cdot (154 - 4 \cdot 33)
    = 5 \cdot 33 - 1 \cdot 154 \\
    &= 5 \cdot (803 - 5 \cdot 154) - 1 \cdot 154
    = 5 \cdot 803 - 26 \cdot 154.
  \end{align*}
  Thus we have found that
  $(803, 154) = 5 \cdot 803 - 26 \cdot 154$.
  Note that this representation
  is not unique, e.g.
  we can also write
  $11 = 19 \cdot 803 - 99 \cdot 154$.
  In fact, there are infinitely many such
  representations.
\end{example}

  \chapter{Aug.~27 --- Fundamental Theorem of Arithmetic}

\begin{quote}
  \emph{What's the difference between a mediocre clown and a rabbit in the gym? One's a bit funny, the other's a fit bunny.}
\end{quote}

\section{The Fundamental Theorem of Arithmetic}

\begin{lemma}[Euclid]\label{lem:prime-divides-product}
  Let $a, b \in \Z$ and let $p$ be a prime.
  If $p \mid ab$, then $p \mid a$
  or $p \mid b$.
\end{lemma}

\begin{proof}
  If $p \mid a$, then we are done, so
  assume $p \nmid a$. Then
  $(p, a) = 1$. Thus we can write
  $1 = ma + np$ for some $m, n \in \Z$.
  Since $p \mid ab$, we can write
  $ab = pc$ for some $c \in \Z$.
  Multiplying by $b$, we have
  \[
    b = bma + bnp
    = m(cp) + nb p
    = p(mc + nb).
  \]
  Thus we see that
  $p \mid b$, as desired.
\end{proof}

\begin{remark}
  This fails if $p$ is composite:
  Take $p = 6$, $a = 2$, and $b = 3$.
\end{remark}

\begin{exercise}
  Determine where the proof fails
  if $p$ is composite.
\end{exercise}

\begin{corollary}
  Let $a_1, \dots, a_n \in \Z$ and $p$
  a prime. If $p \mid a_1 \cdots a_n$,
  then $p \mid a_i$ for some $1 \le i \le n$.
\end{corollary}

\begin{proof}
  Induct on $n$. The base case $n = 1$
  is trivial. If $n = 2$, then this is
  just Lemma \ref{lem:prime-divides-product}.
  Now suppose $n \ge 2$, and we show the
  result for $n + 1$. Specifically,
  assume that if $p \mid a_1 \cdots a_n$,
  then $p \mid a_i$ for some
  $1 \le i \le n$. Suppose
  $p \mid a_1 \cdots a_n a_{n+1}$.
  Then $p \mid (a_1 \cdots a_n) a_{n+1}$.
  So by Lemma \ref{lem:prime-divides-product},
  we have $p \mid a_1 \cdots a_n$
  or $p \mid a_{n + 1}$. If $p \mid a_{n + 1}$, then we are done.
  Otherwise, $p \mid a_1 \cdots a_n$,
  so $p \mid a_i$ for some $1 \le i \le n$
  by the induction hypothesis.
  In particular, $p \mid a_i$ for some
  $1 \le i \le n + 1$, as desired.
\end{proof}

\begin{theorem}[Fundamental theorem of arithmetic]
  Every integer $m > 1$ may be
  expressed in the form
  $m = p_1^{a_1} \cdots p_n^{a_n}$
  where $p_1, \dots, p_n$ are distinct
  primes and $a_1, \dots, a_n$ are
  positive integers. This form is
  called the \emph{prime factorization}
  of the integer $m$. Moreover, this
  factorization is essentially unique, i.e.
  unique up to permutations
  of the factors $p_i^{a_i}$.
\end{theorem}

\begin{proof}
  We first prove existence.
  Assume to the contrary that there exists
  $m > 1$ that does not have a prime
  factorization. Without loss of generality,
  we can assume $m$ is the smallest such
  integer by the well-ordering principle.
  In particular, $m$ cannot be prime.
  So $m = ab$ for some $1 < a, b < m$. Then
  $a, b$ have prime factorizations. Thus
  so too does $m$, a contradiction.

  Now we prove uniqueness. Assume
  that $m = p_1^{a_1} \cdots p_n^{a_n} = q_1^{b_1} \cdots q_r^{b_r}$.
  Without loss of generality, we
  can assume $p_1 < p_2 < \cdots < p_n$
  and $q_1 < q_2 < \cdots < q_r$.
  We need to show that
  $n = r$, $p_i = q_i$ for each $i$, and
  $a_i = b_i$ for each $i$. Let $p_i \mid m$.
  Then $p_i \mid q_1^{b_1} \cdots q_r^{b_r}$,
  so $p_i \mid q_j$ for some $1 \le j \le r$.
  Thus $p_i = q_j$ since both are prime.
  Similarly, given $q_i$, we have
  $q_i = p_j$ for some $j$. Thus the
  primes in the two factorizations (as
  sets) are the same. Thus $n = r$, and
  by the ordering assumption, we have
  $p_i = q_i$ for each $1 \le i \le n$.
  So
  \[
    m = p_1^{a_1} \cdots p_n^{a_n}
    = p_1^{b_1} \cdots p_n^{b_n}.
  \]
  Suppose to the contrary that
  $a_i \ne b_i$ for some $i$. Without
  loss of generality, assume $a_i < b_i$.
  We have $p_i^{b_i} \mid m$, so
  $p_i^{b_i} \mid p_1^{a_1} \cdots p_{i - 1}^{a_{i - 1}} p_i^{a_i} p_{i + 1}^{a_{i + 1}} \cdots p_n^{a_n}$.
  Thus $p_i^{b_i - a_i} \mid p_1^{a_1} \cdots p_{i - 1}^{a_{i - 1}} p_{i + 1}^{a_{i + 1}} \cdots p_n^{a_n}$.
  Since $a_i < b_i$, we have
  $b_i - a_i > 0$, so
  $p_i \mid p_1^{a_1} \cdots p_{i - 1}^{a_{i - 1}} p_{i + 1}^{a_{i + 1}} \cdots p_n^{a_n}$
  by the transitivity of divisibility.
  Then $p_i \mid p_j$ for some $j \ne i$,
  so $p_i = p_j$, which is a contradiction
  since the $p_i$ are all distinct primes.
  This proves uniqueness.
\end{proof}

\begin{remark}
  This is one reason why we do not
  consider $1$ to be a prime, as we would
  lose uniqueness.
\end{remark}

\begin{example}
  We can write $60 = 2^2 \cdot 3 \cdot 5$
  and $756 = 2^2 \cdot 3^3 \cdot 7$.
\end{example}

\section{Least Common Multiples}

\begin{definition}
  Let $a, b \in \Z$ with $a, b > 0$.
  The \emph{least common multiple}
  of $a$ and $b$, denoted $[a, b]$, is the
  least positive integer $m$ such that
  $a \mid m$ and $b \mid m$.
\end{definition}

\begin{remark}
  Since $ab$ is a common multiple
  of $a$ and $b$, $[a, b]$ always
  exists by the well-ordering principle.
\end{remark}

\begin{example}
  We will compute $[6, 7]$. The multiples
  of $6$ and $7$ include:
  \begin{align*}
    6 &: 6, 12, 18, 24, 30, 36, 42, 48, \dots; \\
    7 &: 7, 14, 21, 28, 35, 42, 49, \dots.
  \end{align*}
  So we can see that $[6, 7] = 42 = 6 \cdot 7$.
  On the other hand, $[6, 8] = 24 \ne 6 \cdot 8$.
\end{example}

\begin{remark}
  The fundamental theorem of arithmetic
  can be used to calculate both GCDs and
  LCMs.
\end{remark}

\begin{prop}\label{prop:gcd-lcm-prime-factorization}
  Let $a, b \in \Z$ with $a, b > 1$.
  Write $a = p_1^{a_1} \cdots p_n^{a_n}$
  and $b = p_1^{b_1} \cdots p_n^{b_n}$,
  where the $p_i$ are distinct primes,
  and $a_i, b_i \ge 0$. Then we have
  \[
    (a, b) = p_1^{\min\{a_1, b_1\}} \cdots p_n^{\min\{a_n, b_n\}} \quad \text{and} \quad
    [a, b] = p_1^{\max\{a_1, b_1\}} \cdots p_n^{\max\{a_n, b_n\}}.
  \]
\end{prop}

\begin{proof}
  Left as an exercise.
\end{proof}

\begin{example}
  Calculate $(756, 2205)$ and
  $[756, 2205]$. We can write
  \[
    756 = 2^2 \cdot 3^3 \cdot 5^0 \cdot 7^1
    \quad \text{and} \quad
    2205 = 2^0 \cdot 3^2 \cdot 5^1 \cdot 7^2.
  \]
  So we have
  $(756, 2205) = 2^0 \cdot 3^2 \cdot 5^0 \cdot 7^1 = 63$ and
  $[756, 2205] = 2^2 \cdot 3^3 \cdot 5 \cdot 7^2 = 26460$.
\end{example}

\begin{lemma}\label{lem:min-max-sum}
  Given $x, y \in \R$, we have
  $\min\{x, y\} + \max\{x, y\} = x + y$.
\end{lemma}

\begin{proof}
  The result is obvious if $x = y$.
  Otherwise, one is the minimum and the
  other is the maximum.
\end{proof}

\begin{theorem}
  Let $a, b \in \Z$ with $a, b > 1$. Then
  $(a, b) [a, b] = ab$.
\end{theorem}

\begin{proof}
  Write $a = p_1^{a_1} \cdots p_n^{a_n}$
  and $b = p_1^{b_1} \cdots p_n^{b_n}$
  with $a_i, b_i \ge 0$ and $p_i$
  distinct. By Proposition
  \ref{prop:gcd-lcm-prime-factorization},
  \begin{align*}
    (a, b) [a, b]
    &= p_1^{\min\{a_1, b_1\}} \cdots p_n^{\min\{a_n, b_n\}}
      p_1^{\max\{a_1, b_1\}} \cdots p_n^{\max\{a_n, b_n\}} \\
    &= p_1^{\min\{a_1, b_1\} + \max\{a_1, b_1\}} \cdots
      p_n^{\min\{a_n, b_n\} + \max\{a_n, b_n\}}
    = p_1^{a_1 + b_1} \cdots p_n^{a_n + b_n}
    = ab,
  \end{align*}
  where the third equality follows
  from Lemma \ref{lem:min-max-sum}.
\end{proof}

  \chapter{Sept.~3 --- Congruences}

\begin{quote}
  \emph{No, Tony's the guy with no shins.}
\end{quote}

\section{Dirichlet's Theorem}

\begin{theorem}[Dirichlet's theorem on primes in arithmetic progressions]
  Let $a, b \in \Z$ with
  $a, b > 0$ and $(a, b) = 1$. Then
  the arithmetic progression
  $a, a + b, a + 2b, a + 3b, \dots$
  contains infinitely many primes.
\end{theorem}

\begin{remark}
  Setting $a = b = 1$ recovers the
  fact that there are infinitely many
  primes.
\end{remark}

\begin{remark}
  The general case of Dirichlet's theorem
  is difficult, but we can use the
  fundamental theorem of arithmetic
  to prove some special cases, e.g.
  when $a = 3$ and $b = 4$.
\end{remark}

\begin{lemma}\label{lem:4n+1-product}
  Let $a, b \in \Z$. If $a$ and $b$ are
  expressible as $4n + 1$, then so is
  their product $ab$.
\end{lemma}

\begin{proof}
  Let $a = 4m + 1$ and $b = 4n + 1$. Then
  \[
    ab = (4m + 1)(4n + 1)
    = 16mn + 4m + 4n + 1
    = 4(4mn + m + n) + 1,
  \]
  which proves the desired result.
\end{proof}

\begin{prop}
  There are infinitely many primes
  of the form $4n + 3$ with $n \ge 0$.
\end{prop}

\begin{proof}
  Assume to the contrary that there are
  finitely many primes of the form
  $4n + 3$, say $3, p_1, \dots, p_r$.
  Then consider the integer
  $N = 4p_1 \cdots p_r + 3$. The prime
  factorization of $N$ must contain a
  prime of the form $4n + 3$, since
  otherwise $N$ would be a product of
  primes of the form $4n + 1$, which
  must again be of the form $4n + 1$.
  Thus we have $3 \mid N$ or $p_i \mid N$
  for some $1 \le i \le r$.

  If $3 \mid N$, then $3 \mid N - 3 = 4p_1 \dots p_r$,
  which is a contradiction.
  Otherwise, $p_i \mid N$ for some
  $1 \le i \le r$, and we have
  $p_i \mid N - 4p_1 \cdots p_r = 3$,
  which is a contradiction as well.
\end{proof}

\begin{remark}
  The same proof does not work for
  primes of the form $4n + 1$, since a
  product of numbers of the form
  $4n + 3$ is not necessarily again
  of the form $4n + 3$.
\end{remark}

\section{Congruences}

\begin{definition}
  Let $a, b, m \in \Z$ with $m > 0$.
  Then we say that $a$ is \emph{congruent
  to $b$ modulo $m$}, and we write
  $a \equiv b \Pmod{m}$,
  if $m \mid (a - b)$. The integer
  $m$ is called the \emph{modulus}
  of the congruence. We write
  $a \not\equiv b \Pmod{m}$ if
  $a$ is not congruent to $b$ modulo $m$.
\end{definition}

\begin{example}
  We have $25 \equiv 1 \Pmod{4}$ and
  $25 \equiv 4 \Pmod{7}$.
\end{example}

\begin{prop}\label{prop:congruence-equiv}
  Congruence modulo $m$ is
  an equivalence relation on $\Z$.
\end{prop}

\begin{proof}
  Reflexivity is clear since
  $m \mid 0 = (a - a)$ any $a \in \Z$,
  so $a \equiv a \Pmod{m}$.
  For symmetry, suppose that
  $a \equiv b \Pmod{m}$. Then
  $m \mid a - b$. But then $m \mid (-1)(a - b) = b - a$, so
  $b \equiv a \Pmod{m}$ as well.

  Finally, for transitivity, suppose that
  $a \equiv b \Pmod{m}$ and
  $b \equiv c \Pmod{m}$. Then
  $m \mid a - b$ and $m \mid b - c$, so
  $m$ also divides their sum
  $m \mid (a - b) + (b - c) = a - c$, i.e.
  $a \equiv c \Pmod{m}$.
\end{proof}

\begin{remark}
  A consequence of Proposition
  \ref{prop:congruence-equiv}
  is that $\Z$ is partitioned into
  its equivalence classes under congruence
  modulo $m$. For $a \in \Z$, we
  write $[a]$ to denote the equivalence
  class of $a$ modulo $m$ (not to be
  confused with the floor function).
\end{remark}

\begin{example}
  The equivalence classes of $\Z$
  under congruence modulo $4$ are
  \begin{align*}
    [0]
    &= \{\dots, -8, -4, 0, 4, 8, \dots\}, \\
    [1]
    &= \{\dots, -7, -3, 1, 5, 9, \dots\}, \\
    [2]
    &= \{\dots, -6, -2, 2, 6, 10, \dots\}, \\
    [3]
    &= \{\dots, -5, -1, 3, 7, 11, \dots\}.
  \end{align*}
\end{example}

\begin{definition}
  A set of $m$ integers such that every
  integer is congruent modulo $m$ to
  exactly one integer of the set is
  called a \emph{complete residue system}
  modulo $m$.
\end{definition}

\begin{example}
  $\{0, 1, 2, 3\}$ is a complete
  residue system modulo $4$.
  So is $\{4, 5, -6, -1\}$.
\end{example}

\begin{prop}
  The set $\{0, 1, \dots, m - 1\}$
  is a complete residue system
  modulo $m$.
\end{prop}

\begin{proof}
  First we prove that every integer
  is congruent to one of
  $0, 1, \dots, m - 1$ modulo $m$.
  By the division algorithm, for any
  $a \in \Z$, there exist $q, r \in \Z$
  with $0 \le r \le m - 1$ such that
  $a = qm + r$. Thus we have
  $a - r = qm$, so $m \mid a - r$, i.e.
  $a \equiv r \Pmod{m}$. This proves
  existence since $r \in \{0, 1, \dots, m - 1\}$.

  Now we show uniqueness. Suppose
  $a \equiv r_1 \Pmod{m}$ and
  $a \equiv r_2 \Pmod{m}$ where
  $r_1, r_2 \in \{0, 1, \dots, m - 1\}$.
  By transitivity, we have
  $r_1 \equiv r_2 \Pmod{m}$, so
  $m \mid r_1 - r_2$. But
  $0 \le r_1, r_2 \le m - 1$, so
  \[
    -(m - 1) \le r_1 - r_2 \le m - 1,
  \]
  so we must have $r_1 - r_2 = 0$, i.e.
  $r_1 = r_2$. This proves uniqueness.
\end{proof}

\begin{definition}
  The set $\{0, 1, \dots, m - 1\}$ is
  called the set of \emph{least nonnegative residues}
  modulo $m$.
\end{definition}

\begin{prop}
  Let $a, b, c, d, m \in \Z$, $m > 0$
  such that $a \equiv b \Pmod{m}$
  and $c \equiv d \Pmod{m}$. Then
  \begin{enumerate}
    \item $a + c \equiv b + d \Pmod{m}$;
    \item $ac \equiv bd \Pmod{m}$.
  \end{enumerate}
\end{prop}

\begin{proof}
  Since $a \equiv b \Pmod{m}$ and
  $c \equiv d \Pmod{m}$, we have
  $m \mid b - a$ and $m \mid d - c$.
  Then $m$ divides
  \[
    (b - a) + (d - c)
    = (b + d) - (a + c),
  \]
  so we have $a + c \equiv b + d \Pmod{m}$.
  This proves (1).

  To prove (2), note that
  since $m \mid a - b$, we also have
  $m \mid c(a - b)$. Likewise,
  $m \mid d - c$ implies $m \mid b(d - c)$.
  Then $m$ divides the difference
  \[
    c(a - b) - b(d - c)
    = ac - bd,
  \]
  which shows that $ac \equiv bd \Pmod{m}$.
  This shows (2).
\end{proof}

\begin{remark}
  This shows that the congruence classes
  of $\Z$ modulo $m$ form a \emph{ring}.
\end{remark}

\begin{example}
  Consider the complete residue system
  $\{0, 1, 2, 3\}$ modulo $4$. Their
  squares mod $4$ are
  \[
    \{0^2, 1^2, 2^2, 3^2\}
    \equiv \{0, 1, 0, 1\}
    \equiv \{0, 1\} \pmod{4}.
  \]
\end{example}

  \chapter{Sept.~8 --- Congruences, Part 2}

\section{More on Congruences}
\begin{example}
  Compute a complete residue system
  modulo $5$ using
  \begin{itemize}
    \item only even numbers:
      $\{0, 2, 4, 6, 8\}$,
    \item only prime numbers:
      $\{2, 3, 5, 11, 19\}$.
  \end{itemize}
\end{example}

\begin{example}
  Compute a complete residue system
  modulo $5$ using only numbers
  $\equiv 1 \Pmod{4}$.
\end{example}

\begin{remark}
  Recall that the set of equivalence
  classes of $\Z$ modulo $m$ form a ring.
  In particular, we can construct
  addition and multiplication tables.
  For $m = 4$, this looks like:
  \begin{center}
    \begin{tabular}{c|cccc}
      + & 0 & 1 & 2 & 3 \\
      \hline
      0 & 0 & 1 & 2 & 3 \\
      1 & 1 & 2 & 3 & 0 \\
      2 & 2 & 3 & 0 & 1 \\
      3 & 3 & 0 & 1 & 2
    \end{tabular}
    \quad \quad \quad
    \begin{tabular}{c|cccc}
      $\times$ & 0 & 1 & 2 & 3 \\
      \hline
      0 & 0 & 0 & 0 & 0 \\
      1 & 0 & 1 & 2 & 3 \\
      2 & 0 & 2 & 0 & 2 \\
      3 & 0 & 3 & 2 & 1
    \end{tabular}
  \end{center}
  Addition modulo $5$ is similar, but
  the multiplication table for $m = 5$ is:
  \begin{center}
    \begin{tabular}{c|ccccc}
      $\times$ & 0 & 1 & 2 & 3 & 4 \\
      \hline
      0 & 0 & 0 & 0 & 0 & 0 \\
      1 & 0 & 1 & 2 & 3 & 4 \\
      2 & 0 & 2 & 4 & 1 & 3 \\
      3 & 0 & 3 & 1 & 4 & 2 \\
      4 & 0 & 4 & 3 & 2 & 1
    \end{tabular}
  \end{center}
  Recall that a ring with no zero divisors
  (nonzero elements $a, b$ such that $ab = 0$)
  is an \emph{integral domain}, in
  particular we see from the multiplication
  table that $\Z / 5\Z$ is an integral
  domain. Since a finite integral domain
  is automatically a \emph{field}, we
  see that $\Z / 5\Z$ is a field.
\end{remark}

\begin{prop}
  Let $a, b, c, m, \in \Z$ with $m > 0$.
  Then
  \[
    ca \equiv cb \Pmod{m}
    \quad \text{if and only if} \quad
    a \equiv b \Pmod{m / (m, c)}.
  \]
  In particular, if $m$ is prime, then
  $ca \equiv cb \Pmod{m}$
  if and only if $a \equiv b \Pmod{m}$
  for $c \not\equiv 0 \Pmod{m}$.
\end{prop}

\begin{proof}
  $(\Rightarrow)$ We have
  $ca \equiv cb \Pmod{m}$ if and only if
  $m \mid ca - cb = c(a - b)$. Let
  $d = (m, c)$. By the transitivity of
  divisibility, we have
  $(m / d) \mid (c / d)(a - b)$.
  But $(m / d, c / d) = 1$, so
  $(m / d) \mid a - b$. Then we
  have $a \equiv b \Pmod{m / d}$
  by the definition of congruence.

  $(\Leftarrow)$ Again let $d = (m, c)$.
  Then $a \equiv b \Pmod{m / d}$, so
  $(m / d) \mid a - b$. Then
  $m \mid d(a - b)$, and so
  \[
    m \mid d(a - b)(c / d)
    = c(a - b) = ca - cb,
  \]
  which means $ca \equiv cb \Pmod{m}$
  by the definition of congruence.
\end{proof}

\begin{remark}
  This shows that the congruence
  classes modulo $m$ form a field if
  and only if $m$ is prime.
\end{remark}

\section{Linear Congruences in One Variable}

\begin{definition}
  Let $a, b \in \Z$. A congruence of
  the form
  \[
    ax \equiv b \Pmod{m}
  \]
  is called a \emph{linear congruence} in
  the variable $x$.
\end{definition}

\begin{example}
  Consider the following linear congruences:
  \begin{itemize}
    \item $2x \equiv 3 \Pmod{4}$
      has no solutions;
    \item $2x \equiv 4 \Pmod{6}$
      has $x = 2, 5$ as solutions;
    \item $3x \equiv 9 \Pmod{6}$ has
      $x = 1, 3, 5$ as solutions.
  \end{itemize}
\end{example}

\begin{theorem}
  Let $ax \equiv b \Pmod{m}$, and
  let $d = (a, m)$. If $d \nmid b$, then
  there are no solutions for $x$ in $\Z$. If
  $d \mid b$, then the congruence
  has exactly $d$ incongruent solutions
  modulo $m$ in $\Z$.
\end{theorem}

\begin{proof}
  Note that $ax \equiv b \Pmod{m}$ if
  and only if $m \mid ax - b$, if
  and only if $ax - b = my$ for some
  integer $y$. This is equivalent to
  $ax - my = b$. Thus $ax \equiv b \Pmod{m}$
  is solvable in $x$ if and only if
  the equation $ax - my = b$ is solvable
  in $x, y$.

  Let $x, y$ be a solution of
  $ax - my = b$. Since $d \mid a$ and
  $d \mid m$, we must have $d \mid b$.
  Taking contrapositives, this
  proves the first part of the theorem.

  Assume now that $d \mid b$. We prove
  the second part in 4 steps:
  \begin{enumerate}
    \item We will show that $ax \equiv b \Pmod{m}$
      has a solution $x_0$.
    \item We will show that there are
      infinitely many solutions
      of a particular form involving $x_0$.
    \item We will show that
      any solution has a particular
      form involving $x_0$. (Note that
      this combines with $(2)$ to give
      all possible solutions.)
    \item We will show that there are
      exactly $d$ equivalence classes
      of solutions.
  \end{enumerate}

  $(1)$ Since $d = (a, m)$, there exist
  $r, s \in \Z$ such that $d = ra + sm$.
  Since $d \mid b$, we can write
  \[
    b = \frac{b}{d} \cdot d
    = \frac{b}{d}(ra + sm)
    = \frac{br}{d} \cdot a + \frac{bs}{d} \cdot m.
  \]
  Thus $b - a(b r/ d) = (b s/ d) m$,
  so $m \mid b - a(b r/ d)$, so
  $a(br / d) \equiv b \Pmod{m}$.
  Thus $x_0 = br / d$ is a solution.

  $(2)$ Let $x_0$ be any solution of
  $ax \equiv b \Pmod{m}$. Consider
  $x_0 + (m / d) n$ for $n \in \Z$. Then
  \[
    a (x_0 + (m / d) n)
    \equiv a x_0 + a (m / d) n
    \equiv b + (a / d)mn
    \equiv b \pmod{m},
  \]
  so $x_0 + (m / d) n$ is also solution
  for any $n \in \Z$.

  $(3)$ Let $x_0$ be a solution of
  $ax \equiv b \Pmod{m}$. Recall from the
  beginning of the proof that this is
  equivalent to there being
  $y_0 \in \Z$ such that
  $ax_0 - my_0 = b$. Let $x$ be any other
  solution. Then $ax - my = b$
  for some $y \in \Z$, so
  \[
    0 = b - b = (ax_0 - my_0) - (ax - my)
    = a(x_0 - x) - m(y_0 - y),
  \]
  which gives $a(x_0 - x) = m(y_0 - y)$.
  This is equivalent to
  $(a / d)(x_0 - x) = (m / d)(y_0 - y)$.
  Note that if $y_0 - y = 0$, then
  $x_0 - x = 0$ as well since
  $a / d \ne 0$. So we may assume
  $y_0 - y \ne 0$. Then
  \[(m / d) \mid (a / d)(x_0 - x),\]
  and since $(a / d, m / d) = 1$, we have
  $(m / d) \mid (x_0 - x)$. Thus
  $x \equiv x_0 \Pmod{m / d}$. In
  particular, all solutions to
  $ax \equiv b \Pmod{m}$ are given by
  $x = x_0 + (m / d) n$ for
  $n \in \Z$ and any particular
  solution $x_0$.

  $(4)$ Let $x_0 + (m / d) n_1$ and
  $x_0 + (m / d) n_2$ be solutions.
  Then we have
  \[
    x_0 + (m / d) n_1
    \equiv x_0 + (m / d) n_2 \Pmod{m}
  \]
  if and only if
  $(m / d) n_1 \equiv (m / d) n_2 \Pmod{m}$.
  This happens if and only if
  $m \mid (m / d)(n_1 - n_2)$, if and only if
  $(m / d) (n_1 - n_2) = km$ for some
  $K \in \Z$, if and only if
  $n_1 - n_2 = kd$. In particular, this
  is equivalent to $n_1 \equiv n_2 \Pmod{d}$.
  Since there are exactly $d$ congruence
  classes for $n$, there are exactly $d$
  congruence classes of solutions as well,
  which completes the proof.
\end{proof}

  \chapter{Sept.~10 --- Chinese Remainder Theorem}

\section{More on Linear Congruences}

\begin{corollary}
  Consider the linear congruence
  $ax \equiv b \Pmod{m}$ and let
  $d = (a, m)$. If $d \mid b$, then
  there are exactly $d$ incongruent
  solutions modulo $m$, given by
  \[
    x = x_0 + \frac{m}{d} \cdot n,
    \quad n = 0, 1, \dots, d - 1
  \]
  where $x_0$ is any particular solution.
\end{corollary}

\begin{example}
  We solve $16 x \equiv 8 \Pmod{28}$. We
  compute
  $d = (16, 28)$ by the Euclidean algorithm:
  \begin{align*}
    28 &= 1 \cdot 16 + 12 \\
    16 &= 1 \cdot 12 + 4 \\
    12 &= 3 \cdot 4 + 0.
  \end{align*}
  So $d = 4$. Since $4 \mid 8$, the
  congruence has $4$ incongruent solutions.
  Working backwards, we have
  \[
    4 = 2 \cdot 16 + (-1) \cdot 28.
  \]
  Multiplying by $2$, we get that
  $8 = 4 \cdot 16 + (-2) \cdot 28$.
  Taking this equation modulo $28$, we get
  \[
    16 \cdot 4 \equiv 8 \Pmod{28},
  \]
  so $x_0 = 4$ is a particular solution.
  Thus all the incongruent solutions
  are given by
  $x = 4 + (28 / 4) n$ for
  $n = 0, 1, 2, 3$, that is
  $x = 4, 11, 18, 25$.
\end{example}

\begin{definition}
  Any solution of $ax \equiv 1 \Pmod{m}$
  is called the \emph{multiplicative inverse}
  of $a$ modulo $m$. The multiplicative
  inverse of $a$ is often denoted
  $\overline{a}$.
\end{definition}

\begin{corollary}
  The congruence $ax \equiv 1 \Pmod{m}$
  has a solution if and only if
  $(a, m) = 1$. In this case, the
  congruence has a unique solution.
  In particular, the multiplicative
  inverse, if it exists, is unique.
\end{corollary}

\section{The Chinese Remainder Theorem}

\begin{example}\label{ex:crt-example}
  Consider the following problem:
  Find a positive integer having
  remainder $2$ when divided by $3$,
  remainder $1$ when divided by $4$,
  and remainder $3$ when divided by $5$.
  The problem can be rephrased as
  asking for a solution to the system
  of congruences:
  \[
    \begin{cases}
      x \equiv 2 \Pmod{3} \\
      x \equiv 1 \Pmod{4} \\
      x \equiv 3 \Pmod{5}.
    \end{cases}
  \]
\end{example}

\begin{theorem}[Chinese remainder theorem]
  Let $m_1, \dots, m_n$ be pairwise
  relatively prime positive integers,
  and let $b_1, \dots, b_n \in \Z$.
  Then the system of congruences
  \[
    \begin{cases}
      x \equiv b_1 \Pmod{m_1} \\
      x \equiv b_2 \Pmod{m_2} \\
      \quad \vdots \\
      x \equiv b_n \Pmod{m_n}
    \end{cases}
  \]
  has a unique solution modulo
  $M = m_1 \cdots m_n$.
\end{theorem}

\begin{proof}
  Let $M = m_1 \cdots m_n$ and
  $M_i = M / m_i$. Then
  $(M_i, m_i) = 1$, so there are solutions
  to each system
  $M_i x_i \equiv 1 \Pmod{m_i}$ given
  by $x_i = \overline{M}_i$. Consider
  \[
    x = b_1 M_1 \overline{M}_1
    + b_2 M_2 \overline{M}_2
    + \cdots
    + b_n M_n \overline{M}_n.
  \]
  Note that $m_i \mid M_j$ for
  $i \ne j$, so
  $x \equiv b_i M_i \overline{M}_i \equiv b_i \Pmod{m_i}$,
  so $x$ is a solution to the system.

  For uniqueness modulo $M$, let
  $x'$ be another solution. Then
  $x' \equiv b_i \Pmod{m_i}$ for
  each $1 \le i \le n$. Then
  \[
    x \equiv x' \Pmod{m_i}, \quad
    1 \le i \le n.
  \]
  Thus $m_i \mid x - x'$, so
  $M \mid x - x'$ since the $m_i$
  are pairwise relatively prime,
  so $x \equiv x' \Pmod{M}$.
\end{proof}

\begin{example}
  We now solve Example
  \ref{ex:crt-example}. Using the
  notation in the proof, we have
  \[
    (m_1, m_2, m_3) = (3, 4, 5), \quad
    (b_1, b_2, b_3) = (2, 1, 3), \quad
    M = 60,
    \quad
    (M_1, M_2, M_3) = (20, 15, 12).
  \]
  We still need to compute
  $\overline{M}_i$. In general, this
  can be done via the Euclidean algorithm.
  In this case,
  \[
    (\overline{M}_1, \overline{M}_2, \overline{M}_3)
    = (2, 3, 3).
  \]
  Now we can calculate the solution
  using
  \[
    x = b_1 M_1 \overline{M}_1
    + b_2 M_2 \overline{M}_2
    + b_3 M_3 \overline{M}_3
    = (2 \cdot 20 \cdot 2)
    + (1 \cdot 15 \cdot 3)
    + (3 \cdot 12 \cdot 3)
    = 233.
  \]
  Reducing modulo $60$, we get that the
  unique solution is given by
  $x \equiv 53 \Pmod{60}$.
\end{example}

\section{Wilson's Theorem}

\begin{lemma}\label{lem:self-inverse}
  Let $p$ be a prime and let
  $a \in \Z$. Then
  $a$ is its own inverse modulo $p$
  (i.e., $a \equiv \overline{a} \Pmod{p}$)
  if and only if $a \equiv \pm 1 \Pmod{p}$.
\end{lemma}

\begin{proof}
  $(\Rightarrow)$ Suppose
  $a \equiv \overline{a} \Pmod{p}$.
  Then $a^2 \equiv a \overline{a} \equiv 1 \Pmod{p}$, so
  $p \mid a^2 - 1 = (a - 1)(a + 1)$.
  Since $p$ is prime, we have
  $p \mid a - 1$ or $p \mid a + 1$, so
  $a \equiv \pm 1 \Pmod{p}$.

  $(\Leftarrow)$ This is obvious since
  $(\pm 1)^2 = 1$ in $\Z$, so they
  are also equal after reducing modulo $p$.
\end{proof}

\begin{theorem}[Wilson's theorem]\label{thm:wilson}
  Let $p$ be a prime. Then
  $(p - 1)! \equiv -1 \Pmod{p}$.
\end{theorem}

\begin{example}
  The idea behind the proof is the
  following: Concretely, if $p = 11$, we
  have
  \[
    (11 - 1)! =
    10 \cdot 9 \cdot 8 \cdot 7 \cdot 6 \cdot
    5 \cdot 4 \cdot 3 \cdot 2 \cdot 1
    \pmod{11}
  \]
  By Lemma \ref{lem:self-inverse},
  $10$ and $1$ are their own inverses
  modulo $11$.
  For each other integer $2 \le n \le 9$,
  we can pair them with their inverses:
  $(2, 6)$, $(3, 4)$, $(5, 9)$, $(7, 8)$.
  Then we can write
  \[
    (11 - 1)!
    \equiv (9 \cdot 5) \cdot (8 \cdot 7)
    \cdot (6 \cdot 2) \cdot (4 \cdot 3)
    \cdot 10 \cdot 1
    \equiv 10 \cdot 1 \equiv -1 \pmod{11}.
  \]
\end{example}

\begin{proof}[Proof of Theorem \ref{thm:wilson}]
  We can easily check the theorem
  for $p = 2, 3$, so suppose
  $p > 3$ is a prime. Then each $a$
  with $1 \le a \le p - 1$ has a unique
  inverse modulo $p$, and this inverse
  is distinct from $a$ if
  $2 \le a \le p - 2$. Pair each such
  integer with its inverse modulo $p$,
  say $a$ and $a'$. The product of
  all of these pairs is
  $(p - 2)!$, so
  $(p - 2)! \equiv 1 \Pmod{p}$.
  Thus $(p - 1)! \equiv p - 1 \equiv -1 \Pmod{p}$.
\end{proof}

\begin{prop}[Converse of Wilson's theorem]
  Let $n \in \Z$ with $n > 1$. If
  $(n - 1)! \equiv -1 \Pmod{n}$, then
  $n$ is prime.
\end{prop}

\begin{proof}
  Suppose $n = ab$ with $1 \le a < n$.
  It suffices to show that $a = 1$.
  Since $a < n$, we have
  $a \mid (n - 1)!$. Also,
  $n \mid (n - 1)! + 1$ by assumption,
  so $a \mid (n - 1)! + 1$ also since
  $a \mid n$. Thus
  \[
    a \mid ((n - 1)! + 1) - (n - 1)! = 1,
  \]
  so we must have $a = 1$.
\end{proof}

\begin{definition}
  A prime $p$ is a \emph{Wilson prime}
  if $(p - 1)! \equiv -1 \Pmod{p^2}$.
\end{definition}

\begin{example}
  The first few Wilson primes are
  $5, 13, 563$.
  In fact, these are the only known
  ones.
\end{example}

  \chapter{Sept.~15 --- Fermat's Little Theorem}

\begin{quote}
  \emph{What do you call it when you have
  your grandmother on speed dial?
  It's an insta gram.}
\end{quote}

\section{Fermat's Little Theorem}

\begin{theorem}[Fermat's little theorem]
  Let $p$ be a prime and $a \in \Z$
  such that $p \nmid a$. Then
  \[
    a^{p - 1} \equiv 1 \pmod{p}.
  \]
\end{theorem}

\begin{proof}
  Consider the $p - 1$ integers
  $a, 2a, 3a, \dots, (p - 1)a$. Note that
  $p \nmid a_i$ for any $1 \le i \le p - 1$.
  Note also that no two of these integers
  are congruent modulo $p$: If $a i \equiv a j \Pmod{p}$
  for some $i \ne j$, then we
  can multiply by the inverse $\overline{a}$
  of $a$ (which exists since $p \nmid a$)
  to get $i \equiv j \Pmod{p}$, which
  is impossible.
  Thus $\{a, 2a \dots, (p - 1)a\}$ is a
  complete nonzero residue system, so
  \[
    a(2a)(3a) \cdots ((p - 1)a)
    \equiv 1 \cdot 2 \cdot 3 \cdots (p - 1)
    \pmod{p}.
  \]
  Then $a^{p - 1}(p - 1)! \equiv (p - 1)! \Pmod{p}$, so
  $a^{p - 1} \equiv 1 \Pmod{p}$
  since $p \nmid (p - 1)!$.
\end{proof}

\begin{corollary}
  Let $p$ be prime and $a \in \Z$ with
  $p \nmid a$. Then
  $a^{p - 2}$ is the inverse of $a$
  modulo $p$.
\end{corollary}

\begin{proof}
  By Fermat's little theorem,
  $a \cdot a^{p - 2} = a^{p - 1} \equiv 1 \Pmod{p}$.
\end{proof}

\begin{corollary}
  Let $p$ be prime and $a \in \Z$.
  Then $a^p \equiv a \Pmod{p}$.
\end{corollary}

\begin{proof}
  If $p \mid a$, then both
  sides are congruent to
  $0$ modulo $p$. Otherwise, if
  $p \nmid a$, then we can write
  $a^p = a \cdot a^{p - 1} \equiv a \cdot 1 = a \Pmod{p}$
  by Fermat's little theorem.
\end{proof}

\begin{corollary}
  Let $p$ be prime. Then
  $2^p \equiv 2 \Pmod{p}$.
\end{corollary}

\begin{definition}
  If $n \in \Z$ is composite and
  $2^n \equiv 2 \Pmod{n}$,
  then $n$ is called a \emph{pseudoprime}.
\end{definition}

\begin{remark}
  It is known that there are infinitely
  many (even and odd) pseudoprimes.
\end{remark}

\begin{example}
  Consider $n = 341 = 11 \cdot 31$.
  To prove that $2^{341} \equiv 2 \Pmod{341}$,
  it suffices to show that
  $2^{341} \equiv 2 \Pmod{11}$
  and $2^{341} \equiv 2 \Pmod{31}$
  by the Chinese remainder theorem.
  Note that
  \begin{align*}
    2^{341}
    = (2^{10})^{34} \cdot 2 \equiv
    1^{34} \cdot 2 &= 2 \pmod{11} \\
    2^{341} = (2^{30})^{11} \cdot 2^{11}
    \equiv 1^{11} \cdot (2^5)^2 \cdot 2
    \equiv 1^2 \cdot 2 &= 2 \pmod{31}
  \end{align*}
  by Fermat's little theorem, so
  $341$ is a pseudoprime.
\end{example}

\section{Euler's Theorem}

\begin{definition}
  Let $n \in \Z$, $n > 0$.
  \emph{Euler's phi function}, denoted
  $\varphi(n)$, is the number of
  positive integers $\le n$ that are
  relatively prime to $n$. In other words,
  \[
    \varphi(n)
    = \#\{m \in \Z : 1 \le m \le n, (m, n) = 1\}.
  \]
\end{definition}

\begin{example}
  We have $\varphi(4) = 2$,
  $\varphi(14) = 6$, and
  $\varphi(p) = p - 1$ for any prime $p$.
\end{example}

\begin{theorem}[Euler's theorem]
  Let $a, m \in \Z$ with $m > 0$. If
  $(a, m) = 1$, then
  \[
    a^{\varphi(m)} \equiv 1 \pmod{m}.
  \]
\end{theorem}

\begin{proof}
  Let $r_1, r_2, \dots, r_{\varphi(m)}$
  be the distinct positive integers
  not exceeding $m$ such that
  $(r_i, m) = 1$. Then consider the
  integers $a r_1, a r_2, \dots, a r_{\varphi(m)}$.
  Note first that
  $(a r_i, m) = 1$ since
  $(r_i, m) = 1$ and $(a, m) = 1$
  by assumption. Note also that
  $a r_i \not\equiv a r_j \Pmod{m}$
  for $i \ne j$ since
  $\overline{a}$ exists (since
  $(a, m) = 1$), and multiplying
  by $\overline{a}$ implies
  $r_i \equiv r_j \Pmod{m}$, which is
  impossible. Thus the least
  nonnegative residues of
  $\{a r_1, \dots, a r_{\varphi(m)}\}$
  coincide with
  $\{r_1, \dots, r_{\varphi(m)}\}$, so
  we have
  \[
    (a r_1)(a r_2) \cdots (a r_{\varphi(m)})
    = r_1 r_2 \cdots r_{\varphi(m)} \pmod{m},
  \]
  thus $a^{\varphi(m)} (r_1 \cdots r_{\varphi(m)}) \equiv (r_1 \cdots r_{\varphi(m)}) \Pmod{m}$.
  Since $(r_1 \cdots r_{\varphi(m)}, m) = 1$,
  the inverse of $r_1 \cdots r_{\varphi(m)}$
  modulo $m$ exists, and
  multiplying by the inverse
  gives $a^{\varphi(m)} \equiv 1 \Pmod{m}$.
\end{proof}

\begin{remark}
  Taking $m = p$ recovers Fermat's little theorem
  since $\varphi(p) = p - 1$ for prime $p$.
\end{remark}

\begin{definition}
  Let $m$ be a positive integer.
  A set of $\varphi(m)$ integers
  such that each integer is relatively
  prime to $m$ and no two are
  congruent modulo $m$ is called
  a \emph{reduced residue system}
  modulo $m$.
\end{definition}

\begin{example}
  $\{1, 5, 7, 11\}$ is a reduced
  residue system modulo $12$.
  So is
  \[
    \{5(1), 5(5), 5(7), 5(11)\}
    = \{5, 25, 35, 55\}.
  \]
  For a prime $p$, the set
  $\{1, 2, \dots, p - 1\}$ is always a
  reduced residue system modulo $p$.
\end{example}

\begin{corollary}
  Let $a, m \in \Z$ with $m > 0$
  and $(a, m) = 1$. Then
  $\overline{a} \equiv a^{\varphi(m) - 1} \Pmod{m}$.
\end{corollary}

\section{Arithmetic Functions and Multiplicativity}

\begin{definition}
  An \emph{arithmetic function} is a
  function whose domain is the set of
  positive integers.
\end{definition}

\begin{example}\label{ex:arith-funcs}
  The following are examples of
  arithmetic functions:
  \begin{enumerate}
    \item Euler's $\varphi$ function;
    \item $\tau(n)$, the number of
      positive divisors of $n$;
    \item $\sigma(n)$, the sum of
      the positive divisors of $n$;
    \item $\omega(n)$, the number of
      distinct prime factors of $n$;
    \item $p(n)$, the number of
      integer partitions of $n$;
    \item $\Omega(n)$, the number of
      total prime factors (counted with
      multiplicity) of $n$.
  \end{enumerate}
\end{example}

\begin{definition}
  An arithmetic function
  $f$ is \emph{multiplicative} if
  $f(mn) = f(m) f(n)$ whenever
  $(m, n) = 1$.
  We say that $f$ is \emph{completely multiplicative}
  if $f(mn) = f(m) f(n)$ for all
  $m, n$.
\end{definition}

\begin{remark}
  Note that if $n > 1$, then we can write
  $n = p_1^{a_1} \cdots p_r^{a_r}$.
  If $f$ is multiplicative, then
  \[
    f(n) = f(p_1^{a_1} \cdots p_r^{a_r})
    = f(p_1^{a_1}) \cdots f(p_r^{a_r}).
  \]
  So multiplicative functions are
  determined by their values at
  prime powers. If
  $f$ is completely multiplicative,
  then $f(n) = f(p_1)^{a_1} \cdots f(p_r)^{a_r}$
  and $f$ is determined by its values
  at primes.
\end{remark}

\begin{example}
  The functions $\varphi, v, \sigma$
  from Example \ref{ex:arith-funcs}
  are multiplicative, while
  $\omega, p, \Omega$ are not.
\end{example}

\begin{example}
  The functions $f(n) = 1$ and $f(n) = 0$
  are completely multiplicative. The
  function $f$ defined by
  $f(1) = 1$ and $f(n) > 0$ if $n > 1$
  is also completely multiplicative.
\end{example}

\begin{remark}
  If $f$ is multiplicative and not
  identically zero, then $f(1) = 1$.
  To see this, take $n$ such
  that $f(n) \ne 0$ (since $f$ is not
  identically zero). Then
  $f(n) = f(n \cdot 1) = f(n) f(1)$, so
  $f(1) = 1$ since $f(n) \ne 0$.
\end{remark}

  \chapter{Sept.~17 --- Arithmetic Functions}

\section{Properties of Multiplicative Functions}

\begin{remark}
  We write $\sum_{d \mid n} f(d)$
  to denote a sum over the positive
  divisors of $n$. For instance,
  \[
    \sum_{d \mid 12} f(d)
    = f(1) + f(2) + f(3) + f(4) + f(6) + f(12).
  \]
\end{remark}

\begin{theorem}
  Let $f$ be an arithmetic function, and
  for $n \in \Z$, $n > 0$, define
  \[
    F(n) = \sum_{d \mid n} f(d).
  \]
  If $f$ is multiplicative, then so is
  $F$.
\end{theorem}

\begin{proof}
  Let $m, n$ be relative prime. We need
  to show that $F(mn) = F(m)F(n)$. We have
  \[
    F(mn) = \sum_{d \mid mn} f(d).
  \]
  We claim that every divisor $d$ of
  $mn$ can be written uniquely as
  $d = d_1 d_2$, where $d_1 \mid m$ and
  $d_2 \mid m$. Moreover, any such
  product $d_1 d_2$ is a divisor of $mn$.
  To see this, write
  $m = p_1^{a_1} \dots p_r^{a_r}$ and
  $n = q_1^{b_1} \dots q_s^{b_s}$, where
  all the $p_1, \dots, p_r, q_1, \dots, q_s$
  are distinct. Then if $d \mid mn$, then
  \[
    d = p_1^{e_1} \dots p_r^{e_r} q_1^{f_1} \dots q_s^{f_s},
    \quad 0 \le e_i \le q_i, 0 \le f_j \le b_j.
  \]
  Then we must choose $d_1 = p_1^{e_1} \dots p_r^{e_r}$
  and $d_2 = q_1^{f_1} \dots q_s^{f_s}$,
  which proves the claim.

  Using the claim, we can split the sum
  into
  \[
    F(mn) = \sum_{d \mid mn} f(d)
    = \sum_{d_1 \mid m} \sum_{d_2 \mid n} f(d_1 d_2)
    = \sum_{d_1 \mid m} \sum_{d_2 \mid n} f(d_1) f(d_2)
    = F(m) F(n),
  \]
  where we note that $(d_1, d_2) = 1$
  since $(m, n) = 1$.
\end{proof}

\begin{example}
  Let $m = 4$, $n = 3$. Then we can write
  \begin{align*}
    F(3 \cdot 4)
    = \sum_{d \mid 12}
    f(d)
    &= f(1) + f(2) + f(3) + f(4) + f(6) + f(12) \\
    &= f(1 \cdot 1) + f(1 \cdot 2) + f(3 \cdot 1) + f(1 \cdot 4) + f(3 \cdot 2) + f(3 \cdot 4) \\
    &= f(1) f(1) + f(1) f(2) + f(3) f(1) + f(1) f(4) + f(3) f(2) + f(3) f(4) \\
    &= (f(1) + f(3))(f(1) + f(2) + f(4))
    = F(3) F(4).
  \end{align*}
\end{example}

\section{Properties of the Euler Phi Function}

\begin{theorem}
  The Euler $\varphi$ function is
  multiplicative.
\end{theorem}

\begin{proof}
  Let $m, n \in \Z$, $m, n > 0$ with
  $(m, n) = 1$. We need to show that
  $\varphi(mn) = \varphi(m) \varphi(n)$.
  Consider the array of positive integers
  $\le mn$ organized as follows:
  \[
  \begin{matrix}
    1 & m + 1 & 2m + 1 & \cdots & (n - 1)m + 1 \\
    2 & m + 2 & 2m + 2 & \cdots & (n - 1)m + 2 \\
    \vdots & \vdots & \vdots & \ddots & \vdots \\
    i & m + i & 2m + i & \cdots & (n - 1)m + i \\
    \vdots & \vdots & \vdots & \ddots & \vdots \\
    m & 2m & 3m & \cdots & nm
  \end{matrix}
  \]
  Consider the $i$th row. If $(i, m) > 1$,
  then no element on the $i$th row is
  relatively prime to $m$ (and hence
  cannot be relatively prime to $mn$). Thus
  we may restrict our attention to those
  $i$ that satisfy $(i, m) = 1$. There
  are, by definition, $\varphi(m)$
  such values of $i$. The entries in the
  $i$th row are
  \[
    i, \quad m + i, \quad 2m + i, \quad \dots, \quad (n - 1)m + i.
  \]
  We claim that this is a complete
  residue system modulo $n$. To see this,
  suppose that
  \[
    km + i \equiv jm + i \pmod{n},
    \quad 0 \le k, j \le n - 1.
  \]
  Then $km \equiv jm \Pmod{n}$. Since
  $(m, n) = 1$, this implies
  $k \equiv j \Pmod{n}$. Since
  $0 \le k, j \le n - 1$, we must have
  $k = j$. The claim follows since
  we have $n$ non-congruent elements
  (modulo $n$)
  in the list. Thus, there are
  $\varphi(n)$ elements in the $i$th row
  that are relatively prime to $n$. Also,
  $(km + i, m) = (i, m) = 1$
  by the Euclidean algorithm, so
  they are relatively prime to $m$ as well. Thus
  $\varphi(mn) = \varphi(m) \varphi(n)$.
\end{proof}

\begin{theorem}
  Let $p$ be prime, $a \in \Z$, $a > 0$.
  Then $\varphi(p^a) = p^a - p^{a - 1}$.
\end{theorem}

\begin{proof}
  The total number of integers
  not exceeding $p^a$ is $p^a$. The only
  integers not relatively prime to
  $p^a$ are the multiples of $p$:
  $p, 2p, 3p, \dots, (p^{a - 1}) p$.
  There are $p^{a - 1}$ such integers, so
  $\varphi(p^a) = p^a - p^{a - 1}$.
\end{proof}

\begin{theorem}\label{thm:phi-product}
  Let $n \in \Z$, $n > 0$. Then
  \[
    \varphi(n) = n \prod_{p \mid n} \left(1 - \frac{1}{p}\right).
  \]
\end{theorem}

\begin{proof}
  Write $n = p_1^{a_1} \dots p_r^{a_r}$.
  Then
  \begin{align*}
    \varphi(n)
    = \varphi(p_1^{a_1} \cdots p_r^{a_r})
    &= \varphi(p_1^{a_1}) \cdots \varphi(p_r^{a_r})
    = (p_1^{a_1} - p_1^{a_1 - 1}) \cdots (p_r^{a_r} - p_r^{a_r - 1}) \\
    &= p_1^{a_1} \cdots p_r^{a_r} \left(1 - \frac{1}{p_1}\right) \cdots \left(1 - \frac{1}{p_r}\right)
    = n \prod_{p \mid n} \left(1 - \frac{1}{p}\right).
  \end{align*}
  This proves the desired formula.
\end{proof}

\begin{remark}
  One can interpret
  Theorem \ref{thm:phi-product}
  probabilistically: It says that
  $\varphi(n)$ is $n$ times the ``probability''
  that an integer is not divisible by any
  of the primes dividing $n$.
\end{remark}

\begin{example}
  Consider $n = 504 = 2^3 \cdot 3^2 \cdot 7$.
  Then $\varphi(n)$ is given by
  \[
    \varphi(504)
    = 504 \left(1 - \frac{1}{2}\right)
    \left(1 - \frac{1}{3}\right)
    \left(1 - \frac{1}{7}\right)
    = 504 \cdot \frac{1}{2} \cdot \frac{2}{3} \cdot \frac{6}{7} = 144.
  \]
\end{example}

\begin{theorem}[Gauss]
  Let $n \in \Z$, $n > 0$. Then
  \[
    \sum_{d \mid n} \varphi(d) = n.
  \]
\end{theorem}

\begin{proof}
  Let $d$ be a divisor of $n$. Define
  the set
  \[
    S_d = \{1 \le m \le n : (m, n) = d\}.
  \]
  Note that $(m, n) = d$ if and only if
  $(m / d, n / d) = 1$. Thus
  $|S_d| = \varphi(n / d)$. Note also
  that every integer less than or equal to
  $n$ belongs to exactly one of the
  $S_d$, so
  \[
    n = \sum_{d \mid n} |S_d| = \sum_{d \mid n} \varphi(n / d)
    = \sum_{d \mid n} \varphi(d),
  \]
  which the last equality follows since
  $\{d : d \mid n\} = \{n / d : d \mid n\}$.
\end{proof}

\begin{example}
  Let $n = 12$. We verify that
  $12 = \sum_{d \mid 12} \varphi(d)$.
  Write the table
  \begin{center}
  \begin{tabular}{c|c}
    $d$ & $S_d$ \\
    \hline
    $1$ & $\{1, 5, 7, 11\}$ \\
    $2$ & $\{2, 10\}$ \\
    $3$ & $\{3, 9\}$ \\
    $4$ & $\{4, 8\}$ \\
    $6$ & $\{6\}$ \\
    $12$ & $\{12\}$
  \end{tabular}
  \end{center}
  Summing the $|S_d| = \varphi(12 / d)$, we
  indeed get
  $12 = 4 + 2 + 2 + 2 + 1 + 1$.
\end{example}

  \chapter{Sept.~22 --- Exam 1 Review}

\begin{quote}
  \emph{Why are Saturday and Sunday the
  strongest days? The other
  are week days.}
\end{quote}

\section{Practice Problems}

\begin{exercise}
  Show that $\varphi$ is multiplicative
  but not completely multiplicative.
\end{exercise}

\begin{proof}
  The idea for the first part is to
  draw an $m \times n$ table of the
  first $mn$ integers,
  see the proof of Theorem
  \ref{thm:phi-multiplicative} for
  the details.
  For the second part,
  note that
  $\varphi(2) = 1$, $\varphi(2)^2 = 1$,
  but $\varphi(4) = 2$.
\end{proof}

\begin{proof}[Alternative proof]
  Let $R_k$ denote the set of residue
  classes modulo $k$ that are coprime
  to $k$.
  Note that $|R_k| = \varphi(k)$, so
  it suffices to show there is a
  bijection $\psi : R_{mn} \to R_m \times R_n$
  for $(m, n) = 1$.
  Define
  \[\psi(a) = (a \Mod{m}, a \Mod{n}).\]
  To see that $\psi$ is surjective,
  let $(b, c) \in R_m \times R_n$. Since
  $(m, n) = 1$, by
  the Chinese remainder theorem there
  exists $a \in \Z$, defined modulo
  $mn$, such that
  $a \equiv b \Pmod{m}$ and
  $a \equiv c \Pmod{n}$.
  So $\psi(a) = (b, c)$. Note that
  $(a, mn) = 1$ since
  $(a, m) = (b, m) = 1$
  and $(a, n) = (c, n) = 1$, so
  $a \in R_{mn}$.
  Injectivity follows
  since the choice of
  $a$ is unique modulo $mn$ by
  the Chinese remainder theorem.
\end{proof}

\begin{exercise}
  Compute $(163, 67)$ by the Euclidean
  algorithm.
\end{exercise}

\begin{proof}
  We compute that
  \begin{align*}
  (163, 67)
  &= (163 - 134, 67) = (29, 67) \\
  &= (29, 67 - 2 \cdot 29) = (29, 9) \\
  &= (29 - 3 \cdot 9, 9) = (2, 9) \\
  &= (2, 9 - 4 \cdot 2) = (2, 1),
  \end{align*}
  so we have $(163, 67) = 1$.
\end{proof}

\begin{exercise}
  State and prove Wilson's theorem.
\end{exercise}

\begin{proof}
  Wilson's theorem states
  that $(p - 1)! \equiv -1 \Pmod{p}$
  for prime $p$ (the converse also holds).
  The idea behind the
  proof is to note that each residue
  modulo $p$
  other than $\pm 1$ can be paired with
  its (distinct) additive
  inverse modulo $p$.
  For the details, see the proof of
  Theorem \ref{thm:wilson}.
\end{proof}

\begin{exercise}
  Find the least positive solution
  $x$ to the congruence
  $x \equiv 20^{110} \Pmod{17}$.
\end{exercise}

\begin{proof}
  Use Fermat's little theorem: The division
  algorithm gives
  $110 = 6 \cdot 16 + 14$, so
  \begin{align*}
    x
    &\equiv 20^{6 \cdot 16 + 14}
    \equiv (20^{16})^6 \cdot 20^{14}
    \pmod{17} \\
    &= 1^6 \cdot 20^{14}
    \equiv 20^{14}
    = 3^{14} \pmod{17}.
  \end{align*}
  Multiplying both sides by $3^2$ gives
  $9x \equiv 3^2 x \equiv 3^{16} \equiv 1 \Pmod{17}$, so
  it suffices to find the inverse
  of $9$ modulo $17$. Using the
  Euclidean algorithm, we have
  \[
    (17, 9)
    = (17 - 9, 9) = (8, 9)
    = (8, 9 - 8) = (8, 1)
    = 1,
  \]
  so $1 = 9 - 8 = 9 - (17 - 9) = 2 \cdot 9 - 17$.
  Thus $\overline{9} \equiv 2 \Pmod{17}$,
  so we can take $x = 2$.
\end{proof}

\begin{exercise}
  Find the least positive solution
  $x$ to the congruence
  $x \equiv 38^{110} \Pmod{21}$.
\end{exercise}

\begin{proof}
  First we compute that
  $\varphi(21) = \varphi(7) \varphi(3) = 6 \cdot 2 = 12$.
  By Euler's theorem,
  \begin{align*}
    x \equiv 38^{110}
    \equiv 17^{110}
    \equiv 17{9 \cdot 12 + 2}
    \equiv (17^{12})^9 \cdot 17^2
    \equiv 17^2 \pmod{21}.
  \end{align*}
  Now we notice that
  $17^2 \equiv (-4)^2 \equiv 16 \Pmod{21}$, so
  we can take $x = 16$.
\end{proof}

\begin{exercise}
  Let $a, m \in \Z$ and $m > 1$.
  If $(a, m) = 1$, show that
  $a^{\varphi(m) - 1}$ is the multiplicative
  inverse of $a$ modulo $m$.
\end{exercise}

\begin{proof}
  By Euler's theorem,
  $a \cdot a^{\varphi(m) - 1} = a^{\varphi(m)} \equiv 1 \Pmod{m}$,
  so $\overline{a} \equiv a^{\varphi(m) - 1} \Pmod{m}$.
\end{proof}

\begin{exercise}
  Prove that for odd primes $p$, we
  have $2(p - 3)! \equiv -1 \Pmod{p}$.
\end{exercise}

\begin{proof}
  By Wilson's theorem, 
  $(p - 1)! \equiv -1 \Pmod{p}$. Then we
  have
  \[
    (p - 3)! (p - 2)(p - 1)
    \equiv -1 \Pmod{p},
  \]
  so $2(p - 3)! \equiv -1 \Pmod{p}$ since
  $(p - 2)(p - 1) \equiv 2 \Pmod{p}$.
\end{proof}

\begin{exercise}
  Find integers $a, b$ such that
  $(a, b) = 3$ and $a + b = 66$.
\end{exercise}

\begin{proof}
  It suffices to write
  $a = 3a_1$, $b = 3b_1$, where
  $(a_1, b_1) = 1$. One way to do this
  is $(a_1, b_1) = (1, 21)$:
  \[
    3(a_1 + b_1) = 3(1 + 21)
    = 3 \cdot 22 = 66.
  \]
  Thus we may take
  $a = 3$, $b = 63$.
\end{proof}

\begin{remark}
  Recall that a \emph{reduced} 
  residue system modulo $m$
  is a set $\{r_1, \dots, r_{\varphi(m)}\}$
  of integers coprime to $m$ and
  pairwise incongruent modulo $m$.
  Note that the $r_i$ themselves need
  not be coprime, in fact they may
  share arbitrarily large common factors:
  Take any $r_\ell \ne 1$ and consider
  \[
    \{r_\ell r_1, \dots, r_{\ell} r_{\varphi(m)}\}.
  \]
  By repeating this, we can get
  arbitrarily large powers of $r_\ell$
  as a common factor.
\end{remark}

  \chapter{Sept.~29 --- Arithmetic Functions, Part 2}

\section{The Divisor Function}

\begin{definition}
  Let $n \in \Z$. The \emph{number of
  positive divisors} of $n$, denoted
  $\tau(n)$, is defined by
  \[
    \tau(n) =
    \#\{d \in \Z : d > 0, d \mid n\}.
  \]
\end{definition}

\begin{theorem}
  $\tau(n)$ is multiplicative.
\end{theorem}

\begin{proof}
  Observe that
  $\tau(n) = \sum_{d \mid n} 1$ and
  $1$ is multiplicative, so
  the result follows from
  Theorem \ref{thm:sum-mult}.
\end{proof}

\begin{remark}
  Since $\tau(n)$ is multiplicative,
  it is determined by its behavior on
  prime powers.
\end{remark}

\begin{theorem}\label{thm:tau-prime-power}
  Let $p$ be prime and let
  $a \in \Z$, $a > 0$. Then
  $\tau(p^a) = a + 1$.
\end{theorem}

\begin{proof}
  The divisors of $p^a$ are
  exactly the integers
  $1, p, p^2, \dots, p^a$. There
  are $a + 1$ of these.
\end{proof}

\begin{theorem}
  Let $n = p_1^{a_1} \cdots p_r^{a_r}$
  with $p_1, \dots, p_r$ distinct
  primes and $a_1, \dots, a_r$ positive
  integers. Then
  \[
    \tau(n) = \prod_{i = 1}^r (a_i + 1).
  \]
\end{theorem}

\begin{proof}
  This follows from
  $\tau$ being multiplicative and
  Theorem \ref{thm:tau-prime-power}.
\end{proof}

\begin{remark}
  For some interesting reading about $\tau(n)$,
  see \emph{Dirichlet's
  divisor problem}.
\end{remark}

\begin{example}
  Consider $504 = 2^3 \cdot 3^2 \cdot 7$.
  Then
  $\tau(504) = (3 + 1)(2 + 1)(1 + 1) = 24$.
\end{example}

\section{The Sum of Divisors Function}

\begin{definition}
  Let $n \in \Z$, $n > 0$. The
  \emph{sum of divisors function},
  denoted $\sigma(n)$, is defined by
  \[
    \sigma(n) = \sum_{d \mid n} d.
  \]
\end{definition}

\begin{theorem}
  $\sigma(n)$ is multiplicative.
\end{theorem}

\begin{proof}
  This follows from
  $f(d) = d$ being multiplicative
  and Theorem \ref{thm:sum-mult}.
\end{proof}

\begin{theorem}
  Let $p$ be a prime and $a \in \Z$,
  $a > 0$. Then
  $\sigma(p^a) = (p^{a + 1} - 1) / (p - 1)$.
\end{theorem}

\begin{proof}
  The divisors of $p^a$ are
  $1, p, p^2, \dots, p^a$, so
  \[
    \sigma(p^a)
    = 1 + p + p^2 + \cdots + p^a
    = \frac{p^{a + 1} - 1}{p - 1}
  \]
  by the formula for a (finite)
  geometric series.
\end{proof}

\begin{theorem}
  Let $n = p_1^{a_1} \cdots p_r^{a_r}$
  with $p_1, \dots, p_r$ distinct
  primes and $a_1, \dots, a_r$ positive
  integers. Then
  \[
    \sigma(n) = \prod_{i = 1}^r
    \frac{p_i^{a_i + 1} - 1}{p_i - 1}.
  \]
\end{theorem}

\begin{example}
  Consider $504 = 2^3 \cdot 3^2 \cdot 7$.
  Then
  \[
    \sigma(504)
    = \frac{2^{3 + 1} - 1}{2 - 1}
    \cdot \frac{3^{2 + 1} - 1}{3 - 1}
    \cdot \frac{7^{1 + 1} - 1}{7 - 1}
    = 15 \cdot 13 \cdot 8 = 1560.
  \]
\end{example}

\section{Perfect Numbers}

\begin{definition}
  Let $n \in \Z$, $n > 0$. Then
  $n$ is a \emph{perfect number} if
  $\sigma(n) = 2n$, or
  $\sigma(n) - n = n$.
\end{definition}

\begin{remark}
  Note that $\sigma(n) - n$ is the
  sum of \emph{proper} divisors of $n$, so
  perfect numbers are those that
  equal the sum of their proper
  divisors.
\end{remark}

\begin{example}
  $6$ and $28$ are perfect numbers.
\end{example}

\begin{conjecture}
  There are infinitely many perfect
  numbers.
\end{conjecture}

\begin{conjecture}
  All perfect numbers are even.
\end{conjecture}

\begin{theorem}\label{thm:even-perfect}
  Let $n \in \Z$, $n > 0$.
  Then $n$ is an even perfect number
  if and only if
  \[
    n = 2^{p - 1}(2^p - 1)
  \]
  for some prime $p$, and $2^p - 1$
  is prime (i.e. $2^p - 1$ is
  a Mersenne prime).
\end{theorem}

\begin{proof}
  $(\Rightarrow \text{Euler})$
  Assume $n$ is an even perfect number.
  Then we can write $n = 2^a b$ with $a, b \in \Z$,
  $a \ge 1$, and $b$ odd. Then we have
  \[
    \sigma(2^a b)
    = \sigma(2^a) \sigma(b)
    = (2^{a + 1} - 1) \sigma(b).
  \]
  Also, since $n$ is perfect, we
  can also write
  $\sigma(2^a b) = 2 \cdot 2^a b = 2^{a + 1} b$,
  thus
  \[
    (2^{a + 1} - 1) \sigma(b)
    = 2^{a + 1} b. \tag{$1$}
  \]
  This implies that $2^{a + 1} \mid (2^{a + 1} - 1) \sigma(b)$,
  so $2^{a + 1} \mid \sigma(b)$
  since $(2^{a + 1}, 2^{a + 1} - 1) = 1$.
  Then $\sigma(b) = 2^{a + 1} c$ (2) for
  some integer $c \ge 1$. Substituting
  this into $(1)$, we get
  \[
    (2^{a + 1} - 1) 2^{a + 1} c
    = 2^{a + 1} b,
  \]
  so we have $(2^{a + 1} - 1) c = b$
  (3).
  We now show that $c = 1$.
  Suppose to the contrary that $c > 1$.
  Then (3) implies that $b$ has at least
  $3$ distinct divisors, namely
  $1, b, c$. Then
  \[
    \sigma(b) \ge 1 + b + c.
  \]
  But (2) implies
  $\sigma(b) = 2^{a + 1} c = (2^{a + 1} - 1 + 1) c = (2^{a + 1} - 1)c + c = b + c$
  by (3), a contradiction.
  So $c = 1$, and by (3), $b = 2^{a + 1} - 1$.
  Also, (2) implies $\sigma(b) = b + 1$,
  so $b$ must be prime.
  One can show that $2^{a + 1} - 1$
  being prime implies $a + 1$ is prime,
  so $n = 2^a (2^{a + 1} - 1)$ with
  $2^{a + 1} - 1$ and $a + 1$ prime.

  $(\Leftarrow \text{Euclid})$
  Assume that $n = 2^{p - 1}(2^p - 1)$
  with $p$ and $2^p - 1$ both prime.
  Then
  \[
    \sigma(2^{p - 1}(2^p - 1))
    = \sigma(2^{p - 1}) \sigma(2^p - 1)
    = (2^p - 1)(2^p - 1 + 1)
    = (2^p - 1) 2^p
    = 2 \cdot 2^{p - 1}(2^p - 1),
  \]
  which shows that $\sigma(n) = 2n$.
  Thus $n$ is perfect.
\end{proof}

\begin{remark}
  Theorem \ref{thm:even-perfect} gives a characterization of
  even perfect numbers and a bijection
  between even perfect numbers and
  Mersenne primes.
\end{remark}

\begin{example}
  The first $5$ perfect numbers
  correspond to $p = 2, 3, 5, 7, 13$.
\end{example}

\section{The M\"obius Function}

\begin{definition}
  Let $n \in \Z$, $n > 0$. The
  \emph{M\"obius function}, denoted
  $\mu(n)$, is defined by
  \[
    \mu(n) =
    \begin{cases}
      1 & \text{if } n = 1, \\
      0 & \text{if } p^2 \mid n \text{ for some } p, \\
      (-1)^r & \text{if $n = p_1, \dots, p_r$ with $p_i$ distinct primes}.
    \end{cases}
  \]
\end{definition}

\begin{example}
  Since $504 = 2^3 \cdot 3^2 \cdot 7$,
  we have $\mu(504) = 0$. On the other
  hand,
  \[\mu(6) = (-1)^2 = 1 \quad \text{and}\quad
  \mu(30) = (-1)^3 = -1\]
\end{example}

\begin{theorem}
  $\mu(n)$ is multiplicative.
\end{theorem}

\begin{proof}
  Let $m, n$ be relatively prime
  positive integers. We need to show
  that $\mu(mn) = \mu(m) \mu(n)$.
  This is clear if $m = 1$ or $n = 1$,
  so we may assume $m, n > 1$.
  Note that $m$ or $n$ is divisible
  by a square if and only if $mn$
  is divisible by a square, since
  $(m, n) = 1$. In this case, both
  $\mu(m) \mu(n)$ and $\mu(mn)$ are $0$.

  Now suppose $m, n$ are products
  of distinct primes, say
  $m = p_1 \cdots p_r$ and
  $n = q_1 \cdots q_s$. Since
  $(m, n) = 1$, we have $p_i \ne q_j$
  for any $1 \le i \le r$ and
  $1 \le j \le s$. Thus
  \[
    \mu(mn)
    = \mu(p_1 \cdots p_r q_1 \cdots q_s)
    = (-1)^{r + s}
    = (-1)^r (-1)^s
    = \mu(p_1 \cdots p_r) \mu(q_1 \cdots q_s)
    = \mu(m) \mu(n),
  \]
  as desired. So $\mu$ is multiplicative.
\end{proof}

\begin{prop}\label{prop:mobius-sum}
  Let $n \in \Z$, $n > 0$. Then
  \[
    \sum_{d \mid n} \mu(d)
    =
    \begin{cases}
      1 & \text{if } n = 1, \\
      0 & \text{if } n > 1.
    \end{cases}
  \]
\end{prop}

\begin{proof}
  Since $\mu$ is multiplicative,
  so is $F(n) = \sum_{d \mid n} \mu(d)$
  by Theorem \ref{thm:sum-mult}. Thus
  it suffices to show that
  $F(p^a) = 0$ for prime powers $p^a$.
  We have
  \[
    F(p^a)
    = \sum_{d \mid p^a} \mu(d)
    = \mu(1) + \mu(p) + \mu(p^2) + \cdots + \mu(p^a).
  \]
  Note that $p^2 \mid p^j$ for $j \ge 2$,
  so $\mu(p^j) = 0$ for $j \ge 2$.
  Thus
  \[
    F(p^a) = \mu(1) + \mu(p) = 1 - 1 = 0.
  \]
  It is clear that
  $F(1) = \sum_{d \mid 1} \mu(d) = \mu(1) = 1$, so
  the result follows.
\end{proof}

\begin{example}
  Let $n = 12$. Then we have
  \begin{align*}
    \sum_{d \mid 12} \mu(d)
    &= \mu(1) + \mu(2) + \mu(3) + \mu(4) + \mu(6) + \mu(12) \\
    &= 1 - 1 - 1 + 0 + 1 + 0
    = 0.
  \end{align*}
\end{example}

  \chapter{Oct.~1 --- Quadratic Residues}

\begin{quote}
  \emph{What do you call a bear with no
  ear? A bee.}

  \emph{What do you call a magician
  who loses his magic? Ian.}
\end{quote}

\section{M\"obius Inversion}

\begin{theorem}[M\"obius inversion]
  Let $f, g$ be arithmetic functions.
  Then
  \[
    f(n) = \sum_{d \mid n} g(d)
    \quad \text{if and only if} \quad
    g(n) = \sum_{d \mid n} \mu(d) f(n / d)
    = \sum_{d \mid n} \mu(n / d) f(d).
  \]
\end{theorem}

\begin{proof}
  $(\Rightarrow)$ Assume
  $f(n) = \sum_{d \mid n} g(d)$.
  Then we can write
  \[
    \sum_{d \mid n} \mu(d) f(n / d)
    = \sum_{d \mid n} \mu(d)
    \sum_{a \mid (n / d)} g(a).
  \]
  Note that $a \mid (n / d)$ if and only
  if $d \mid (n / a)$, so we can
  switch the order of summation to get
  \[
    \sum_{d \mid n} \mu(d) f(n / d)
    = \sum_{a \mid n} g(a)
    \sum_{d \mid (n / a)} \mu(d)
    = \sum_{a \mid n} g(a)
    \begin{cases}
      1 & \text{if } n = a \\
      0 & \text{otherwise}
    \end{cases}
    = g(n),
  \]
  where the second equality is by
  by Proposition \ref{prop:mobius-sum}.
  This proves the forward direction.

  $(\Leftarrow)$ Assume
  that $g(n) = \sum_{d \mid n} \mu(n / d) f(d)$.
  Then we have
  \[
    \sum_{d \mid n} g(d)
    = \sum_{d \mid n} \sum_{a \mid d}
    \mu(d / a) f(a)
    = \sum_{a \mid n} f(a)
    \sum_{\substack{d \mid n \\ a \mid d}} \mu(d / a)
    = \sum_{a \mid n} f(a)
    \sum_{b \mid (n / a)} \mu(b)
  \]
  where we let
  $d = ab$ and noted that
  $ab \mid n$ if and only $ab \mid n$
  if and only if $b \mid (n / a)$.
  Then
  \[
    \sum_{d \mid n} g(d)
    = \sum_{a \mid n} f(a)
    \begin{cases}
      1 & \text{if } n = a \\
      0 & \text{otherwise}
    \end{cases}
    = f(n)
  \]
  by Proposition \ref{prop:mobius-sum},
  which proves the reverse direction.
\end{proof}

\begin{example}
  Recall that $\sum_{d \mid n} \varphi(d) = n$ by
  Theorem \ref{thm:euler-phi-sum}.
  By M\"obius inversion,
  \[
    \varphi(n)
    = \sum_{d \mid n} \mu(d) \frac{n}{d}
    = n \sum_{d \mid n} \frac{\mu(d)}{d}
    = n \prod_{p^a \mid n} \sum_{d \mid p^a} \frac{\mu(d)}{d}
    = n \prod_{p \mid n} \left(1 - \frac{1}{p}\right),
  \]
  where $\mu(d) / d$ is multiplicative
  since $\mu(d)$ and $1 / d$ both are.
  This recovers the product formula
  for $\varphi$.
\end{example}

\begin{example}
  We have $\tau(n) = \sum_{d \mid n} 1$.
  So by M\"obius inversion,
  \[
    1 = \sum_{d \mid n} \tau(n / d) \mu(d).
  \]
\end{example}

\begin{example}
  We have $\sigma(n) = \sum_{d \mid n} d$.
  So by M\"obius inversion,
  \[
    n = \sum_{d \mid n} \mu(d) \sigma(n / d).
  \]
\end{example}

\section{Quadratic Residues}

\begin{remark}
  So far, we have only studied
  linear congruences, which take
  the form $ax \equiv b \Pmod{m}$.
  Now we will be interested in
  \emph{quadratic} congruences, i.e.
  congruences of the form
  $ax^2 + bx \equiv c \Pmod{m}$.
  We will primarily restrict to the
  case $x^2 \equiv a \Pmod{p}$
  for $p$ an odd prime (the question
  is easy for $p = 2$).
\end{remark}

\begin{definition}
  Let $a, m \in \Z$, $m > 0$, and
  $(a, m) = 1$. Then $a$ is
  a \emph{quadratic residue modulo $m$}
  if the congruence
  $x^2 \equiv a \Pmod{m}$ has a
  solution. Otherwise, $a$ is a
  \emph{quadratic non-residue modulo $m$}.
\end{definition}

\begin{example}
  The quadratic residues modulo $11$
  are
  \[
    \{
      1^2, 2^2, 3^2, 4^2, 5^2, 6^2, 7^2, 8^2, 9^2, 10^2
    \}
    \equiv \{1, 4, 9, 5, 3, 3, 5, 9, 4, 1\}
    \equiv \{1, 3, 4, 5, 9\} 
    \pmod{11}.
  \]
  The quadratic non-residues are
  $\{2, 6, 7, 8, 10\}$. Note that
  the sizes of these sets are the same.
\end{example}

\begin{prop}
  Let $p$ be an odd prime and
  $a \in \Z$, $p \nmid a$.
  Then $x^2 \equiv a \Pmod{p}$ has
  either $0$ or $2$ incongruent
  solutions modulo $p$.
\end{prop}

\begin{proof}
  Assume $x^2 \equiv a \Pmod{p}$ has
  a solution $x_0$. Then
  $-x_0$ is also a solution. It
  is also incongruent to $p$, since
  if $x_0 \equiv -x_0 \Pmod{p}$,
  then $2x_0 \equiv 0 \Pmod{p}$,
  which implies $p \mid 2 x_0$.
  Since $p$ is odd, we must have
  $p \mid x_0$, so
  $x_0 \equiv 0 \Pmod{p}$. But then
  $a \equiv x_0^2 \equiv 0 \Pmod{p}$,
  a contradiction.
  Thus $x^2 \equiv a \Pmod{p}$ has at least
  two incongruent solutions modulo
  $p$ if it has a solution at all.
  
  We now show $x^2 \equiv a \Pmod{p}$
  has at most $2$ incongruent solutions.
  Suppose $x_0, x_1$ are
  solutions. Then
  \[
    x_0^2 \equiv x_1^2 \equiv a \pmod{p}.
  \]
  Then $x_0^2 - x_1^2 \equiv 0 \Pmod{p}$,
  so $p \mid x_0^2 - x_1^2 = (x_0 - x_1)(x_0 + x_1)$.
  Thus $p \mid x_0 - x_1$ or
  $p \mid x_0 + x_1$. In the first
  case, $x_0 \equiv x_1 \Pmod{p}$,
  and in the second case,
  $x_0 \equiv -x_1 \Pmod{p}$. So any
  solution is congruent to either
  $x_0$ or $-x_0$, which means
  that $x^2 \equiv a \Pmod{p}$
  has at most $2$ incongruent
  solutions.
\end{proof}

\begin{corollary}
  Let $p$ be an odd prime and
  $a \in \Z$, $p \nmid a$.
  If $x^2 \equiv a \Pmod{p}$ is
  solvable with $x = x_0$, then
  the two solutions are given by
  $x_0$ and $p - x_0$.
\end{corollary}

\begin{prop}\label{prop:quadratic-residues-count}
  Let $p$ be an odd prime. Then there
  are exactly $(p - 1) / 2$ quadratic
  residues and $(p - 1) / 2$ quadratic
  non-residues modulo $p$.
\end{prop}

\begin{proof}
  For each $1 \le x \le p - 1$, if
  $x^2 \equiv a \Pmod{p}$, then
  $(p - x)^2 \equiv a \Pmod{p}$ as well,
  and these are the only two such
  residues which square to $a$.
  That is, for each pair
  \[
    (1, p - 1), \quad (2, p - 2), \quad
    \dots, \quad (i, p - i)
  \]
  for $1 \le i \le (p - 1) / 2$,
  we get a unique quadratic residue,
  namely $i^2$. Since there are
  $(p - 1) / 2$ pairs of residues
  modulo $p$ formed in this way,
  there are exactly $(p - 1) / 2$
  quadratic residues modulo $p$.
  These are given by
  $1^2, 2^2, \dots, ((p - 1) / 2)^2$.
  The remaining $(p - 1) / 2$
  elements are quadratic
  non-residues.
\end{proof}

\section{The Legendre Symbol}

\begin{definition}
  Let $p$ be an odd prime and
  $a \in \Z$, $p \nmid a$. The
  \emph{Legendre symbol} is
  \[
    \left(\frac{a}{p}\right)
    =
    \begin{cases}
      1 & \text{if $a$ is a quadratic residue modulo $p$,} \\
      -1 & \text{if $a$ is a quadratic non-residue modulo $p$.}
    \end{cases}
  \]
\end{definition}

\begin{example}
  Recall that $1, 3, 4, 5, 9$ are
  quadratic residues modulo $11$, so
  \[
    \left(\frac{1}{11}\right)
    = \left(\frac{3}{11}\right)
    = \left(\frac{4}{11}\right)
    = \left(\frac{5}{11}\right)
    = \left(\frac{9}{11}\right)
    = 1.
  \]
  The quadratic non-residues modulo
  $11$ were $2, 6, 7, 8, 10$, so
  \[
    \left(\frac{2}{11}\right)
    = \left(\frac{6}{11}\right)
    = \left(\frac{7}{11}\right)
    = \left(\frac{8}{11}\right)
    = \left(\frac{10}{11}\right)
    = -1.
  \]
\end{example}

\begin{example}
  Evaluate $\left(\frac{3}{7}\right)$.
  This asks whether $3$ is a quadratic
  residue modulo $7$. That is, whether
  there is a solution to the
  quadratic congruence
  $x^2 \equiv 3 \Pmod{7}$. One can check
  that
  \[
    \{
      1^2, 2^2, 3^2
    \}
    \equiv \{1, 4, 2\} \pmod{7},
  \]
  and these are all of the quadratic
  residues by Proposition
  \ref{prop:quadratic-residues-count}.
  Thus $\left(\frac{3}{7}\right) = -1$.
\end{example}

  \chapter{Oct.~8 --- The Legendre Symbol}

\section{More on the Legendre Symbol}

\begin{exercise}
  Find all the quadratic residues
  modulo $23$.

  We known that the quadratic residues modulo $23$ are the squares of
  $1, \dots, 11$, so
  \begin{align*}
    \{1^2, 2^2, 3^2, 4^2, 5^2, 6^2, 7^2, 8^2, 9^2, 10^2, 11^2\}
    &\equiv
    \{1, 4, 9, 16, 2, 13, 3, 18, 12, 8, 6\} \pmod{23} \\
    &= \{1, 2, 3, 4, 6, 8, 9, 12, 13, 16, 18\}
    \pmod{23}
  \end{align*}
  is the set of quadratic residues modulo $23$.
\end{exercise}

\begin{theorem}[Euler's criterion]
  Let $p$ be an odd prime with
  $a \in \Z$ and $p \nmid a$. Then
  \[
    \left( \frac{a}{p} \right) \equiv a^{(p - 1) / 2} \pmod{p}.
  \]
\end{theorem}

\begin{proof}
  Suppose first that
  $\big(\frac{a}{p}\big) = 1$.
  Then $x^2 \equiv a \pmod{p}$ has a
  solution, say $x = x_0$. Then
  \[
    a^{(p - 1) / 2}
    \equiv (x_0^2)^{(p - 1) / 2}
    \equiv x_0^{p - 1}
    \equiv 1
    \equiv \left( \frac{a}{p} \right)
    \pmod{p}
  \]
  by Fermat's little theorem. Now
  suppose that
  $\big(\frac{a}{p}\big) = -1$. Since
  $p \nmid a$, for each
  $1 \le i \le p - 1$,
  the linear congruence
  $i j \equiv a \Pmod{p}$ has a unique
  solution
  $j$ with $1 \le j \le p - 1$.
  Note that $i \ne j$ since
  $a$ is not a quadratic residue modulo $p$.
  Thus we can pair the residues
  $1, 2, \dots, p - 1$ into
  $(p - 1) / 2$ pairs $(i, j)$ such
  that $i j \equiv a \Pmod{p}$.
  Multiplying these pairs together,
  we get
  \[
    (p - 1)!
    \equiv 1 \cdot 2 \cdot 3 \cdots (p - 1)
    \equiv a^{(p - 1) / 2} \pmod{p}.
  \]
  The left-hand side is congruent to
  $-1 = \big(\frac{a}{p}\big)$
  by Wilson's theorem, which completes
  the proof.
\end{proof}

\begin{example}
  We compute
  $\big(\frac{3}{7}\big)$. Using
  Euler's criterion,
  \[
    \left( \frac{3}{7} \right)
    \equiv 3^{(7 - 1) / 2}
    \equiv 3^3
    \equiv 27
    \equiv 6
    \equiv -1 \pmod{7},
  \]
  so we get that
  $\big(\frac{3}{7}\big) = -1$.
\end{example}

\begin{prop}
  Let $p$ be an odd prime and
  $a, b \in \Z$ with
  $p \nmid a$ and $p \nmid b$. Then
  \begin{enumerate}
    \item $\big(\frac{a^2}{p}\big) = 1$;
    \item if $b \equiv a \Pmod{p}$, then
      $\big(\frac{b}{p}\big) = \big(\frac{a}{p}\big)$;
    \item $\big(\frac{ab}{p}\big)
      = \big(\frac{a}{p}\big)
      \big(\frac{b}{p}\big)$.
  \end{enumerate}
\end{prop}

\begin{proof}
  (1) The congruence
  $x^2 \equiv a^2 \Pmod{p}$
  has a solution $x = a$.

  (2) The congruence
  $x^2 \equiv a \Pmod{p}$
  is equivalent to the congruence
  $x^2 \equiv b \equiv a \Pmod{p}$.

  (3) By Euler's criterion,
  $\big(\frac{ab}{p}\big) \equiv (ab)^{(p - 1) / 2} \equiv a^{(p - 1) / 2} b^{(p - 1) / 2} \equiv \big(\frac{a}{p}\big) \big(\frac{b}{p}\big) \Pmod{p}$.
  Since $\big(\frac{ab}{p}\big)$,
  $\big(\frac{a}{p}\big)$, and
  $\big(\frac{b}{p}\big)$ are each
  $\pm 1$, congruence modulo $p$
  is equivalent to equality
  (since $1 \ne -1$ for $p \ge 3$).
\end{proof}

\begin{example}
  Calculate $\big(\frac{-11}{7}\big)$.
  Using the above properties and
  Euler's criterion, we have
  \[
    \left( \frac{-11}{7} \right)
    =
    \left( \frac{-1}{7} \right)
    \left( \frac{11}{7} \right)
    =
    \left( \frac{-1}{7} \right)
    \left( \frac{4}{7} \right)
    =
    \left( \frac{-1}{7} \right)
    \equiv (-1)^3 \equiv -1 \pmod{7}
  \]
  since $4$ is a quadratic residue
  modulo $7$. So
  $\big(\frac{-11}{7}\big) = -1$.
\end{example}

\section{Particular Cases of the Legendre Symbol}
\begin{remark}
  If $a = \pm 2^{a_0} p_1^{a_1} \cdots p_r^{a_r}$,
  then we have
  \[
    \left( \frac{a}{p} \right)
    =
    \left( \frac{\pm 1}{p} \right)
    \left( \frac{2}{p} \right)^{a_1}
    \left( \frac{p_1}{p} \right)^{a_1}
    \cdots
    \left( \frac{p_r}{p} \right)^{a_r}.
  \]
  Thus to evaluate $\big(\frac{a}{p}\big)$,
  it suffices to understand
  $\big(\frac{-1}{p}\big)$,
  $\big(\frac{2}{p}\big)$, and
  $\big(\frac{q}{p}\big)$ for odd primes $q$.
\end{remark}

\begin{theorem}
  Let $p$ be an odd prime. Then
  \[
    \left( \frac{-1}{p} \right)
    = (-1)^{(p - 1) / 2}
    = \begin{cases}
      1 & \text{if $p \equiv 1 \Pmod{4}$,} \\
      -1 & \text{if $p \equiv 3 \Pmod{4}$.}
    \end{cases}
  \]
\end{theorem}

\begin{proof}
  The first equality follows
  from Euler's criterion. The
  second is a direct computation:
  Note that $p$ can only be
  congruent to $1$ or $3$ modulo $4$.
  If $p \equiv 1 \Pmod{4}$, then
  $p = 1 + 4k$ for some $k \in \Z$. Then
  \[
    (-1)^{(p - 1) / 2}
    = (-1)^{(1 + 4k - 1) / 2}
    = (-1)^{2k}
    = 1.
  \]
  Similarly, if $p \equiv 3 \Pmod{4}$,
  then $p = 3 + 4k$ for some
  $k \in \Z$, and
  \[
    (-1)^{(p - 1) / 2}
    = (-1)^{(3 + 4k - 1) / 2}
    = (-1)^{1 + 2k}
    = -1.
  \]
  This proves the second equality.
\end{proof}

\begin{lemma}[Gauss's lemma]
  Let $p$ be an odd prime and let
  $a \in \Z$ with $p \nmid a$. Let
  $n$ be the number of least positive
  residues of the integers
  \[
    a, \quad 2a, \quad 3a, \quad \dots, \quad
    \left(\frac{p - 1}{2}\right) a
  \]
  that are greater than $p / 2$. Then
  $\big(\frac{a}{p}\big) = (-1)^n$.
\end{lemma}

\begin{proof}
  Let $r_1, \dots, r_n$ be the least
  positive residues among
  $a, 2a, \dots, ((p - 1) / 2)a$
  that are greater than $p / 2$, and
  let $s_1, \dots, s_m$ be the residues
  which are less than $p / 2$.
  Note that none of the $r_i, s_j$
  are congruent to $0$ modulo $p$
  since $p \nmid a$. Now consider the
  $(p - 1) / 2$ integers given by
  \[
    p - r_1, \quad p - r_2, \quad,
    \dots, \quad p - r_n, \quad
    s_1, \quad s_2, \quad \dots, \quad s_m.
  \]
  We claim that this is the set of
  residues $1, 2, \dots, (p - 1) / 2$
  in some order. All elements
  are $\ge 1$, and are $\le (p - 1) / 2$ 
  since they are $< p / 2$ and
  are integers. So it suffices to
  show that there are no duplicates.

  If $p - r_i \equiv p - r_j \Pmod{p}$,
  then $r_i \equiv r_j \Pmod{p}$, so
  $k_i a \equiv k_j a \Pmod{p}$
  for some $k_i \ne k_j$. Since
  $(a, p) = 1$, we can multiply by its
  inverse $\overline{a}$ to get
  $k_i \equiv k_j \Pmod{p}$, which is
  a contradiction.
  By a similar argument, the
  $s_j$ are all distinct.
  It only remains to consider
  $p - r_i \equiv s_j \Pmod{p}$. Then
  \[
    -k_i a \equiv k_j a \pmod{p}
  \]
  for some $1 \le k_i, k_j \le (p - 1) / 2$.
  The congruence then implies
  $-k_i \equiv k_j \Pmod{p}$. But
  $p - k_i > p / 2 \ge (p - 1) \ge k_j$,
  so this congruence is impossible.
  This proves the claim.

  Thus, multiplying all the numbers
  together gives
  \begin{align*}
    \left(\frac{p - 1}{2}\right)!
    \equiv (p - r_1) \cdots (p - r_n)
    s_1 \cdots s_m
    &\equiv (-1)^n a \cdot 2a \cdots \left(\frac{p - 1}{2}\right) a \\
    &\equiv (-1)^n a^{(p - 1) / 2} \left(\frac{p - 1}{2}\right)!
    \pmod{p}.
  \end{align*}
  Since $p \nmid ((p - 1) / 2)!$,
  we can multiply by its inverse to
  get $a^{(p - 1) / 2} \equiv (-1)^n \Pmod{p}$.
  The result then follows since the
  left-hand side is congruent to
  $\big(\frac{a}{p}\big)$ by Euler's
  criterion and $\big(\frac{a}{p}\big),
  (-1)^n$
  are $\pm 1$.
\end{proof}

\begin{example}
  We use Gauss's lemma to calculate
  $\big(\frac{6}{11}\big)$. We have
  $\big(\frac{6}{11}\big) = (-1)^n$,
  where $n$ is the number of least
  positive residues among
  \[
    6, \quad 2 \cdot 6, \quad
    3 \cdot 6, \quad 4 \cdot 6, \quad
    5 \cdot 6
  \]
  that are larger than $11 / 2 = 5.5$.
  Reducing the above gives
  $\{6, 1, 7, 2, 8\}$, so
  $n = 3$. Thus
  $\big(\frac{6}{11}\big) = -1$.
\end{example}

  \chapter{Oct.~13 --- Quadratic Reciprocity}

\begin{quote}
  \begin{emph}
    Why did the man bring his watch to the
    bank? He wanted to save time.
  \end{emph}
\end{quote}

\section{Applications of Gauss's Lemma}

\begin{exercise}
  Calculate the following:
  \[
    \left(\frac{-1}{13}\right), \quad
    \left(\frac{2}{17}\right), \quad
    \left(\frac{-14}{11}\right), \quad
    \left(\frac{18}{23}\right).
  \]
  For the first, since
  $13 \equiv 1 \Pmod{4}$, we have
  $\big(\frac{-1}{13}\big) = 1$.
  For the second, we compute
  \[
    \{2, 2(2), 3(2), 4(2), 5(2), 6(2), 7(2), 8(2)\}
    \equiv \{2, 4, 6, 8, 10, 12, 14, 16\}
    \pmod{17}.
  \]
  Four of these residues are
  greater that $17 / 2 = 8.5$, so
  $\big(\frac{2}{17}\big) = (-1)^4 = 1$.
  For the third, write
  \[
    \left(\frac{-14}{11}\right)
    = \left(\frac{-1}{11}\right)
    \left(\frac{14}{11}\right)
    = \left(\frac{-1}{11}\right)
    \left(\frac{3}{11}\right)
    = -\left(\frac{3}{11}\right)
  \]
  since $11 \equiv 3 \Pmod{4}$.
  To compute $\big(\frac{3}{11}\big)$,
  we list
  \[
    \{3, 2(3), 3(3), 4(3), 5(3)\}
    \equiv \{3, 6, 9, 1, 4\}
    \pmod{11}.
  \]
  Two of the above are greater than $11 / 2 = 5.5$,
  so $\big(\frac{3}{11}\big) = (-1)^2 = 1$.
  Thus $\big(\frac{-14}{11}\big) = -1$.
  For the last,
  \[
    \left(\frac{18}{23}\right)
    = \left(\frac{2}{23}\right)
    \left(\frac{9}{23}\right)
    = \left(\frac{2}{23}\right),
  \]
  which we can compute by enumerating
  \begin{align*}
    &\{2, 2(2), 3(2), 4(2), 5(2), 6(2), 7(2), 8(2), 9(2), 10(2), 11(2)\} \\
    &\quad \quad \equiv \{2, 4, 6, 8, 10, 12, 14, 16, 18, 20, 22\}
    \pmod{23}.
  \end{align*}
  Six of the above are greater than
  $23 / 2 = 11.5$, so
  $\big(\frac{18}{23}\big) = \big(\frac{2}{23}\big) = (-1)^6 = 1$.
\end{exercise}

\begin{theorem}
  Let $p$ be an odd prime. Then
  \[
    \left(\frac{2}{p}\right)
    = (-1)^{(p^2 - 1) / 8}
    =
    \begin{cases}
      1 & \text{if } p \equiv 1, 7 \Pmod{8}, \\
      -1 & \text{if } p \equiv 3, 5 \Pmod{8}.
    \end{cases}
  \]
\end{theorem}

\begin{proof}
  By Gauss's lemma, we have
  $\big(\frac{2}{p}\big) = (-1)^n$,
  where $n$ is the number of least positive
  residues of
  \[
    2, \quad 2(2), \quad 3(2), \quad 4(2), \quad \dots, \quad
    \left(\frac{p - 1}{2}\right) 2.
  \]
  Let $k \in \Z$ with $1 \le k \le (p - 1) / 2$.
  Note that $2k < p / 2$ if and only if
  $k < p / 4$ (we always have $2k < p$),
  so there are $\lfloor p / 4 \rfloor$
  values of $k$ for which $2k < p / 2$.
  Thus, there are $(p - 1) / 2 - \lfloor p / 4 \rfloor$
  values of $k$ for which
  $2k > p / 2$ (recall that $p$ is odd),
  so $n = (p - 1) / 2 - \lfloor p / 4 \rfloor$.
  To show that $(p^2 - 1) / 8$
  and $(p - 1) / 2 - \lfloor p / 4 \rfloor$
  always have the same parity, we can
  just check the four cases:
  \begin{itemize}
    \item $p \equiv 1 \Pmod{8}$.
      Then $p = 8m + 1$ for some $m \in \Z$.
      Then
      \[
        n = \frac{p - 1}{2} - \left\lfloor \frac{p}{4} \right\rfloor
        = \frac{8m + 1 - 1}{2} - \left\lfloor \frac{8m + 1}{4} \right\rfloor
        = 4m - 2m = 2m.
      \]
      Note that this is even.
      On the other hand, we can check that
      \[
        \frac{p^2 - 1}{8}
        = \frac{(8m + 1)^2 - 1}{8}
        = 8m^2 + 2m.
      \]
      This is also even, so
      the parity matches in this case.
    \item Check the cases
      $p \equiv 3, 5, 7 \Pmod{8}$ similarly
      as an exercise.
  \end{itemize}
  Since $(p^2 - 1) / 8$ and $n$ agree
  modulo $2$, we have
  $(-1)^{(p^2 - 1) / 8} = (-1)^n = \big(\frac{2}{p}\big)$.
\end{proof}

\begin{example}
  We have $\big(\frac{2}{23}\big) = 1$
  since $23 \equiv 7 \Pmod{8}$.
\end{example}

\section{Quadratic Reciprocity}

\begin{remark}
  We will now try to understand
  $\big(\frac{q}{p}\big)$ for distinct
  odd primes $p, q$.
\end{remark}

\begin{theorem}[Law of quadratic reciprocity]\label{thm:quadratic-reciprocity}
  Let $p, q$ be distinct odd primes. Then
  \[
    \left(\frac{p}{q}\right)
    \left(\frac{q}{p}\right)
    = (-1)^{(p - 1)(q - 1) / 4}
    =
    \begin{cases}
      1 & \text{if } p \equiv 1 \text{ or } q \equiv 1 \Pmod{4}, \\
      -1 & \text{if } p \equiv q \equiv 3 \Pmod{4}.
    \end{cases}
  \]
\end{theorem}

\begin{remark}
  Quadratic reciprocity allows us to
  simplify the calculation for
  $\big(\frac{p}{q}\big)$.
  For example, consider
  \begin{quote}
    Which primes are quadratic residues modulo
    $17$, i.e. evaluate $\big(\frac{p}{17}\big)$?
  \end{quote}
  This is a finite problem: We may just
  compute all squares modulo $17$.
  Now consider
  \begin{quote}
    For which primes $p$ is $17$ a
    quadratic residue, i.e. evaluate
    $\big(\frac{17}{p}\big)$?
  \end{quote}
  This is a priori an infinite problem,
  but we can convert it to the previous one
  by quadratic reciprocity.
\end{remark}

\begin{example}
  Compute $\big(\frac{7}{53}\big)$.
  We use quadratic reciprocity:
  $7 \equiv 3 \Pmod{4}$ and
  $53 \equiv 1 \Pmod{4}$, so
  \[
    \left(\frac{7}{53}\right)
    =
    \left(\frac{53}{7}\right)
    =
    \left(\frac{4}{7}\right)
    = 1
  \]
  since $4$ is always a square modulo any
  prime.
\end{example}

\begin{example}
  Calculate $\big(\frac{-158}{101}\big)$.
  We can first write
  \[
    \left(\frac{-158}{101}\right)
    = \left(\frac{-1}{101}\right)
    \left(\frac{158}{101}\right)
    = \left(\frac{158}{101}\right)
    = \left(\frac{57}{101}\right)
    = \left(\frac{3}{101}\right)
    \left(\frac{19}{101}\right)
  \]
  since $101 \equiv 1 \Pmod{4}$,
  $158 \equiv 57 \Pmod{101}$, and
  $57 = 3 \cdot 19$. We can now
  apply quadratic reciprocity (note that
  we could not have done this earlier,
  since $158, 57$ are not prime):
  \[
    \left(\frac{-158}{101}\right)
    = \left(\frac{101}{3}\right)
    \left(\frac{101}{19}\right)
    = \left(\frac{2}{3}\right)
    \left(\frac{6}{19}\right)
    = \left(\frac{2}{3}\right)\left(\frac{25}{19}\right)
    = -1 \cdot 1 = 1.
  \]
\end{example}

\begin{lemma}\label{lem:qr-lemma}
  Let $p$ be an odd prime number and
  let $a \in \Z$, $p \nmid a$, $a$ odd.
  Let
  \[
    N = \sum_{j = 1}^{(p - 1) / 2}
    \left\lfloor \frac{ja}{p} \right\rfloor.
  \]
  Then $\big(\frac{a}{p}\big) = (-1)^N$.
\end{lemma}

\begin{proof}
  Let $r_1, r_2, \dots, r_n$ be the least
  non-negative residues among
  $a, 2a, \dots, ((p - 1) / 2) a$
  that are $> p / 2$. Likewise, let
  $s_1, \dots, s_m$ be the remaining
  residues that are $< p / 2$. Note that
  \[
    r_1, \dots, r_n, s_1, \dots, s_m
  \]
  are all distinct modulo $p$ (they come
  from $a, 2a, 3a, \dots , ((p - 1) / 2)a$,
  which are distinct since $p \nmid a$).
  This means that the fractions
  $r_i / p$, $s_j / p$  are also all
  distinct. Then
  \[
    ja
    = p \cdot \frac{ja}{p}
    = p \left(\left\lfloor \frac{ja}{p} \right\rfloor + \frac{\text{remainder}}{p}\right)
    = p \left\lfloor \frac{ja}{p} \right\rfloor + \text{remainder depending on $j$},
  \]
  where the remainders are exactly the
  numbers $r_1, \dots, r_n, s_1, \dots, s_m$.
  Then
  \[
    \sum_{j = 1}^{(p - 1) / 2} ja
    = \sum_{j = 1}^{(p - 1) / 2}
    p \left\lfloor \frac{ja}{p} \right\rfloor
    + \sum_{i = 1}^n r_i + \sum_{j = 1}^m s_j.
    \tag{$1$}
  \]
  Note also that
  \[
    \sum_{j = 1}^{(p - 1) / 2} j
    = \sum_{i = 1}^n (p - r_i) + \sum_{j = 1}^m s_j
    = pn - \sum_{i = 1}^n r_i + \sum_{j = 1}^m s_j.
    \tag{$2$}
  \]
  Subtracting $(2)$ from $(1)$ gives the
  equation
  \[
    \sum_{j = 1}^{(p - 1) / 2} j(a - 1)
    = \sum_{j = 1}^{(p - 1) / 2}
    p \left\lfloor \frac{ja}{p} \right\rfloor
    - pn + 2\sum_{i = 1}^n r_i.
  \]
  Taking the above equation modulo $2$,
  since $a$ is odd, we get
  \[
    \sum_{j = 1}^{(p - 1) / 2} p \left\lfloor \frac{ja}{p} \right\rfloor
    - pn \equiv 0 \Pmod{2},
  \]
  so $pN \equiv pn \Pmod{2}$,
  so $N \equiv n \Pmod{2}$
  since $2 \nmid p$.
  So $(-1)^N = (-1)^n = \big(\frac{a}{p}\big)$
  by Gauss's lemma.
\end{proof}

\begin{example}
  We compute $\big(\frac{7}{11}\big)$
  using Lemma \ref{lem:qr-lemma}.
  We calculate
  \[
    N = \sum_{j = 1}^5 \left\lfloor \frac{7j}{11} \right\rfloor
    = \left\lfloor \frac{7}{11} \right\rfloor
    + \left\lfloor \frac{14}{11} \right\rfloor
    + \left\lfloor \frac{21}{11} \right\rfloor
    + \left\lfloor \frac{28}{11} \right\rfloor
    + \left\lfloor \frac{35}{11} \right\rfloor
    = 0 + 1 + 1 + 2 + 3 = 7,
  \]
  so $\big(\frac{7}{11}\big) = (-1)^7 = -1$
  by Lemma \ref{lem:qr-lemma}.
\end{example}

  \chapter{Oct.~15 --- Quadratic Reciprocity, Part 2}

\begin{quote}
  \emph{What do you call a place in America that receives shipments of pollinators? A U.S. bee port.}
\end{quote}

\section{Proof of Quadratic Reciprocity}

\begin{proof}[Proof of Theorem \ref{thm:quadratic-reciprocity}]
  Without loss of generality, assume
  $p > q$.
  Consider a $q \times p$ grid on $\R^2$.
  Let $L$ be the line from
  $(0, 0)$ to $N = (q, p)$. Let
  $A = ((p - 1) / 2, 0)$,
  $B = (0, (q - 1) / 2)$. Let
  $M$ be the intersection of
  $L$ with the line
  $x = (p - 1) / 2$ and $D$ be the
  intersection of $L$ 
  with the line $y = (q - 1) / 2$.
  Also let $C = ((p - 1) / 2, (q - 1) / 2)$.
  We count the number of lattice points in
  the rectangle $OABC$, not including the
  axes.
  This number is clearly
  $(p - 1)(q - 1) / 4$. Now observe that:
  \begin{enumerate}
    \item The line $ON$ has slope
      $q / p$. In particular, $ON$
      contains no lattice points.
    \item The $y$-coordinate of $M$ is
      $((p - 1) / 2) (q / p) = q / 2 - q / 2p$.
      This lies between the consecutive
      integers
      $(q - 1) / 2$ and $(q + 1) / 2$:
      \[
        \frac{q - 1}{2} = \frac{q}{2} - \frac{1}{2}
        < \frac{q}{2} - \frac{q}{2p}
        < \frac{q}{2} < \frac{q + 1}{2}.
      \]
  \end{enumerate}
  So the number of lattice points
  in $OABC$ excluding the axes, and
  below the line $ON$ is
  \[
    N_1 = \sum_{j = 1}^{(p - 1) / 2}
    \left\lfloor \frac{jq}{p} \right\rfloor.
  \]
  Likewise, the number of lattice points
  above the line $ON$ is
  \[
    N_2 = \sum_{j = 1}^{(q - 1) / 2} \left\lfloor \frac{jp}{q} \right\rfloor.
  \]
  Thus the total number of lattice points
  in question is
  $N_1 + N_2 = (p - 1)(q - 1) / 4$. Then
  \[
    \left(\frac{p}{q}\right)
    \left(\frac{q}{p}\right)
    = (-1)^{N_2} (-1)^{N_1}
    = (-1)^{N_1 + N_2}
    = (-1)^{(p - 1)(q - 1) / 4}
  \]
  by Lemma \ref{lem:qr-lemma}, which
  proves the claim.
\end{proof}

\section{Applications of Quadratic Reciprocity}

\begin{remark}
  Note that we have characterized the
  primes for which $-1$ and $2$ are
  quadratic residues.
\end{remark}

\pagebreak
\begin{example}
  For what primes $p$ is
  $3$ a quadratic residue?
  It suffices to compute when $\big(\frac{3}{p}\big) = 1$.
  By quadratic reciprocity, we have
  \[
    \left(\frac{3}{p}\right)
    =
    \begin{cases}
      \big(\frac{p}{3}\big) & p \equiv 1 \Pmod{4},\\
      -\big(\frac{p}{3}\big) & p \equiv 3 \Pmod{4}.
    \end{cases}
  \]
  Note that the only (non-zero) quadratic
  residue modulo $3$ is $1$.
  If $p \equiv 1 \Pmod{4}$, then
  $\big(\frac{p}{3}\big) = 1$
  if and only if $p \equiv 1 \Pmod{3}$.
  If $p \equiv 3 \Pmod{4}$,
  then $\big(\frac{p}{3}\big) = -1$
  if and only if $p \equiv 2 \Pmod{3}$.
  By the Chinese remainder theorem,
  we can rewrite the first condition
  as $p \equiv 1 \Pmod{12}$ and the
  second condition as
  $p \equiv -1 \Pmod{12}$. Thus
  we see that $\big(\frac{3}{p}\big) = 1$
  if and only if
  $p \equiv \pm 1 \Pmod{12}$.
\end{example}

\begin{example}
  Characterize the primes $p$ for
  which both $2$ and $3$ are
  quadratic residues modulo $p$.
  We want $p$ such that
  $\big(\frac{2}{p}\big) = \big(\frac{3}{p}\big) = 1$.
  We already know that
  $\big(\frac{2}{p}\big) = 1$
  if and only if $p \equiv \pm 1 \Pmod{8}$
  and $\big(\frac{3}{p}\big) = 1$
  if and only if $p \equiv \pm 1 \Pmod{12}$.
  So by the Chinese remainder theorem,
  we have
  $\big(\frac{2}{p}\big) = \big(\frac{3}{p}\big) = 1$
  if and only if
  $p \equiv \pm 1 \Pmod{24}$.
\end{example}

\begin{example}
  Characterize the primes $p$ for which
  $13$ is a quadratic residue modulo $p$.
  By quadratic reciprocity,
  we have
  $\big(\frac{13}{p}\big) = \big(\frac{p}{13}\big)$.
  The non-zero
  quadratic residues modulo $13$ are
  \[
    \{1, 3, 4, 9, 10, 12\},
  \]
  So $\big(\frac{13}{p}\big) = \big(\frac{p}{13}\big) = 1$
  if and only if
  $p \equiv 1, 3, 4, 9, 10, 12 \equiv \pm 1, \pm 3, \pm 4 \Pmod{13}$.
\end{example}

\begin{example}
  Characterize the primes $p$ for
  which $11$ is a quadratic residue
  modulo $p$. By quadratic reciprocity,
  we have that
  \[
    \left(\frac{11}{p}\right)
    =
    \begin{cases}
      \big(\frac{p}{11}\big) & p \equiv 1 \Pmod{4},\\
      -\big(\frac{p}{11}\big) & p \equiv 3 \Pmod{4}.
    \end{cases}
  \]
  The quadratic residues modulo $11$ are
  $1, 3, 4, 5, 9$, and the
  quadratic non-residues are
  $2, 6, 7, 8, 10$. If
  $p \equiv 1 \Pmod{4}$, then
  $\big(\frac{p}{11}\big) = 1$
  if and only if $p \equiv 1, 3, 4, 5, 9 \Pmod{11}$,
  and if $p \equiv 3 \Pmod{4}$, then
  $\big(\frac{p}{11}\big) = -1$
  if and only if
  $p \equiv 2, 6, 7, 8, 10 \Pmod{11}$.
  One can check each of these
  cases by the Chinese
  remainder theorem, and one gets
  $\big(\frac{11}{p}\big) = 1$
  if and only if
  $p \equiv 1, 5, 7, 9, 19, 25, 35, 37, 39, 43 \Pmod{44}$, or
  \[
    p \equiv \pm 1, \pm 5, \pm 7, \pm 9, \pm 19 \pmod{44}.
  \]
\end{example}

\begin{example}
  Characterize the primes $p$ for which
  $-1$ and $2$ are both quadratic
  residues modulo $p$. Recall that
  $\big(\frac{-1}{p}\big) = 1$
  if and only if $p \equiv 1 \Pmod{4}$
  and $\big(\frac{2}{p}\big) = 1$
  if and only if $p \equiv \pm 1 \Pmod{8}$,
  so we see that
  $\big(\frac{-1}{p}\big) = \big(\frac{2}{p}\big) = 1$
  if and only if $p \equiv 1 \Pmod{8}$.
\end{example}

\begin{example}
  Characterize the primes $p$ for which
  both $-1$ and $3$ are quadratic
  residues modulo $p$.
  Recall that $\big(\frac{-1}{p}\big) = 1$
  if and only if $p \equiv 1 \Pmod{4}$
  and $\big(\frac{3}{p}\big) = 1$
  if and only if
  $p \equiv \pm 1 \Pmod{12}$.
  So we have
  $\big(\frac{-1}{p}\big) = \big(\frac{3}{p}\big) = 1$
  if and only if $p \equiv 1 \Pmod{12}$.
\end{example}

\begin{exercise}
  Characterize the primes $p$ for which
  $3$ and $5$ are quadratic residues
  modulo $p$.
\end{exercise}

  \chapter{Oct.~20 --- Primitive Roots}

\begin{quote}
  \emph{
  What happens when you step on a grape?
Nothing, it lets out a little whine.}
\end{quote}

\section{Orders}

\begin{remark}
  Let $m$ be a positive integer and
  $(a, m) = 1$. By Euler's theorem, we know
  that
  \[
    a^{\varphi(m)} \equiv 1 \pmod{m}.
  \]
  However, it may happen that
  $a^g \equiv 1 \pmod{m}$ for some smaller
  $g$.
\end{remark}

\begin{definition}
  Let $a, m \in \Z$ with $m > 0$,
  $(a, m) = 1$. Then the \emph{order of $a$ modulo $m$},
  denoted $\ord_m a$, is the least
  positive integer $n$ such that
  $a^n \equiv 1 \pmod{m}$.
\end{definition}

\begin{example}
  We compute $\ord_7 2$. We can compute that
  \begin{align*}
    2^1 &\equiv 2 \pmod{7},\\
    2^2 &\equiv 4 \pmod{7},\\
    2^3 &\equiv 1 \pmod{7},
  \end{align*}
  so we see that $\ord_7 2 = 3$. Note
  that Euler's theorem only guarantees
  $\ord_7 2 \le \phi(7) = 6$.
\end{example}

\begin{example}\label{ex:order-7-3}
  We compute $\ord_7 3$. We can compute
  that
  \begin{align*}
    3^1 &\equiv 3 \pmod{7},\\
    3^2 &\equiv 2 \pmod{7},\\
    3^3 &\equiv 6 \pmod{7},\\
    3^4 &\equiv 4 \pmod{7},\\
    3^5 &\equiv 5 \pmod{7},\\
    3^6 &\equiv 1 \pmod{7},
  \end{align*}
  so we see that $\ord_7 3 = 6$.
\end{example}

\begin{prop}\label{prop:order-divides}
  Let $a, m \in \Z$ with $m > 0$ and
  $(a, m) = 1$. Then $a^n \equiv 1 \Pmod{m}$
  for some positive integer $n$ if and only
  if $\ord_m a \mid n$. In particular,
  $\ord_m a \mid \varphi(m)$.
\end{prop}

\begin{proof}
  $(\Rightarrow)$ Suppose that
  $a^n \equiv 1 \Pmod{m}$. By the
  division algorithm, there exists
  $q, r \in \Z$ such that
  \[
    n = q (\ord_m a) + r, \quad
    0 \le r < \ord_m a.
  \]
  Then $1 = a^n \equiv a^{q (\ord_m a) + r} \equiv (a^{\ord_m a})^q a^r \equiv a^r \Pmod{m}$,
  which can only happen if $r = 0$
  by the definition of $\ord_m a$
  and $0 \le r < \ord_m a$. Therefore,
  $\ord_m a \mid n$.

  $(\Leftarrow)$ Suppose that
  $\ord_m a \mid n$. Then $n = q (\ord_m a)$,
  so $a^n = a^{q (\ord_m a)} \equiv (a^{\ord_m a})^q \equiv 1 \Pmod{m}$.
\end{proof}

\begin{example}
  We compute $\ord_{13} 2$.
  By Proposition \ref{prop:order-divides},
  it suffices to check divisors
  of $\varphi(13) = 12$:
  \begin{align*}
    2^1 &\equiv 2 \pmod{13},\\
    2^2 &\equiv 4 \pmod{13},\\
    2^3 &\equiv 8 \pmod{13},\\
    2^4 &\equiv 3 \pmod{13},\\
    2^6 &\equiv 12 \pmod{13},\\
    2^{12} &\equiv 1 \pmod{13},
  \end{align*}
  thus $\ord_{13} 2 = 12$. Note that we
  did not need to compute $2^7$, $2^8$, etc.
  to verify this.
\end{example}

\begin{prop}\label{prop:order-congruence}
  Let $a, m \in \Z$ with $m > 0$ and
  $(a, m) = 1$. If $i, j$ are
  non-negative integers, then
  $a^i \equiv a^j \Pmod{m}$ if and only
  if $i \equiv j \Pmod{{\ord_m a}}$.
\end{prop}

\begin{proof}
  Without loss of generality, suppose
  $i \ge j$.

  $(\Rightarrow)$ Assume
  $a^i \equiv a^j \Pmod{m}$. Then we can
  write
  \[
    a^j \equiv a^i \equiv a^j a^{i - j}
    \pmod{m}.
  \]
  Since $(a, m) = 1$, we can cancel
  $a^j$ to get
  $1 \equiv a^{i - j} \Pmod{m}$.
  Then by Proposition \ref{prop:order-divides},
  $\ord_m a \mid i - j$.

  $(\Leftarrow)$ Assume
  $i \equiv j \Pmod{{\ord_m a}}$. Then
  $\ord_m a \mid i - j$, so there exists
  $n \in \Z$ such that $i - j = n (\ord_m a)$.
  Thus $i = j + n (\ord_m a)$, and we have
  $a^i \equiv a^{j + n (\ord_m a)} \equiv a^j (a^{\ord_m a})^n \equiv a^j \Pmod{m}$.
\end{proof}

\begin{example}
  We have seen previously that $\ord_7 2 = 3$.
  So if $i, j$ are non-negative integers,
  then $2^i \equiv 2^j \Pmod{7}$
  if and only if $i \equiv j \Pmod{3}$
  by Proposition \ref{prop:order-congruence}.
  Note that
  \[
    2000 \equiv 2 \pmod{3},
  \]
  so we can calculate
  $2^{2000} \equiv 2^2 \equiv 4 \Pmod{7}$.
\end{example}

\section{Primitive Roots}

\begin{definition}
  Let $r, m \in \Z$ with $m > 0$ and
  $(r, m) = 1$. Then $r$ is called a
  \emph{primitive root modulo $m$} if
  $\ord_m r = \varphi(m)$.
\end{definition}

\begin{remark}
  See \emph{primitive root diffusers}
  for an interesting application
  (also \emph{quadratic residue diffusers}).
\end{remark}

\begin{example}
  We have seen that
  $3$ is a primitive root modulo $7$ and
  $2$ is a primitive root modulo $13$.
  On the other hand, $2$ is not a primitive
  root modulo $7$.
\end{example}

\begin{example}
  We prove that there are no primitive
  roots modulo $8$. The reduced residues
  modulo $8$ are $1, 3, 5, 7$, and
  $\varphi(8) = 4$. But
  $1^2 \equiv 3^2 \equiv 5^2 \equiv 7^2 \equiv 1$,
  so none of these are primitive roots
  modulo $8$.

  In particular, not all
  integers $m$ possess a primitive root.
  The \emph{primitive root theorem} (later)
  tells us that $m$ has a primitive root
  if and only if $m = 1, 2, 4, p^k, 2p^k$,
  where $p$ is an odd prime.
\end{example}

\begin{prop}\label{prop:primitive-root-generate}
  Let $r$ be a primitive root modulo $m$.
  Then
  $\{r, r^2, r^3, \dots, r^{\varphi(m)}\}$
  is a complete set of reduced residues
  modulo $m$.
\end{prop}

\begin{proof}
  Since $r$ is a primitive root modulo $m$,
  we have $(r, m) = 1$, and so
  $(r^n, m) = 1$ for any $n \ge 1$.
  Also, there are $\varphi(m)$ elements
  in the list, so it remains to show that
  they are distinct modulo $m$.

  To do this, suppose that $r^i \equiv r^j \Pmod{m}$
  for some $1 \le i, j \le \varphi(m)$.
  Then Proposition \ref{prop:order-congruence}
  implies that
  $i \equiv j \Pmod{{\varphi(m)}}$, so
  $i = j$. Thus the $r^i$ are distinct
  modulo $m$.
\end{proof}

\begin{remark}
  Proposition \ref{prop:primitive-root-generate}
  says that a primitive root (when it exists)
  generates the reduced residues modulo
  $m$.
\end{remark}

\begin{example}
  Recall that $3$ is a primitive root
  modulo $7$. We saw in Example
  \ref{ex:order-7-3} that
  \[
    \{3^1, 3^2, 3^3, 3^4, 3^5, 3^6\}
    \equiv \{3, 2, 6, 4, 5, 1\}
    \pmod{7},
  \]
  in particular this is a complete set of
  reduced residues modulo $7$.
\end{example}

\begin{example}
  Recall that $2$ is a primitive root
  modulo $13$. We can compute
  \[
    \{2^1, 2^2, 2^3, 2^4, 2^5, 2^6,
    2^7, 2^8, 2^9, 2^{10}, 2^{11}, 2^{12}\}
    \equiv
    \{2, 4, 8, 3, 6, 12, 11, 9, 5, 10, 7, 1\}
    \pmod{13},
  \]
  which is a complete set of reduced residues
  modulo $13$.
\end{example}

\begin{remark}
  If a primitive root exists, it is
  in general not unique. We will determine
  how many there are next lecture
  (we will see that there are
  $\varphi(\varphi(m))$ of them).
\end{remark}

\begin{exercise}
  Show there are no primitive roots
  modulo $12$.

  To do this, write
  $\Z / 12\Z \cong \Z / 3\Z \times \Z / 4\Z$,
  then we have
  \[(\Z / 12\Z)^\times \cong (\Z / 3\Z)^\times \times (\Z / 4\Z)^\times = \Z / 2\Z \times \Z / 2\Z\]
  which is not cyclic.
  Alternatively, one can just compute
  directly
  for $(\Z / 12\Z)^\times = \{1, 5, 7, 11\}$
  that
  \[
    1^2 \equiv 5^2 \equiv 7^2 \equiv 11^2 \equiv 1 \pmod{12},
  \]
  so none of these can be primitive
  roots modulo $12$.
\end{exercise}

  \chapter{Oct.~22 --- Primitive Roots, Part 2}

\section{More on Primitive Roots}

\begin{prop}
  Let $a, m \in \Z$ with $m > 0$ and
  $(a, m) = 1$. If $i$ is a positive
  integer, then
  \[
    \ord_m(a^i)
    = \frac{\ord_m a}{(\ord_m a, i)}.
  \]
\end{prop}

\begin{proof}
  Let $d = (\ord_m a, 1)$. Then there exist
  $b, c \in \Z$ such that
  $\ord_m a = db$, $i = dc$ and
  $(b, c) = 1$. Note
  \[
    (a^i)^b
    = (a^{dc})^{(\ord_m a)/d}
    = (a^c)^{\ord_m a}
    = (a^{\ord_m a})^c
    = 1 \pmod{m}.
  \]
  By Proposition \ref{prop:order-divides},
  this implies $\ord_m(a^i) \mid b$. Also,
  \[
    1 \equiv (a^i)^{\ord_m(a^i)}
    \equiv a^{i \ord_m (a^i)} \pmod{m},
  \]
  so by Proposition \ref{prop:order-divides},
  $\ord_m a \mid i \ord_m(a^i)$. Thus
  $db \mid dc \ord_m(a^i)$, so
  $b \mid c \ord_m(a^i)$. Since
  $(b, c) = 1$, we must have
  $b \mid \ord_m(a^i)$. Thus we see that
  $\ord_m(a^i) = b = (\ord_m a) / d = (\ord_m a) / (\ord_m a, i)$.
\end{proof}

\begin{corollary}\label{cor:order-power-coprime}
  Let $a, m \in \Z$ with $m > 0$ and
  $(a, m) = 1$. If $i$ is a positive
  integer, then
  \[
    \ord_m(a^i) = \ord_m a
  \]
  if and only if $(\ord_m a, i) = 1$.
\end{corollary}

\begin{corollary}\label{cor:primitive-root-count}
  If a primitive root modulo $m$ exists,
  then there are exactly $\varphi(\varphi(m))$
  incongruent primitive roots modulo $m$.
\end{corollary}

\begin{proof}
  Let $r$ be a primitive root. Then
  the $\ord_m r = \varphi(m)$. By
  Proposition \ref{prop:primitive-root-generate},
  the set
  \[
    \{r^1, r^2, \dots, r^{\varphi(m)}\}
  \]
  is a reduced residue system modulo $m$.
  If $1 \le i \le \varphi(m)$, then by
  Corollary \ref{cor:order-power-coprime},
  $\ord_m(r^i) = \ord_m r = \varphi(m)$
  if and only if $(i, \varphi(m)) = 1$.
  There are $\varphi(\varphi(m))$
  such $i$, and each gives a distinct primitive root.
\end{proof}

\begin{example}
  We showed previously that $3$ is a
  primitive root modulo $7$. There are
  exactly
  \[\varphi(\varphi(7)) = \varphi(6) = 2\]
  primitive roots modulo $7$. In particular,
  we must have $\ord_m (3^i) = \varphi(7)$
  if and only if $(i, \varphi(7)) = (1, 6) = 1$.
  Thus $i = 1, 5$, so $3^1 = 3$ and
  $3^5 \equiv 5 \pmod{7}$ are the
  two primitive roots modulo $7$.
\end{example}

\begin{example}
  Recall that
  $2$ is a primitive root modulo $13$.
  Thus there are $\varphi(\varphi(13)) = \varphi(2) = 4$
  primitive roots. Find the other three
  primitive roots as an exercise.
\end{example}

\begin{remark}
  Note that $\varphi(\varphi(8)) = \varphi(4) = 2$,
  but this does not imply that $8$ has
  $2$ primitive roots. We need to know that
  a primitive root exists first for Corollary
  \ref{cor:primitive-root-count}
  to apply.
\end{remark}

\section{Primitive Roots for Primes}

\begin{theorem}[Lagrange]\label{thm:lagrange}
  Let $p$ be a prime and let
  \[
    f(x) = a_n x^n + a_{n-1} x^{n-1} + \cdots + a_1 x + a_0
  \]
  be a polynomial with degree $n$ and
  integer coefficients $a_0, a_1, \dots, a_n$,
  such that $p \nmid a_n$. Then the
  congruence
  \[
    f(x) \equiv 0 \pmod{p}
  \]
  has at most $n$ incongruent solutions.
\end{theorem}

\begin{proof}
  We proceed by induction on $n$. Suppose
  $n = 1$. Then $f(x) = a_1 x + a_0$
  where $p \nmid a_1$. Then
  \[
    a_1 x + a_0 \equiv 0 \pmod{p},
  \]
  which is equivalent to
  $a_1 x = -a_0 \Pmod{p}$. Now since
  $p \nmid a_1$, we can multiply both
  sides by $\overline{a}_1$ to get
  $x = -a_0 \overline{a}_1$.
  This proves the base case.

  Suppose $k \ge 1$ and that the theorem
  holds for polynomials of degree
  $k$. Let $n = k + 1$, then we can write
  \[
    f(x) = a_{k + 1} x^{k + 1} + \cdots + a_1 x + a_0
  \]
  where $p \nmid a_{k + 1}$. If
  $f(x) \equiv 0 \Pmod{p}$ has no solutions,
  then we are done. Now suppose $x_0$ is a
  solution. By polynomial long division,
  there exists a polynomial $q(x)$ with
  integer coefficients such that
  \[
    f(x) = (x - x_0) q(x) + r
  \]
  for some integer $r$, where $q(x)$
  has degree $k$. Note that
  \[
    0 \equiv f(x_0) \equiv (x_0 - x_0) q(x_0) + r
    \equiv r \pmod{p},
  \]
  so $r \equiv 0 \Pmod{p}$, and we have
  $f(x) \equiv (x - x_0) q(x) \Pmod{p}$. If
  \[
    0 \equiv f(x_1) \equiv (x_1 - x_0) q(x)
    \pmod{p},
  \]
  then $x_1 - x_0 \equiv 0 \Pmod{p}$
  or $q(x_1) \equiv 0 \Pmod{p}$.
  If $x_1 \not\equiv x_0 \Pmod{p}$,
  then $q(x_1) \equiv 0 \Pmod{p}$, and
  $q(x)$ has at most $k$ roots by the
  induction hypothesis. Thus
  $f(x)$ has at most $k + 1$ roots.
\end{proof}

\begin{prop}\label{prop:d-roots-mod-p}
  Let $p$ be a prime and $d \in \Z$ with
  $d > 0$ and $d \mid p - 1$. Then the
  congruence
  \[
    x^d - 1 \equiv 0 \pmod{p}
  \]
  has exactly $d$ incongruent solutions
  modulo $p$.
\end{prop}

\begin{proof}
  Since $d \mid p - 1$, there exists
  $e \in \Z$ such that $p - 1 = de$.
  Note that if $p \nmid x$, then
  \[
    0 \equiv x^{p - 1} - 1
    \equiv x^{de - 1}
    \equiv (x^d - 1)(x^{d(e - 1)} + x^{d(e - 2)} + \cdots + x^d + 1)
    \pmod{p}.
  \]
  Thus $x^d - 1 \equiv 0 \Pmod{p}$
  (call this $(1)$) or
  $x^{d(e - 1)} + x^{d(e - 2)} + \cdots + x^d + 1 \equiv 0 \Pmod{p}$
  (call this $(2)$). By
  Theorem \ref{thm:lagrange},
  $(2)$ has at most $d(e - 1) = p - 1 - d$
  solutions. Also $(1)$ has at most $d$
  solutions. By Fermat's little theorem,
  $x^{p - 1} - 1 \equiv \Pmod{p}$
  has exactly $p - 1$ solutions, so
  $x^d - 1 \equiv 0 \Pmod{p}$ has least
  $d$ solutions. Therefore, it has
  exactly $d$ solutions.
\end{proof}

\begin{remark}
  Proposition \ref{prop:d-roots-mod-p}
  is a generalization of the fact that
  $x^2 \equiv 1 \Pmod{p}$ has exactly
  $2$ solutions for odd primes $p$.
\end{remark}

\begin{example}
  Prove that $3$ is a primitive root
  modulo $43$, and then use this to
  calculate all elements of order $14$.
  To show that $3$ is a primitive root,
  we need to check $3^i$ for
  $i \mid \varphi(43) = 42$, so for
  \[
    i = 1, 2, 3, 6, 7, 14, 21, 42.
  \]
  We can compute that
  \begin{align*}
    3^1 &\equiv 3 \pmod{43},\\
    3^2 &\equiv 9 \pmod{43},\\
    3^3 &\equiv 27 \pmod{43},\\
    3^6 &\equiv 3^4 \cdot 3^2 \equiv (-5) \cdot 9 \equiv -2 \pmod{43},\\
    3^7 &\equiv -6 \pmod{43},\\
    3^{14} &\equiv 36 \equiv -7 \pmod{43},\\
    3^{21} &\equiv 42 \equiv -1 \pmod{43},\\
    3^{42} &\equiv 1 \pmod{43}.
  \end{align*}
  This confirms that $3$ is a primitive root modulo
  $43$. To find elements of order $14$, we
  want $i$ such that
  \[
    14 = \ord_{43}(3^i)
    = \frac{\ord_{43}(3)}{(\ord_{43}(3), i)}
    = \frac{42}{(42, i)},
  \]
  so we want $(42, i) = 42 / 14 = 3$.
  This works for
  $i = 3, 9, 15, 27, 33, 39$. Thus the
  elements of order $14$ are represented
  by $3^3, 3^9, 3^{15}, 3^{27}, 3^{33}, 3^{39}$ modulo $43$.
\end{example}

\end{document}
