\documentclass[12pt, letterpaper, oneside]{book}
\usepackage[margin={0.6in, 0.75in}]{geometry}
\usepackage{microtype}
% \usepackage{kpfonts}
\usepackage{amsmath, amssymb, amsthm}
\usepackage{parskip}
\usepackage[many]{tcolorbox}
\usepackage{footnote}
\usepackage{cancel}
\usepackage{titlesec}
\usepackage{pgffor}
\usepackage[shortlabels, inline]{enumitem}
\usepackage{hyperref}
\usepackage{tikz-cd}

\usepackage[overload]{textcase}

\renewcommand{\chaptername}{Lecture}
\newtheorem{axiom}{Axiom}[chapter]
\newtheorem{theorem}{Theorem}[chapter]
\newtheorem{prop}{Proposition}[chapter]
\newtheorem{corollary}{Corollary}[theorem]
\newtheorem{lemma}{Lemma}[chapter]
\newtheorem{conjecture}{Conjecture}[theorem]
\theoremstyle{definition}
\newtheorem{definition}{Definition}[chapter]
\newtheorem{exercise}{Exercise}[chapter]
\newtheorem{example}{Example}[definition]
\newtheorem*{remark}{Remark}

\tcbset{sharp corners, breakable, enhanced, parbox=false}
\newtcolorbox{mybox}[3][]
{
  colframe = #2!150,
  colback  = #2!5,
  coltitle = #2!0!white,  
  title    = {#3},
  #1,
}

\titleformat{\chapter}[display]
    {\normalfont\huge\bfseries}{\chaptertitlename\ \thechapter}{20pt}{\Huge}
\titlespacing*{\chapter}{0pt}{0pt}{40pt}

\newcommand{\R}{\mathbb{R}}
\newcommand{\N}{\mathbb{N}}
\newcommand{\Z}{\mathbb{Z}}
\newcommand{\C}{\mathbb{C}}
\newcommand{\Q}{\mathbb{Q}}
\newcommand{\F}{\mathbb{F}}
\newcommand{\Mod}[1]{\ {\mathrm{mod}\ #1}}
\newcommand{\Pmod}[1]{\ (\mathrm{mod}\ #1)}

\DeclareMathOperator{\ord}{ord}
\DeclareMathOperator{\lcm}{lcm}
\DeclareMathOperator{\re}{Re}
\DeclareMathOperator{\im}{Im}

\title{MATH 4150: Introduction to Number Theory}
\author{Frank Qiang\\Instructor: Joshua Stucky}
\date{Georgia Institute of Technology\\Fall 2025}

\begin{document}
  \maketitle

  \begingroup
  \let\cleardoublepage\clearpage
  \tableofcontents
  \endgroup

  % \foreach \i in {00, 01, 02, 03, 04, ..., 50} {%
  %   \edef\FileName{lectures/lecture\i.tex}%     The % here are necessary to eliminate any
  %   \IfFileExists{\FileName}{%  spurious spaces that may get inserted
  %      \input{\FileName}%       at these points
  %   }
  % }
  \chapter{Aug.~18 --- Divisibility}

\begin{quote}
  \emph{Something something pair a' docks.} (I forgot to write it down oops.)
\end{quote}

\section{Basic Properties of Divisibility}

\begin{definition}
  Let $a, b \in \Z$. We say that
  $a$ \emph{divides} $b$, and we write
  $a \mid b$, if there exists $c \in \Z$
  such that $b = ac$. We also say that
  $a$ is a \emph{divisor} (or \emph{factor})
  of $b$. We write
  $a \nmid b$ if $a$ does not divide $b$.
\end{definition}

\begin{example} We have the following:
  \begin{enumerate}
    \item We have $3 \mid 6$ since
      $6 = 3 \cdot 2$, and $3 \mid -6$ since
      $-6 = 3 \cdot (-2)$.
    \item For any $a \in \Z$, we have $a \mid 0$
      since $0 = a \cdot 0$.
    \item Technically, we have
      $0 \mid 0$, but do not confuse this
      with the indeterminate form $0 / 0$.
  \end{enumerate}
\end{example}

\begin{prop}
  Let $a, b, c \in \Z$. If $a \mid b$ and
  $b \mid c$, then $a \mid c$. In particular,
  divisibility is transitive.
\end{prop}

\begin{proof}
  Since $a \mid b$ and $b \mid c$, there
  exist integers $e, f$ such that
  $b = ae$ and $c = bf$. We can write
  \[
    c = bf = (ae)f = a(ef),
  \]
  so that $a$ divides $c$ by definition.
\end{proof}

\begin{prop}
  Let $a, b, c, m, n \in \Z$. If $c \mid a$
  and $c \mid b$, then $c \mid (am + bn)$. In
  other words, $c$ divides any integral
  linear combination of $a$ and $b$.
\end{prop}

\begin{proof}
  Since $c \mid a$ and $c \mid b$, we have
  $a = ce$ and $b = cf$ for some $e, f \in \Z$.
  Then
  \[
    am + bn = (ce)m + (cf)n
    = c(em + fn),
  \]
  so that $c$ divides $am + bn$ by definition.
\end{proof}

\section{The Division Algorithm}

\begin{definition}
  Let $x \in \R$. The
  \emph{greatest integer function} (or
  \emph{floor function}) of
  $x$, denoted $[x]$ (or
  $\lfloor x \rfloor$),
  is the greatest integer less than or equal
  to $x$.
\end{definition}

\begin{example}
  We have the following:
  \begin{enumerate}
    \item If $a \in \Z$, then $[a] = a$.
      The converse is also true:
      If $[a] = a$ for $a \in \R$, then
      $a \in \Z$.
    \item We have
      $[\pi] = 3$, $[e] = 2$, $[-1.5] = -2$,
      and $[-\pi] = -4$.
  \end{enumerate}
\end{example}

\begin{lemma}\label{lem:floor-bound}
  Let $x \in \R$. Then $x - 1 < [x] \le x$.
\end{lemma}

\begin{proof}
  The upper bound is obvious. To show the
  lower bound, suppose to the contrary that
  $[x] \le x - 1$. Then $[x] < [x] + 1 \le x$,
  which contradicts the maximality of $[x]$
  as $[x] + 1$ is an integer.
\end{proof}

\begin{example}
  We can write $5 = 3 \cdot 1 + 2$ and
  $26 = 6 \cdot 4 + 2$; this is
  the \emph{division algorithm}.
\end{example}

\begin{theorem}[Division algorithm]
  Let $a, b \in \Z$ with $b > 0$. Then there
  exist unique $q, r \in \Z$ such
  that
  \[
    a = bq + r, \quad 0 \le r < b.
  \]
  Call $q$ the \emph{quotient} and
  $r$ the \emph{remainder} of the division.
\end{theorem}

\begin{proof}
  First we show existence. Let $q = [a / b]$
  and $r = a - b[a / b]$. By construction,
  $a = bq + r$. To check that
  $0 \le r < b$, note that by Lemma
  \ref{lem:floor-bound}, we have
  $a / b - 1 < [a / b] \le a / b$. Multiplying
  by $-b$ gives
  \[
    -a \le -b[a / b] < b - a,
  \]
  and adding $a$ gives the desired inequality
  $0 \le a - b[a / b] = r < b$.

  Now we prove uniqueness. Assume there are
  $q_1, q_2, r_1, r_2 \in \Z$ such that
  \[
    a = bq_1 + r_1 = bq_2 + r_2, \quad
    0 \le r_1, r_2 < b.
  \]
  Then $0 = (bq_1 + r_1) - (bq_2 + r_2) = b(q_1 - q_2) + (r_1 - r_2)$, so we find that
  \[
    r_2 - r_1 = b(q_1 - q_2).
  \]
  So $b \mid r_2 - r_1$. But
  $0 \le r_1, r_2 < b$
  implies $-b < r_2 - r_1 < b$, so
  we must have $r_2 - r_1 = 0$, i.e.
  $r_1 = r_2$. This then implies
  $0 = b(q_1 - q_2)$, which gives
  $q_1 - q_2 = 0$ since $b > 0$, so
  $q_1 = q_2$ as well.
\end{proof}

\begin{remark}
  In the division algorithm, we have $r = 0$
  if and only if $b \mid a$.
\end{remark}

\begin{example}
  Suppose $a = -5$, $b = 3$. Then
  $q = [a / b] = -2$ and
  $r = a - b[a / b] = 1$, i.e.
  \[
    -5 = 3 \cdot (-2) + 1.
  \]
  Note that $-5 = 3 \cdot (-1) + (-2)$ also, but
  this does not contradict uniqueness since
  $-2 \notin [0, 3)$.
\end{example}

\begin{definition}
  Let $n \in \Z$. Then $n$ is \emph{even}
  if $2 \mid n$, and $\emph{odd}$ otherwise.
\end{definition}

  \chapter{Aug.~20 --- Prime Numbers}

\begin{quote}
  \emph{Two fish are in a tank. One says to the other, "Ha, how do you drive this thing?"}
\end{quote}

\section{Prime Numbers}

\begin{definition}
  Let $p \in \Z$ with $p > 1$. Then $p$
  is \emph{prime} if the only positive
  divisors of $p$ and $1$ and $p$.
  If $n \in \Z$, $n > 1$ and $n$ is not
  prime, then $n$ is \emph{composite}.
\end{definition}

\begin{remark}
  The number $1$ is neither prime
  nor composite.
\end{remark}

\begin{example}
  The following are prime numbers:
  $2, 3, 5, 7, 11, 13, 17, 19, 23, 29, 31, 37, 41, 43, 47, \dots$.
\end{example}

\begin{lemma}\label{lem:has-prime-divisor}
  Every integer greater than $1$ has a
  prime divisor.
\end{lemma}

\begin{proof}
  Assume to the contrary that there exists
  $n > 1$ that has no prime divisor.
  By the well-ordering principle,\footnote{The \emph{well-ordering principle} says that every nonempty subset of the positive integers contains a least element.}
  we may take $n$ to be the smallest
  such positive integer. Since $n$
  has no prime divisors, $n$ cannot be
  prime. Thus $n$ has a divisor $a$
  with $1 < a < n$. Since $1 < a < n$,
  $a$ must have a prime divisor $p$ by the
  minimality of $n$. But then $p \mid a$
  and $a \mid n$, so $p \mid n$ by
  transitivity,
  a contradiction.
\end{proof}

\begin{theorem}[Euclid]
  There are infinitely many prime numbers.
\end{theorem}

\begin{proof}
  Assume to the contrary that there are
  only finitely many primes
  $p_1, p_2, \dots, p_n$.
  Consider
  \[
    N = p_1 p_2 \cdots p_n + 1.
  \]
  By Lemma \ref{lem:has-prime-divisor},
  $N$ has a prime divisor $p = p_j$
  for some $1 \le j \le n$. Since
  $p$ divides $N$ and $p$ divides
  $p_1 p_2 \cdots p_n$,
  $p$ also divides
  $N - p_1 p_2 \cdots p_n = 1$,
  which is a contradiction.
\end{proof}

\begin{exercise}
  Modify the proof and construct
  infinitely many problematic $N$.
\end{exercise}

\section{Sieve of Eratosthenes}

\begin{prop}
  If $n$ is composite, then $n$ has a
  prime divisor that is less than or
  equal to $\sqrt{n}$.
\end{prop}

\begin{proof}
  Since $n$ is composite, $n = ab$ where
  $1 < a, b < n$. Without loss of
  generality, assume $a \le b$. We claim
  $a \le \sqrt{n}$. To see this,
  suppose to the contrary that
  $a > \sqrt{n}$.
  Then $n = ab \ge a^2 > n$, a
  contradiction.
  By Lemma
  \ref{lem:has-prime-divisor},
  $a$ has a prime divisor
  $p \le a \le \sqrt{n}$. But then
  $p \mid a$ and $a \mid n$, so $p \mid n$.
\end{proof}

\begin{remark}
  The proposition implies that if
  all the prime divisors of an integer
  $n$ are greater than $\sqrt{n}$,
  then $n$ is prime. So to check the
  primality of $n$,
  it suffices to check divisibility
  by primes $\le \sqrt{n}$.
\end{remark}

\begin{example}
  The \emph{sieve of Eratosthenes} proceeds
  as follows. To find primes $\le 50$,
  we can delete multiples of primes
  $\le \sqrt{50} \approx 7.07$. To start,
  we know that $2$ is prime. Then cross
  out all multiples of $2$. The smallest
  number remaining is $3$, which we now
  know must be prime. Then cross out all
  multiples of $3$. Continue this process
  until we cross out all multiples of
  $7$, and then all remaining numbers
  are prime.
\end{example}

\section{Gaps in Primes}
\begin{prop}
  For any positive integer $n$, there are
  at least $n$ consecutive composite
  positive integers.
\end{prop}

\begin{proof}
  Consider the following list of $n$
  consecutive numbers:
  \[
    (n + 1)! + 2,\quad (n + 1)! + 3,\quad
    (n + 1)! + 4,\quad
    \dots,\quad (n + 1)! + (n + 1).
  \]
  Note that for any $2 \le m \le n + 1$,
  we have $m \mid m$ and $m \mid (n + 1)!$,
  so $m$ divides $(n + 1)! + m$.
  Thus each number in the above list
  is composite, so we have at least
  $n$ consecutive composite integers.
\end{proof}

\begin{remark}
  With some modifications to this proof
  (namely a more ``efficient''
  construction), one can find asymptotic
  lower bounds for the length of long
  prime gaps.
\end{remark}

\begin{conjecture}
  There are infinitely many pairs
  of primes that differ by exactly $2$.
\end{conjecture}

\begin{remark}
  Zhang (2013) was able to show that
  there are infinitely many pairs of
  pairs of primes whose difference is
  $\le 70,000,000$. This has been
  lowered to $246$ by the Polymath project,
  which included Tao and Maynard. Assuming
  other strong conjectures
  (Elliot-Halberstam), we can get
  down to $6$.
\end{remark}

\begin{remark}
  In addition to long and short prime gaps,
  we can also consider the average length
  of prime gaps. Gauss conjectured that
  as $x \to \infty$, the number of primes
  $\le x$, denoted $\pi(x)$, satisfies
  \[
    \pi(x) \sim \frac{x}{\log x},
  \]
  i.e. $\pi(x)$ is asymptotic
  to $x / {\log x}$. Said
  differently, this says that the
  ``probability'' that an integer $\le x$
  is prime is $\pi(x) / x \sim 1 / {\log x}$.
  This conjecture was proved independently
  in 1896 by de la Vall\'e-Poussin and
  Hadamard, and is now known as the
  \emph{prime number theorem}.
\end{remark}

\begin{definition}
  Let $x \in \R$. Define
  $\pi(x) = |{\{p : \text{$p$ prime}, p \le x\}}|$.
\end{definition}

\begin{theorem}[Prime number theorem]
  As $x \to \infty$,
  $\pi(x)$ is asymptotic to $x / {\log x}$,
  i.e.
  \[
    \lim_{x \to \infty}
    \frac{\pi(x)}{x / {\log x}} = 1.
  \]
\end{theorem}

\section{Other Open Problems}

\begin{conjecture}[Goldbach]
  Every even integer $\ge 4$ is a sum
  of two primes.
\end{conjecture}

\begin{theorem}[Ternary Goldbach]
  Every odd integer $\ge 7$ is a sum of
  three primes.
\end{theorem}

\begin{remark}
  Goldbach's conjecture implies
  ternary Goldbach (subtract $3$), but not
  vice versa.
\end{remark}

\begin{definition}
  Primes of the form $p = 2^n - 1$
  are called \emph{Mersenne primes}, and
  primes of the form $p = 2^{2^n} + 1$
  are called \emph{Fermat primes}.
\end{definition}

\begin{conjecture}
  There are infinitely many Mersenne
  primes but only finitely many
  Fermat primes.
\end{conjecture}

  \chapter{Aug.~25 --- Greatest Common Divisors}

\begin{quote}
  \emph{What do you call a root vegetable, fresh off the oven, and a pig that you throw off the balcony? One is a heated yam, and the other is a yeeted ham.}
\end{quote}

\section{Greatest Common Divisors}

\begin{remark}
  Given $a, b \in \Z$, not both zero,
  we can
  consider the set
  \[S = \{c \in \Z : c \mid a \text{ and } c \mid b\},\]
  of common divisors of both $a$ and $b$.
  Note that $\pm 1 \in S$, so $S$ is
  nonempty, and $S$ is also finite as
  at least one of $a, b$ is nonzero.
  Thus $S$ has a maximal element.
\end{remark}

\begin{definition}
  Let $a, b \in \Z$, not both zero. Then
  the \emph{greatest common divisor}
  of $a$ and $b$, denoted $(a, b)$,
  is the largest integer $d$ such that
  $d \mid a$ and $d \mid b$.
  If $(a, b) = 1$, then we say that
  $a, b$ are \emph{relatively prime} (or
  \emph{coprime}).
\end{definition}

\begin{remark}
  Note that $(0, 0)$ is not defined.
  Also note that if $(a, b) = d$, then
  \[
    (a, b) = (-a, b) = (a, -b)
    = (-a, -b) = d.
  \]
\end{remark}

\begin{example}
  We will compute $(24, 60)$. The
  list of positive divisors of $24$ and
  $60$ are
  \begin{align*}
    24 &: 1, 2, 3, 4, 6, 8, 12, 24; \\
    60 &: 1, 2, 3, 4, 5, 6, 10, 12, 15, 20, 30, 60.
  \end{align*}
  We can then see that $(24, 60) = 12$.
\end{example}

\begin{remark}
  In general, we have $(a, 0) = |a|$.
\end{remark}

\begin{prop}
  Let $(a, b) = d$. Then
  $(a / d, b / d) = 1$.
\end{prop}

\begin{proof}
  Let $d' = (a / d, b / d) > 0$. Then
  $d' \mid (a / d)$ and $d' \mid (b / d)$,
  so there exist $e, f$ such that
  $a / d = ed'$ and $b / d = fd'$.
  We can write this as $a = e d' d$ and
  $b = f d' d$. Thus
  $d' d$ is a common divisor of
  $a$ and $b$, so
  we must have $d' = 1$ by the maximality
  of $d$.
\end{proof}

\begin{prop}
  Let $a, b \in \Z$, not both zero,
and let \[T = \{ma + nb : m, n \in \Z, ma + nb > 0\}.\] Then
  $\min T$ exists and is equal to
  $(a, b)$.
\end{prop}

\begin{proof}
  Without loss of generality, we can
  assume $a \ne 0$. Note that
  $|a| \in T$, so $T$ is nonempty. Thus
  by the well-ordering principle, $T$
  has a minimal element $d$.
  Then $d = m' a + n' b$ for some
  $m', n' \in \Z$. We will show that
  $d \mid a$, a similar argument
  shows that $d \mid b$. By the
  division algorithm, we may write
  \[
    a = dq + r, \quad 0 \le r < d.
  \]
  It suffices to show that $r = 0$.
  We can rewrite the above as
  \[
    r
    = a - dq
    = a - (m' a + n' b)q
    = a(1 - m' q) - b(n' q).
  \]
  So $r$ is an integral linear combination
  of $a, b$. Since
  $d$ is the smallest positive
  integral linear combination of
  $a, b$ and $0 \le r < d$, we must
  have $r = 0$.
  So $d$ is a common divisor of $a, b$.

  Now suppose $c \mid a$ and $c \mid b$,
  then $c \mid (ma + nb)$, so
  $c$ divides $d = m' a + n' b$.
  Thus $c \le d$, so $d = (a, b)$.
\end{proof}

\begin{remark}
  If $(a, b) = d$, then $d = ma + nb$
  for some $m, n \in \Z$. If $d = 1$, then
  the converse also holds: If
  \[
    1 = ma + nb,
  \]
  and $d'$ is a common divisor of $a, b$,
  then $d' \mid 1$, so $d' = 1$.
\end{remark}

\begin{remark}
  Along the way, we showed that any
  common divisor of $a, b$ divides
  $(a, b)$.
\end{remark}

\begin{definition}
  Let $a_1, \dots, a_n \in \Z$, with at
  least one nonzero. Then the
  \emph{greatest common divisor}
  of $a_1, \dots, a_n$, denoted
  $(a_1, \dots, a_n)$, is the
  largest integer $d$ such that
  $d \mid a_i$ for $1 \le i \le n$.
  If $(a_1, \dots, a_n) = 1$, then we say
  that $a_1, \dots, a_n$ are
  \emph{relatively prime}, and if
  $(a_i, a_j) = 1$ for all $1\le i \ne j \le n$,
  then we say that $a_1, \dots, a_n$
  are \emph{pairwise relatively prime}.
\end{definition}

\begin{remark}
  Pairwise relatively prime implies
  relatively prime, but the converse
  is not true (e.g. $\{2, 4, 3\}$).
\end{remark}

\section{The Euclidean Algorithm}

\begin{lemma}
  If $a, b \in \Z$ with $0 < b \le a$ and
  $a = bq + r$ with $q, r \in \Z$, then
  $(a, b) = (r, b)$.
\end{lemma}

\begin{proof}
  It suffices to show that the two sets of
  common divisors (of $a, b$ and of $r, b$)
  are the same. Denote by $S_1$ and $S_2$
  these two sets, respectively. First let
  $c \in S_1$, so $c \mid a$ and $c \mid b$.
  We can write
  \[
    r = a - bq,
  \]
  so we have $c \mid r$. Thus
  $c \in S_2$, so $S_1 \subseteq S_2$.
  Now let $c \in S_2$, so $c \mid r$ and
  $c \mid b$. We have
  \[
    a = bq + r
  \]
  by hypothesis, so $c \mid a$, i.e.
  $c \in S_1$.
  Thus $S_1 = S_2$, so
  $(a, b) = \max S_1 = \max S_2 = (r, b)$.
\end{proof}

\begin{example}
  The above lemma allows us to compute
  greatest common divisors more efficiently.
  We will compute $(803, 154)$. We
  can write
  $803 = 5 \cdot 154 + 33$, so
  $(803, 154) = (154, 33)$.
  Continuing, we get
  \[
    (803, 154)
    = (154, 33)
    = (33, 22)
    = (22, 11)
    = (11, 0) = 11.
  \]
\end{example}

\begin{theorem}[Euclidean algorithm]
  Let $a, b \in \Z$ with $0 < b \le a$.
  Set $r_{-1} = a$, $r_0 = b$, and inductively
  write
  $r_{i - 1} = q_i r_i + r_{i + 1}$
  by the division algorithm for
  $n \ge 1$.
  Then $r_n = 0$ for some $n \ge 1$
  and $(a, b) = r_{n - 1}$.
\end{theorem}

\begin{proof}
  Note that $r_1 > r_2 > r_3 > \cdots$.
  If $r_n \ne 0$ for all $n \ge 1$, then
  this is a strictly decreasing infinite
  sequence of positive integers, which
  is not possible. So $r_n = 0$ for some
  $n \ge 1$. The conclusion
  $(a, b) = r_{n - 1}$ follows by repeatedly
  applying the lemma since
  $(a, b) = (r_i, r_{i + 1}) = (r_{n - 1}, 0) = r_{n - 1}$.
\end{proof}

\begin{example}
  By reversing this process, we can
  write $(a, b)$ explicitly as an integer
  linear combination of $a, b$.
  Using the previous example of computing
  $(803, 154)$, we can see that
  \begin{align*}
    (803, 154)
    &= 11
    = 33 - 1 \cdot 22 \\
    &= 33 - 1 \cdot (154 - 4 \cdot 33)
    = 5 \cdot 33 - 1 \cdot 154 \\
    &= 5 \cdot (803 - 5 \cdot 154) - 1 \cdot 154
    = 5 \cdot 803 - 26 \cdot 154.
  \end{align*}
  Thus we have found that
  $(803, 154) = 5 \cdot 803 - 26 \cdot 154$.
  Note that this representation
  is not unique, e.g.
  we can also write
  $11 = 19 \cdot 803 - 99 \cdot 154$.
  In fact, there are infinitely many such
  representations.
\end{example}

  \chapter{Aug.~27 --- Fundamental Theorem of Arithmetic}

\begin{quote}
  \emph{What's the difference between a mediocre clown and a rabbit in the gym? One's a bit funny, the other's a fit bunny.}
\end{quote}

\section{The Fundamental Theorem of Arithmetic}

\begin{lemma}[Euclid]\label{lem:prime-divides-product}
  Let $a, b \in \Z$ and let $p$ be a prime.
  If $p \mid ab$, then $p \mid a$
  or $p \mid b$.
\end{lemma}

\begin{proof}
  If $p \mid a$, then we are done, so
  assume $p \nmid a$. Then
  $(p, a) = 1$. Thus we can write
  $1 = ma + np$ for some $m, n \in \Z$.
  Since $p \mid ab$, we can write
  $ab = pc$ for some $c \in \Z$.
  Multiplying by $b$, we have
  \[
    b = bma + bnp
    = m(cp) + nb p
    = p(mc + nb).
  \]
  Thus we see that
  $p \mid b$, as desired.
\end{proof}

\begin{remark}
  This fails if $p$ is composite:
  Take $p = 6$, $a = 2$, and $b = 3$.
\end{remark}

\begin{exercise}
  Determine where the proof fails
  if $p$ is composite.
\end{exercise}

\begin{corollary}
  Let $a_1, \dots, a_n \in \Z$ and $p$
  a prime. If $p \mid a_1 \cdots a_n$,
  then $p \mid a_i$ for some $1 \le i \le n$.
\end{corollary}

\begin{proof}
  Induct on $n$. The base case $n = 1$
  is trivial. If $n = 2$, then this is
  just Lemma \ref{lem:prime-divides-product}.
  Now suppose $n \ge 2$, and we show the
  result for $n + 1$. Specifically,
  assume that if $p \mid a_1 \cdots a_n$,
  then $p \mid a_i$ for some
  $1 \le i \le n$. Suppose
  $p \mid a_1 \cdots a_n a_{n+1}$.
  Then $p \mid (a_1 \cdots a_n) a_{n+1}$.
  So by Lemma \ref{lem:prime-divides-product},
  we have $p \mid a_1 \cdots a_n$
  or $p \mid a_{n + 1}$. If $p \mid a_{n + 1}$, then we are done.
  Otherwise, $p \mid a_1 \cdots a_n$,
  so $p \mid a_i$ for some $1 \le i \le n$
  by the induction hypothesis.
  In particular, $p \mid a_i$ for some
  $1 \le i \le n + 1$, as desired.
\end{proof}

\begin{theorem}[Fundamental theorem of arithmetic]
  Every integer $m > 1$ may be
  expressed in the form
  $m = p_1^{a_1} \cdots p_n^{a_n}$
  where $p_1, \dots, p_n$ are distinct
  primes and $a_1, \dots, a_n$ are
  positive integers. This form is
  called the \emph{prime factorization}
  of the integer $m$. Moreover, this
  factorization is essentially unique, i.e.
  unique up to permutations
  of the factors $p_i^{a_i}$.
\end{theorem}

\begin{proof}
  We first prove existence.
  Assume to the contrary that there exists
  $m > 1$ that does not have a prime
  factorization. Without loss of generality,
  we can assume $m$ is the smallest such
  integer by the well-ordering principle.
  In particular, $m$ cannot be prime.
  So $m = ab$ for some $1 < a, b < m$. Then
  $a, b$ have prime factorizations. Thus
  so too does $m$, a contradiction.

  Now we prove uniqueness. Assume
  that $m = p_1^{a_1} \cdots p_n^{a_n} = q_1^{b_1} \cdots q_r^{b_r}$.
  Without loss of generality, we
  can assume $p_1 < p_2 < \cdots < p_n$
  and $q_1 < q_2 < \cdots < q_r$.
  We need to show that
  $n = r$, $p_i = q_i$ for each $i$, and
  $a_i = b_i$ for each $i$. Let $p_i \mid m$.
  Then $p_i \mid q_1^{b_1} \cdots q_r^{b_r}$,
  so $p_i \mid q_j$ for some $1 \le j \le r$.
  Thus $p_i = q_j$ since both are prime.
  Similarly, given $q_i$, we have
  $q_i = p_j$ for some $j$. Thus the
  primes in the two factorizations (as
  sets) are the same. Thus $n = r$, and
  by the ordering assumption, we have
  $p_i = q_i$ for each $1 \le i \le n$.
  So
  \[
    m = p_1^{a_1} \cdots p_n^{a_n}
    = p_1^{b_1} \cdots p_n^{b_n}.
  \]
  Suppose to the contrary that
  $a_i \ne b_i$ for some $i$. Without
  loss of generality, assume $a_i < b_i$.
  We have $p_i^{b_i} \mid m$, so
  $p_i^{b_i} \mid p_1^{a_1} \cdots p_{i - 1}^{a_{i - 1}} p_i^{a_i} p_{i + 1}^{a_{i + 1}} \cdots p_n^{a_n}$.
  Thus $p_i^{b_i - a_i} \mid p_1^{a_1} \cdots p_{i - 1}^{a_{i - 1}} p_{i + 1}^{a_{i + 1}} \cdots p_n^{a_n}$.
  Since $a_i < b_i$, we have
  $b_i - a_i > 0$, so
  $p_i \mid p_1^{a_1} \cdots p_{i - 1}^{a_{i - 1}} p_{i + 1}^{a_{i + 1}} \cdots p_n^{a_n}$
  by the transitivity of divisibility.
  Then $p_i \mid p_j$ for some $j \ne i$,
  so $p_i = p_j$, which is a contradiction
  since the $p_i$ are all distinct primes.
  This proves uniqueness.
\end{proof}

\begin{remark}
  This is one reason why we do not
  consider $1$ to be a prime, as we would
  lose uniqueness.
\end{remark}

\begin{example}
  We can write $60 = 2^2 \cdot 3 \cdot 5$
  and $756 = 2^2 \cdot 3^3 \cdot 7$.
\end{example}

\section{Least Common Multiples}

\begin{definition}
  Let $a, b \in \Z$ with $a, b > 0$.
  The \emph{least common multiple}
  of $a$ and $b$, denoted $[a, b]$, is the
  least positive integer $m$ such that
  $a \mid m$ and $b \mid m$.
\end{definition}

\begin{remark}
  Since $ab$ is a common multiple
  of $a$ and $b$, $[a, b]$ always
  exists by the well-ordering principle.
\end{remark}

\begin{example}
  We will compute $[6, 7]$. The multiples
  of $6$ and $7$ include:
  \begin{align*}
    6 &: 6, 12, 18, 24, 30, 36, 42, 48, \dots; \\
    7 &: 7, 14, 21, 28, 35, 42, 49, \dots.
  \end{align*}
  So we can see that $[6, 7] = 42 = 6 \cdot 7$.
  On the other hand, $[6, 8] = 24 \ne 6 \cdot 8$.
\end{example}

\begin{remark}
  The fundamental theorem of arithmetic
  can be used to calculate both GCDs and
  LCMs.
\end{remark}

\begin{prop}\label{prop:gcd-lcm-prime-factorization}
  Let $a, b \in \Z$ with $a, b > 1$.
  Write $a = p_1^{a_1} \cdots p_n^{a_n}$
  and $b = p_1^{b_1} \cdots p_n^{b_n}$,
  where the $p_i$ are distinct primes,
  and $a_i, b_i \ge 0$. Then we have
  \[
    (a, b) = p_1^{\min\{a_1, b_1\}} \cdots p_n^{\min\{a_n, b_n\}} \quad \text{and} \quad
    [a, b] = p_1^{\max\{a_1, b_1\}} \cdots p_n^{\max\{a_n, b_n\}}.
  \]
\end{prop}

\begin{proof}
  Left as an exercise.
\end{proof}

\begin{example}
  Calculate $(756, 2205)$ and
  $[756, 2205]$. We can write
  \[
    756 = 2^2 \cdot 3^3 \cdot 5^0 \cdot 7^1
    \quad \text{and} \quad
    2205 = 2^0 \cdot 3^2 \cdot 5^1 \cdot 7^2.
  \]
  So we have
  $(756, 2205) = 2^0 \cdot 3^2 \cdot 5^0 \cdot 7^1 = 63$ and
  $[756, 2205] = 2^2 \cdot 3^3 \cdot 5 \cdot 7^2 = 26460$.
\end{example}

\begin{lemma}\label{lem:min-max-sum}
  Given $x, y \in \R$, we have
  $\min\{x, y\} + \max\{x, y\} = x + y$.
\end{lemma}

\begin{proof}
  The result is obvious if $x = y$.
  Otherwise, one is the minimum and the
  other is the maximum.
\end{proof}

\begin{theorem}
  Let $a, b \in \Z$ with $a, b > 1$. Then
  $(a, b) [a, b] = ab$.
\end{theorem}

\begin{proof}
  Write $a = p_1^{a_1} \cdots p_n^{a_n}$
  and $b = p_1^{b_1} \cdots p_n^{b_n}$
  with $a_i, b_i \ge 0$ and $p_i$
  distinct. By Proposition
  \ref{prop:gcd-lcm-prime-factorization},
  \begin{align*}
    (a, b) [a, b]
    &= p_1^{\min\{a_1, b_1\}} \cdots p_n^{\min\{a_n, b_n\}}
      p_1^{\max\{a_1, b_1\}} \cdots p_n^{\max\{a_n, b_n\}} \\
    &= p_1^{\min\{a_1, b_1\} + \max\{a_1, b_1\}} \cdots
      p_n^{\min\{a_n, b_n\} + \max\{a_n, b_n\}}
    = p_1^{a_1 + b_1} \cdots p_n^{a_n + b_n}
    = ab,
  \end{align*}
  where the third equality follows
  from Lemma \ref{lem:min-max-sum}.
\end{proof}

  \chapter{Sept.~3 --- Congruences}

\begin{quote}
  \emph{No, Tony's the guy with no shins.}
\end{quote}

\section{Dirichlet's Theorem}

\begin{theorem}[Dirichlet's theorem on primes in arithmetic progressions]
  Let $a, b \in \Z$ with
  $a, b > 0$ and $(a, b) = 1$. Then
  the arithmetic progression
  $a, a + b, a + 2b, a + 3b, \dots$
  contains infinitely many primes.
\end{theorem}

\begin{remark}
  Setting $a = b = 1$ recovers the
  fact that there are infinitely many
  primes.
\end{remark}

\begin{remark}
  The general case of Dirichlet's theorem
  is difficult, but we can use the
  fundamental theorem of arithmetic
  to prove some special cases, e.g.
  when $a = 3$ and $b = 4$.
\end{remark}

\begin{lemma}\label{lem:4n+1-product}
  Let $a, b \in \Z$. If $a$ and $b$ are
  expressible as $4n + 1$, then so is
  their product $ab$.
\end{lemma}

\begin{proof}
  Let $a = 4m + 1$ and $b = 4n + 1$. Then
  \[
    ab = (4m + 1)(4n + 1)
    = 16mn + 4m + 4n + 1
    = 4(4mn + m + n) + 1,
  \]
  which proves the desired result.
\end{proof}

\begin{prop}
  There are infinitely many primes
  of the form $4n + 3$ with $n \ge 0$.
\end{prop}

\begin{proof}
  Assume to the contrary that there are
  finitely many primes of the form
  $4n + 3$, say $3, p_1, \dots, p_r$.
  Then consider the integer
  $N = 4p_1 \cdots p_r + 3$. The prime
  factorization of $N$ must contain a
  prime of the form $4n + 3$, since
  otherwise $N$ would be a product of
  primes of the form $4n + 1$, which
  must again be of the form $4n + 1$.
  Thus we have $3 \mid N$ or $p_i \mid N$
  for some $1 \le i \le r$.

  If $3 \mid N$, then $3 \mid N - 3 = 4p_1 \dots p_r$,
  which is a contradiction.
  Otherwise, $p_i \mid N$ for some
  $1 \le i \le r$, and we have
  $p_i \mid N - 4p_1 \cdots p_r = 3$,
  which is a contradiction as well.
\end{proof}

\begin{remark}
  The same proof does not work for
  primes of the form $4n + 1$, since a
  product of numbers of the form
  $4n + 3$ is not necessarily again
  of the form $4n + 3$.
\end{remark}

\section{Congruences}

\begin{definition}
  Let $a, b, m \in \Z$ with $m > 0$.
  Then we say that $a$ is \emph{congruent
  to $b$ modulo $m$}, and we write
  $a \equiv b \Pmod{m}$,
  if $m \mid (a - b)$. The integer
  $m$ is called the \emph{modulus}
  of the congruence. We write
  $a \not\equiv b \Pmod{m}$ if
  $a$ is not congruent to $b$ modulo $m$.
\end{definition}

\begin{example}
  We have $25 \equiv 1 \Pmod{4}$ and
  $25 \equiv 4 \Pmod{7}$.
\end{example}

\begin{prop}\label{prop:congruence-equiv}
  Congruence modulo $m$ is
  an equivalence relation on $\Z$.
\end{prop}

\begin{proof}
  Reflexivity is clear since
  $m \mid 0 = (a - a)$ any $a \in \Z$,
  so $a \equiv a \Pmod{m}$.
  For symmetry, suppose that
  $a \equiv b \Pmod{m}$. Then
  $m \mid a - b$. But then $m \mid (-1)(a - b) = b - a$, so
  $b \equiv a \Pmod{m}$ as well.

  Finally, for transitivity, suppose that
  $a \equiv b \Pmod{m}$ and
  $b \equiv c \Pmod{m}$. Then
  $m \mid a - b$ and $m \mid b - c$, so
  $m$ also divides their sum
  $m \mid (a - b) + (b - c) = a - c$, i.e.
  $a \equiv c \Pmod{m}$.
\end{proof}

\begin{remark}
  A consequence of Proposition
  \ref{prop:congruence-equiv}
  is that $\Z$ is partitioned into
  its equivalence classes under congruence
  modulo $m$. For $a \in \Z$, we
  write $[a]$ to denote the equivalence
  class of $a$ modulo $m$ (not to be
  confused with the floor function).
\end{remark}

\begin{example}
  The equivalence classes of $\Z$
  under congruence modulo $4$ are
  \begin{align*}
    [0]
    &= \{\dots, -8, -4, 0, 4, 8, \dots\}, \\
    [1]
    &= \{\dots, -7, -3, 1, 5, 9, \dots\}, \\
    [2]
    &= \{\dots, -6, -2, 2, 6, 10, \dots\}, \\
    [3]
    &= \{\dots, -5, -1, 3, 7, 11, \dots\}.
  \end{align*}
\end{example}

\begin{definition}
  A set of $m$ integers such that every
  integer is congruent modulo $m$ to
  exactly one integer of the set is
  called a \emph{complete residue system}
  modulo $m$.
\end{definition}

\begin{example}
  $\{0, 1, 2, 3\}$ is a complete
  residue system modulo $4$.
  So is $\{4, 5, -6, -1\}$.
\end{example}

\begin{prop}
  The set $\{0, 1, \dots, m - 1\}$
  is a complete residue system
  modulo $m$.
\end{prop}

\begin{proof}
  First we prove that every integer
  is congruent to one of
  $0, 1, \dots, m - 1$ modulo $m$.
  By the division algorithm, for any
  $a \in \Z$, there exist $q, r \in \Z$
  with $0 \le r \le m - 1$ such that
  $a = qm + r$. Thus we have
  $a - r = qm$, so $m \mid a - r$, i.e.
  $a \equiv r \Pmod{m}$. This proves
  existence since $r \in \{0, 1, \dots, m - 1\}$.

  Now we show uniqueness. Suppose
  $a \equiv r_1 \Pmod{m}$ and
  $a \equiv r_2 \Pmod{m}$ where
  $r_1, r_2 \in \{0, 1, \dots, m - 1\}$.
  By transitivity, we have
  $r_1 \equiv r_2 \Pmod{m}$, so
  $m \mid r_1 - r_2$. But
  $0 \le r_1, r_2 \le m - 1$, so
  \[
    -(m - 1) \le r_1 - r_2 \le m - 1,
  \]
  so we must have $r_1 - r_2 = 0$, i.e.
  $r_1 = r_2$. This proves uniqueness.
\end{proof}

\begin{definition}
  The set $\{0, 1, \dots, m - 1\}$ is
  called the set of \emph{least nonnegative residues}
  modulo $m$.
\end{definition}

\begin{prop}
  Let $a, b, c, d, m \in \Z$, $m > 0$
  such that $a \equiv b \Pmod{m}$
  and $c \equiv d \Pmod{m}$. Then
  \begin{enumerate}
    \item $a + c \equiv b + d \Pmod{m}$;
    \item $ac \equiv bd \Pmod{m}$.
  \end{enumerate}
\end{prop}

\begin{proof}
  Since $a \equiv b \Pmod{m}$ and
  $c \equiv d \Pmod{m}$, we have
  $m \mid b - a$ and $m \mid d - c$.
  Then $m$ divides
  \[
    (b - a) + (d - c)
    = (b + d) - (a + c),
  \]
  so we have $a + c \equiv b + d \Pmod{m}$.
  This proves (1).

  To prove (2), note that
  since $m \mid a - b$, we also have
  $m \mid c(a - b)$. Likewise,
  $m \mid d - c$ implies $m \mid b(d - c)$.
  Then $m$ divides the difference
  \[
    c(a - b) - b(d - c)
    = ac - bd,
  \]
  which shows that $ac \equiv bd \Pmod{m}$.
  This shows (2).
\end{proof}

\begin{remark}
  This shows that the congruence classes
  of $\Z$ modulo $m$ form a \emph{ring}.
\end{remark}

\begin{example}
  Consider the complete residue system
  $\{0, 1, 2, 3\}$ modulo $4$. Their
  squares mod $4$ are
  \[
    \{0^2, 1^2, 2^2, 3^2\}
    \equiv \{0, 1, 0, 1\}
    \equiv \{0, 1\} \pmod{4}.
  \]
\end{example}

  \chapter{Sept.~8 --- Congruences, Part 2}

\section{More on Congruences}
\begin{example}
  Compute a complete residue system
  modulo $5$ using
  \begin{itemize}
    \item only even numbers:
      $\{0, 2, 4, 6, 8\}$,
    \item only prime numbers:
      $\{2, 3, 5, 11, 19\}$.
  \end{itemize}
\end{example}

\begin{example}
  Compute a complete residue system
  modulo $5$ using only numbers
  $\equiv 1 \Pmod{4}$.
\end{example}

\begin{remark}
  Recall that the set of equivalence
  classes of $\Z$ modulo $m$ form a ring.
  In particular, we can construct
  addition and multiplication tables.
  For $m = 4$, this looks like:
  \begin{center}
    \begin{tabular}{c|cccc}
      + & 0 & 1 & 2 & 3 \\
      \hline
      0 & 0 & 1 & 2 & 3 \\
      1 & 1 & 2 & 3 & 0 \\
      2 & 2 & 3 & 0 & 1 \\
      3 & 3 & 0 & 1 & 2
    \end{tabular}
    \quad \quad \quad
    \begin{tabular}{c|cccc}
      $\times$ & 0 & 1 & 2 & 3 \\
      \hline
      0 & 0 & 0 & 0 & 0 \\
      1 & 0 & 1 & 2 & 3 \\
      2 & 0 & 2 & 0 & 2 \\
      3 & 0 & 3 & 2 & 1
    \end{tabular}
  \end{center}
  Addition modulo $5$ is similar, but
  the multiplication table for $m = 5$ is:
  \begin{center}
    \begin{tabular}{c|ccccc}
      $\times$ & 0 & 1 & 2 & 3 & 4 \\
      \hline
      0 & 0 & 0 & 0 & 0 & 0 \\
      1 & 0 & 1 & 2 & 3 & 4 \\
      2 & 0 & 2 & 4 & 1 & 3 \\
      3 & 0 & 3 & 1 & 4 & 2 \\
      4 & 0 & 4 & 3 & 2 & 1
    \end{tabular}
  \end{center}
  Recall that a ring with no zero divisors
  (nonzero elements $a, b$ such that $ab = 0$)
  is an \emph{integral domain}, in
  particular we see from the multiplication
  table that $\Z / 5\Z$ is an integral
  domain. Since a finite integral domain
  is automatically a \emph{field}, we
  see that $\Z / 5\Z$ is a field.
\end{remark}

\begin{prop}
  Let $a, b, c, m, \in \Z$ with $m > 0$.
  Then
  \[
    ca \equiv cb \Pmod{m}
    \quad \text{if and only if} \quad
    a \equiv b \Pmod{m / (m, c)}.
  \]
  In particular, if $m$ is prime, then
  $ca \equiv cb \Pmod{m}$
  if and only if $a \equiv b \Pmod{m}$
  for $c \not\equiv 0 \Pmod{m}$.
\end{prop}

\begin{proof}
  $(\Rightarrow)$ We have
  $ca \equiv cb \Pmod{m}$ if and only if
  $m \mid ca - cb = c(a - b)$. Let
  $d = (m, c)$. By the transitivity of
  divisibility, we have
  $(m / d) \mid (c / d)(a - b)$.
  But $(m / d, c / d) = 1$, so
  $(m / d) \mid a - b$. Then we
  have $a \equiv b \Pmod{m / d}$
  by the definition of congruence.

  $(\Leftarrow)$ Again let $d = (m, c)$.
  Then $a \equiv b \Pmod{m / d}$, so
  $(m / d) \mid a - b$. Then
  $m \mid d(a - b)$, and so
  \[
    m \mid d(a - b)(c / d)
    = c(a - b) = ca - cb,
  \]
  which means $ca \equiv cb \Pmod{m}$
  by the definition of congruence.
\end{proof}

\begin{remark}
  This shows that the congruence
  classes modulo $m$ form a field if
  and only if $m$ is prime.
\end{remark}

\section{Linear Congruences in One Variable}

\begin{definition}
  Let $a, b \in \Z$. A congruence of
  the form
  \[
    ax \equiv b \Pmod{m}
  \]
  is called a \emph{linear congruence} in
  the variable $x$.
\end{definition}

\begin{example}
  Consider the following linear congruences:
  \begin{itemize}
    \item $2x \equiv 3 \Pmod{4}$
      has no solutions;
    \item $2x \equiv 4 \Pmod{6}$
      has $x = 2, 5$ as solutions;
    \item $3x \equiv 9 \Pmod{6}$ has
      $x = 1, 3, 5$ as solutions.
  \end{itemize}
\end{example}

\begin{theorem}
  Let $ax \equiv b \Pmod{m}$, and
  let $d = (a, m)$. If $d \nmid b$, then
  there are no solutions for $x$ in $\Z$. If
  $d \mid b$, then the congruence
  has exactly $d$ incongruent solutions
  modulo $m$ in $\Z$.
\end{theorem}

\begin{proof}
  Note that $ax \equiv b \Pmod{m}$ if
  and only if $m \mid ax - b$, if
  and only if $ax - b = my$ for some
  integer $y$. This is equivalent to
  $ax - my = b$. Thus $ax \equiv b \Pmod{m}$
  is solvable in $x$ if and only if
  the equation $ax - my = b$ is solvable
  in $x, y$.

  Let $x, y$ be a solution of
  $ax - my = b$. Since $d \mid a$ and
  $d \mid m$, we must have $d \mid b$.
  Taking contrapositives, this
  proves the first part of the theorem.

  Assume now that $d \mid b$. We prove
  the second part in 4 steps:
  \begin{enumerate}
    \item We will show that $ax \equiv b \Pmod{m}$
      has a solution $x_0$.
    \item We will show that there are
      infinitely many solutions
      of a particular form involving $x_0$.
    \item We will show that
      any solution has a particular
      form involving $x_0$. (Note that
      this combines with $(2)$ to give
      all possible solutions.)
    \item We will show that there are
      exactly $d$ equivalence classes
      of solutions.
  \end{enumerate}

  $(1)$ Since $d = (a, m)$, there exist
  $r, s \in \Z$ such that $d = ra + sm$.
  Since $d \mid b$, we can write
  \[
    b = \frac{b}{d} \cdot d
    = \frac{b}{d}(ra + sm)
    = \frac{br}{d} \cdot a + \frac{bs}{d} \cdot m.
  \]
  Thus $b - a(b r/ d) = (b s/ d) m$,
  so $m \mid b - a(b r/ d)$, so
  $a(br / d) \equiv b \Pmod{m}$.
  Thus $x_0 = br / d$ is a solution.

  $(2)$ Let $x_0$ be any solution of
  $ax \equiv b \Pmod{m}$. Consider
  $x_0 + (m / d) n$ for $n \in \Z$. Then
  \[
    a (x_0 + (m / d) n)
    \equiv a x_0 + a (m / d) n
    \equiv b + (a / d)mn
    \equiv b \pmod{m},
  \]
  so $x_0 + (m / d) n$ is also solution
  for any $n \in \Z$.

  $(3)$ Let $x_0$ be a solution of
  $ax \equiv b \Pmod{m}$. Recall from the
  beginning of the proof that this is
  equivalent to there being
  $y_0 \in \Z$ such that
  $ax_0 - my_0 = b$. Let $x$ be any other
  solution. Then $ax - my = b$
  for some $y \in \Z$, so
  \[
    0 = b - b = (ax_0 - my_0) - (ax - my)
    = a(x_0 - x) - m(y_0 - y),
  \]
  which gives $a(x_0 - x) = m(y_0 - y)$.
  This is equivalent to
  $(a / d)(x_0 - x) = (m / d)(y_0 - y)$.
  Note that if $y_0 - y = 0$, then
  $x_0 - x = 0$ as well since
  $a / d \ne 0$. So we may assume
  $y_0 - y \ne 0$. Then
  \[(m / d) \mid (a / d)(x_0 - x),\]
  and since $(a / d, m / d) = 1$, we have
  $(m / d) \mid (x_0 - x)$. Thus
  $x \equiv x_0 \Pmod{m / d}$. In
  particular, all solutions to
  $ax \equiv b \Pmod{m}$ are given by
  $x = x_0 + (m / d) n$ for
  $n \in \Z$ and any particular
  solution $x_0$.

  $(4)$ Let $x_0 + (m / d) n_1$ and
  $x_0 + (m / d) n_2$ be solutions.
  Then we have
  \[
    x_0 + (m / d) n_1
    \equiv x_0 + (m / d) n_2 \Pmod{m}
  \]
  if and only if
  $(m / d) n_1 \equiv (m / d) n_2 \Pmod{m}$.
  This happens if and only if
  $m \mid (m / d)(n_1 - n_2)$, if and only if
  $(m / d) (n_1 - n_2) = km$ for some
  $K \in \Z$, if and only if
  $n_1 - n_2 = kd$. In particular, this
  is equivalent to $n_1 \equiv n_2 \Pmod{d}$.
  Since there are exactly $d$ congruence
  classes for $n$, there are exactly $d$
  congruence classes of solutions as well,
  which completes the proof.
\end{proof}

  \chapter{Sept.~10 --- Chinese Remainder Theorem}

\section{More on Linear Congruences}

\begin{corollary}
  Consider the linear congruence
  $ax \equiv b \Pmod{m}$ and let
  $d = (a, m)$. If $d \mid b$, then
  there are exactly $d$ incongruent
  solutions modulo $m$, given by
  \[
    x = x_0 + \frac{m}{d} \cdot n,
    \quad n = 0, 1, \dots, d - 1
  \]
  where $x_0$ is any particular solution.
\end{corollary}

\begin{example}
  We solve $16 x \equiv 8 \Pmod{28}$. We
  compute
  $d = (16, 28)$ by the Euclidean algorithm:
  \begin{align*}
    28 &= 1 \cdot 16 + 12 \\
    16 &= 1 \cdot 12 + 4 \\
    12 &= 3 \cdot 4 + 0.
  \end{align*}
  So $d = 4$. Since $4 \mid 8$, the
  congruence has $4$ incongruent solutions.
  Working backwards, we have
  \[
    4 = 2 \cdot 16 + (-1) \cdot 28.
  \]
  Multiplying by $2$, we get that
  $8 = 4 \cdot 16 + (-2) \cdot 28$.
  Taking this equation modulo $28$, we get
  \[
    16 \cdot 4 \equiv 8 \Pmod{28},
  \]
  so $x_0 = 4$ is a particular solution.
  Thus all the incongruent solutions
  are given by
  $x = 4 + (28 / 4) n$ for
  $n = 0, 1, 2, 3$, that is
  $x = 4, 11, 18, 25$.
\end{example}

\begin{definition}
  Any solution of $ax \equiv 1 \Pmod{m}$
  is called the \emph{multiplicative inverse}
  of $a$ modulo $m$. The multiplicative
  inverse of $a$ is often denoted
  $\overline{a}$.
\end{definition}

\begin{corollary}
  The congruence $ax \equiv 1 \Pmod{m}$
  has a solution if and only if
  $(a, m) = 1$. In this case, the
  congruence has a unique solution.
  In particular, the multiplicative
  inverse, if it exists, is unique.
\end{corollary}

\section{The Chinese Remainder Theorem}

\begin{example}\label{ex:crt-example}
  Consider the following problem:
  Find a positive integer having
  remainder $2$ when divided by $3$,
  remainder $1$ when divided by $4$,
  and remainder $3$ when divided by $5$.
  The problem can be rephrased as
  asking for a solution to the system
  of congruences:
  \[
    \begin{cases}
      x \equiv 2 \Pmod{3} \\
      x \equiv 1 \Pmod{4} \\
      x \equiv 3 \Pmod{5}.
    \end{cases}
  \]
\end{example}

\begin{theorem}[Chinese remainder theorem]
  Let $m_1, \dots, m_n$ be pairwise
  relatively prime positive integers,
  and let $b_1, \dots, b_n \in \Z$.
  Then the system of congruences
  \[
    \begin{cases}
      x \equiv b_1 \Pmod{m_1} \\
      x \equiv b_2 \Pmod{m_2} \\
      \quad \vdots \\
      x \equiv b_n \Pmod{m_n}
    \end{cases}
  \]
  has a unique solution modulo
  $M = m_1 \cdots m_n$.
\end{theorem}

\begin{proof}
  Let $M = m_1 \cdots m_n$ and
  $M_i = M / m_i$. Then
  $(M_i, m_i) = 1$, so there are solutions
  to each system
  $M_i x_i \equiv 1 \Pmod{m_i}$ given
  by $x_i = \overline{M}_i$. Consider
  \[
    x = b_1 M_1 \overline{M}_1
    + b_2 M_2 \overline{M}_2
    + \cdots
    + b_n M_n \overline{M}_n.
  \]
  Note that $m_i \mid M_j$ for
  $i \ne j$, so
  $x \equiv b_i M_i \overline{M}_i \equiv b_i \Pmod{m_i}$,
  so $x$ is a solution to the system.

  For uniqueness modulo $M$, let
  $x'$ be another solution. Then
  $x' \equiv b_i \Pmod{m_i}$ for
  each $1 \le i \le n$. Then
  \[
    x \equiv x' \Pmod{m_i}, \quad
    1 \le i \le n.
  \]
  Thus $m_i \mid x - x'$, so
  $M \mid x - x'$ since the $m_i$
  are pairwise relatively prime,
  so $x \equiv x' \Pmod{M}$.
\end{proof}

\begin{example}
  We now solve Example
  \ref{ex:crt-example}. Using the
  notation in the proof, we have
  \[
    (m_1, m_2, m_3) = (3, 4, 5), \quad
    (b_1, b_2, b_3) = (2, 1, 3), \quad
    M = 60,
    \quad
    (M_1, M_2, M_3) = (20, 15, 12).
  \]
  We still need to compute
  $\overline{M}_i$. In general, this
  can be done via the Euclidean algorithm.
  In this case,
  \[
    (\overline{M}_1, \overline{M}_2, \overline{M}_3)
    = (2, 3, 3).
  \]
  Now we can calculate the solution
  using
  \[
    x = b_1 M_1 \overline{M}_1
    + b_2 M_2 \overline{M}_2
    + b_3 M_3 \overline{M}_3
    = (2 \cdot 20 \cdot 2)
    + (1 \cdot 15 \cdot 3)
    + (3 \cdot 12 \cdot 3)
    = 233.
  \]
  Reducing modulo $60$, we get that the
  unique solution is given by
  $x \equiv 53 \Pmod{60}$.
\end{example}

\section{Wilson's Theorem}

\begin{lemma}\label{lem:self-inverse}
  Let $p$ be a prime and let
  $a \in \Z$. Then
  $a$ is its own inverse modulo $p$
  (i.e., $a \equiv \overline{a} \Pmod{p}$)
  if and only if $a \equiv \pm 1 \Pmod{p}$.
\end{lemma}

\begin{proof}
  $(\Rightarrow)$ Suppose
  $a \equiv \overline{a} \Pmod{p}$.
  Then $a^2 \equiv a \overline{a} \equiv 1 \Pmod{p}$, so
  $p \mid a^2 - 1 = (a - 1)(a + 1)$.
  Since $p$ is prime, we have
  $p \mid a - 1$ or $p \mid a + 1$, so
  $a \equiv \pm 1 \Pmod{p}$.

  $(\Leftarrow)$ This is obvious since
  $(\pm 1)^2 = 1$ in $\Z$, so they
  are also equal after reducing modulo $p$.
\end{proof}

\begin{theorem}[Wilson's theorem]\label{thm:wilson}
  Let $p$ be a prime. Then
  $(p - 1)! \equiv -1 \Pmod{p}$.
\end{theorem}

\begin{example}
  The idea behind the proof is the
  following: Concretely, if $p = 11$, we
  have
  \[
    (11 - 1)! =
    10 \cdot 9 \cdot 8 \cdot 7 \cdot 6 \cdot
    5 \cdot 4 \cdot 3 \cdot 2 \cdot 1
    \pmod{11}
  \]
  By Lemma \ref{lem:self-inverse},
  $10$ and $1$ are their own inverses
  modulo $11$.
  For each other integer $2 \le n \le 9$,
  we can pair them with their inverses:
  $(2, 6)$, $(3, 4)$, $(5, 9)$, $(7, 8)$.
  Then we can write
  \[
    (11 - 1)!
    \equiv (9 \cdot 5) \cdot (8 \cdot 7)
    \cdot (6 \cdot 2) \cdot (4 \cdot 3)
    \cdot 10 \cdot 1
    \equiv 10 \cdot 1 \equiv -1 \pmod{11}.
  \]
\end{example}

\begin{proof}[Proof of Theorem \ref{thm:wilson}]
  We can easily check the theorem
  for $p = 2, 3$, so suppose
  $p > 3$ is a prime. Then each $a$
  with $1 \le a \le p - 1$ has a unique
  inverse modulo $p$, and this inverse
  is distinct from $a$ if
  $2 \le a \le p - 2$. Pair each such
  integer with its inverse modulo $p$,
  say $a$ and $a'$. The product of
  all of these pairs is
  $(p - 2)!$, so
  $(p - 2)! \equiv 1 \Pmod{p}$.
  Thus $(p - 1)! \equiv p - 1 \equiv -1 \Pmod{p}$.
\end{proof}

\begin{prop}[Converse of Wilson's theorem]
  Let $n \in \Z$ with $n > 1$. If
  $(n - 1)! \equiv -1 \Pmod{n}$, then
  $n$ is prime.
\end{prop}

\begin{proof}
  Suppose $n = ab$ with $1 \le a < n$.
  It suffices to show that $a = 1$.
  Since $a < n$, we have
  $a \mid (n - 1)!$. Also,
  $n \mid (n - 1)! + 1$ by assumption,
  so $a \mid (n - 1)! + 1$ also since
  $a \mid n$. Thus
  \[
    a \mid ((n - 1)! + 1) - (n - 1)! = 1,
  \]
  so we must have $a = 1$.
\end{proof}

\begin{definition}
  A prime $p$ is a \emph{Wilson prime}
  if $(p - 1)! \equiv -1 \Pmod{p^2}$.
\end{definition}

\begin{example}
  The first few Wilson primes are
  $5, 13, 563$.
  In fact, these are the only known
  ones.
\end{example}

  \chapter{Sept.~15 --- Fermat's Little Theorem}

\begin{quote}
  \emph{What do you call it when you have
  your grandmother on speed dial?
  It's an insta gram.}
\end{quote}

\section{Fermat's Little Theorem}

\begin{theorem}[Fermat's little theorem]
  Let $p$ be a prime and $a \in \Z$
  such that $p \nmid a$. Then
  \[
    a^{p - 1} \equiv 1 \pmod{p}.
  \]
\end{theorem}

\begin{proof}
  Consider the $p - 1$ integers
  $a, 2a, 3a, \dots, (p - 1)a$. Note that
  $p \nmid a_i$ for any $1 \le i \le p - 1$.
  Note also that no two of these integers
  are congruent modulo $p$: If $a i \equiv a j \Pmod{p}$
  for some $i \ne j$, then we
  can multiply by the inverse $\overline{a}$
  of $a$ (which exists since $p \nmid a$)
  to get $i \equiv j \Pmod{p}$, which
  is impossible.
  Thus $\{a, 2a \dots, (p - 1)a\}$ is a
  complete nonzero residue system, so
  \[
    a(2a)(3a) \cdots ((p - 1)a)
    \equiv 1 \cdot 2 \cdot 3 \cdots (p - 1)
    \pmod{p}.
  \]
  Then $a^{p - 1}(p - 1)! \equiv (p - 1)! \Pmod{p}$, so
  $a^{p - 1} \equiv 1 \Pmod{p}$
  since $p \nmid (p - 1)!$.
\end{proof}

\begin{corollary}
  Let $p$ be prime and $a \in \Z$ with
  $p \nmid a$. Then
  $a^{p - 2}$ is the inverse of $a$
  modulo $p$.
\end{corollary}

\begin{proof}
  By Fermat's little theorem,
  $a \cdot a^{p - 2} = a^{p - 1} \equiv 1 \Pmod{p}$.
\end{proof}

\begin{corollary}
  Let $p$ be prime and $a \in \Z$.
  Then $a^p \equiv a \Pmod{p}$.
\end{corollary}

\begin{proof}
  If $p \mid a$, then both
  sides are congruent to
  $0$ modulo $p$. Otherwise, if
  $p \nmid a$, then we can write
  $a^p = a \cdot a^{p - 1} \equiv a \cdot 1 = a \Pmod{p}$
  by Fermat's little theorem.
\end{proof}

\begin{corollary}
  Let $p$ be prime. Then
  $2^p \equiv 2 \Pmod{p}$.
\end{corollary}

\begin{definition}
  If $n \in \Z$ is composite and
  $2^n \equiv 2 \Pmod{n}$,
  then $n$ is called a \emph{pseudoprime}.
\end{definition}

\begin{remark}
  It is known that there are infinitely
  many (even and odd) pseudoprimes.
\end{remark}

\begin{example}
  Consider $n = 341 = 11 \cdot 31$.
  To prove that $2^{341} \equiv 2 \Pmod{341}$,
  it suffices to show that
  $2^{341} \equiv 2 \Pmod{11}$
  and $2^{341} \equiv 2 \Pmod{31}$
  by the Chinese remainder theorem.
  Note that
  \begin{align*}
    2^{341}
    = (2^{10})^{34} \cdot 2 \equiv
    1^{34} \cdot 2 &= 2 \pmod{11} \\
    2^{341} = (2^{30})^{11} \cdot 2^{11}
    \equiv 1^{11} \cdot (2^5)^2 \cdot 2
    \equiv 1^2 \cdot 2 &= 2 \pmod{31}
  \end{align*}
  by Fermat's little theorem, so
  $341$ is a pseudoprime.
\end{example}

\section{Euler's Theorem}

\begin{definition}
  Let $n \in \Z$, $n > 0$.
  \emph{Euler's phi function}, denoted
  $\varphi(n)$, is the number of
  positive integers $\le n$ that are
  relatively prime to $n$. In other words,
  \[
    \varphi(n)
    = \#\{m \in \Z : 1 \le m \le n, (m, n) = 1\}.
  \]
\end{definition}

\begin{example}
  We have $\varphi(4) = 2$,
  $\varphi(14) = 6$, and
  $\varphi(p) = p - 1$ for any prime $p$.
\end{example}

\begin{theorem}[Euler's theorem]
  Let $a, m \in \Z$ with $m > 0$. If
  $(a, m) = 1$, then
  \[
    a^{\varphi(m)} \equiv 1 \pmod{m}.
  \]
\end{theorem}

\begin{proof}
  Let $r_1, r_2, \dots, r_{\varphi(m)}$
  be the distinct positive integers
  not exceeding $m$ such that
  $(r_i, m) = 1$. Then consider the
  integers $a r_1, a r_2, \dots, a r_{\varphi(m)}$.
  Note first that
  $(a r_i, m) = 1$ since
  $(r_i, m) = 1$ and $(a, m) = 1$
  by assumption. Note also that
  $a r_i \not\equiv a r_j \Pmod{m}$
  for $i \ne j$ since
  $\overline{a}$ exists (since
  $(a, m) = 1$), and multiplying
  by $\overline{a}$ implies
  $r_i \equiv r_j \Pmod{m}$, which is
  impossible. Thus the least
  nonnegative residues of
  $\{a r_1, \dots, a r_{\varphi(m)}\}$
  coincide with
  $\{r_1, \dots, r_{\varphi(m)}\}$, so
  we have
  \[
    (a r_1)(a r_2) \cdots (a r_{\varphi(m)})
    = r_1 r_2 \cdots r_{\varphi(m)} \pmod{m},
  \]
  thus $a^{\varphi(m)} (r_1 \cdots r_{\varphi(m)}) \equiv (r_1 \cdots r_{\varphi(m)}) \Pmod{m}$.
  Since $(r_1 \cdots r_{\varphi(m)}, m) = 1$,
  the inverse of $r_1 \cdots r_{\varphi(m)}$
  modulo $m$ exists, and
  multiplying by the inverse
  gives $a^{\varphi(m)} \equiv 1 \Pmod{m}$.
\end{proof}

\begin{remark}
  Taking $m = p$ recovers Fermat's little theorem
  since $\varphi(p) = p - 1$ for prime $p$.
\end{remark}

\begin{definition}
  Let $m$ be a positive integer.
  A set of $\varphi(m)$ integers
  such that each integer is relatively
  prime to $m$ and no two are
  congruent modulo $m$ is called
  a \emph{reduced residue system}
  modulo $m$.
\end{definition}

\begin{example}
  $\{1, 5, 7, 11\}$ is a reduced
  residue system modulo $12$.
  So is
  \[
    \{5(1), 5(5), 5(7), 5(11)\}
    = \{5, 25, 35, 55\}.
  \]
  For a prime $p$, the set
  $\{1, 2, \dots, p - 1\}$ is always a
  reduced residue system modulo $p$.
\end{example}

\begin{corollary}
  Let $a, m \in \Z$ with $m > 0$
  and $(a, m) = 1$. Then
  $\overline{a} \equiv a^{\varphi(m) - 1} \Pmod{m}$.
\end{corollary}

\section{Arithmetic Functions and Multiplicativity}

\begin{definition}
  An \emph{arithmetic function} is a
  function whose domain is the set of
  positive integers.
\end{definition}

\begin{example}\label{ex:arith-funcs}
  The following are examples of
  arithmetic functions:
  \begin{enumerate}
    \item Euler's $\varphi$ function;
    \item $\tau(n)$, the number of
      positive divisors of $n$;
    \item $\sigma(n)$, the sum of
      the positive divisors of $n$;
    \item $\omega(n)$, the number of
      distinct prime factors of $n$;
    \item $p(n)$, the number of
      integer partitions of $n$;
    \item $\Omega(n)$, the number of
      total prime factors (counted with
      multiplicity) of $n$.
  \end{enumerate}
\end{example}

\begin{definition}
  An arithmetic function
  $f$ is \emph{multiplicative} if
  $f(mn) = f(m) f(n)$ whenever
  $(m, n) = 1$.
  We say that $f$ is \emph{completely multiplicative}
  if $f(mn) = f(m) f(n)$ for all
  $m, n$.
\end{definition}

\begin{remark}
  Note that if $n > 1$, then we can write
  $n = p_1^{a_1} \cdots p_r^{a_r}$.
  If $f$ is multiplicative, then
  \[
    f(n) = f(p_1^{a_1} \cdots p_r^{a_r})
    = f(p_1^{a_1}) \cdots f(p_r^{a_r}).
  \]
  So multiplicative functions are
  determined by their values at
  prime powers. If
  $f$ is completely multiplicative,
  then $f(n) = f(p_1)^{a_1} \cdots f(p_r)^{a_r}$
  and $f$ is determined by its values
  at primes.
\end{remark}

\begin{example}
  The functions $\varphi, v, \sigma$
  from Example \ref{ex:arith-funcs}
  are multiplicative, while
  $\omega, p, \Omega$ are not.
\end{example}

\begin{example}
  The functions $f(n) = 1$ and $f(n) = 0$
  are completely multiplicative. The
  function $f$ defined by
  $f(1) = 1$ and $f(n) > 0$ if $n > 1$
  is also completely multiplicative.
\end{example}

\begin{remark}
  If $f$ is multiplicative and not
  identically zero, then $f(1) = 1$.
  To see this, take $n$ such
  that $f(n) \ne 0$ (since $f$ is not
  identically zero). Then
  $f(n) = f(n \cdot 1) = f(n) f(1)$, so
  $f(1) = 1$ since $f(n) \ne 0$.
\end{remark}

  \chapter{Sept.~17 --- Arithmetic Functions}

\section{Properties of Multiplicative Functions}

\begin{remark}
  We write $\sum_{d \mid n} f(d)$
  to denote a sum over the positive
  divisors of $n$. For instance,
  \[
    \sum_{d \mid 12} f(d)
    = f(1) + f(2) + f(3) + f(4) + f(6) + f(12).
  \]
\end{remark}

\begin{theorem}
  Let $f$ be an arithmetic function, and
  for $n \in \Z$, $n > 0$, define
  \[
    F(n) = \sum_{d \mid n} f(d).
  \]
  If $f$ is multiplicative, then so is
  $F$.
\end{theorem}

\begin{proof}
  Let $m, n$ be relative prime. We need
  to show that $F(mn) = F(m)F(n)$. We have
  \[
    F(mn) = \sum_{d \mid mn} f(d).
  \]
  We claim that every divisor $d$ of
  $mn$ can be written uniquely as
  $d = d_1 d_2$, where $d_1 \mid m$ and
  $d_2 \mid m$. Moreover, any such
  product $d_1 d_2$ is a divisor of $mn$.
  To see this, write
  $m = p_1^{a_1} \dots p_r^{a_r}$ and
  $n = q_1^{b_1} \dots q_s^{b_s}$, where
  all the $p_1, \dots, p_r, q_1, \dots, q_s$
  are distinct. Then if $d \mid mn$, then
  \[
    d = p_1^{e_1} \dots p_r^{e_r} q_1^{f_1} \dots q_s^{f_s},
    \quad 0 \le e_i \le q_i, 0 \le f_j \le b_j.
  \]
  Then we must choose $d_1 = p_1^{e_1} \dots p_r^{e_r}$
  and $d_2 = q_1^{f_1} \dots q_s^{f_s}$,
  which proves the claim.

  Using the claim, we can split the sum
  into
  \[
    F(mn) = \sum_{d \mid mn} f(d)
    = \sum_{d_1 \mid m} \sum_{d_2 \mid n} f(d_1 d_2)
    = \sum_{d_1 \mid m} \sum_{d_2 \mid n} f(d_1) f(d_2)
    = F(m) F(n),
  \]
  where we note that $(d_1, d_2) = 1$
  since $(m, n) = 1$.
\end{proof}

\begin{example}
  Let $m = 4$, $n = 3$. Then we can write
  \begin{align*}
    F(3 \cdot 4)
    = \sum_{d \mid 12}
    f(d)
    &= f(1) + f(2) + f(3) + f(4) + f(6) + f(12) \\
    &= f(1 \cdot 1) + f(1 \cdot 2) + f(3 \cdot 1) + f(1 \cdot 4) + f(3 \cdot 2) + f(3 \cdot 4) \\
    &= f(1) f(1) + f(1) f(2) + f(3) f(1) + f(1) f(4) + f(3) f(2) + f(3) f(4) \\
    &= (f(1) + f(3))(f(1) + f(2) + f(4))
    = F(3) F(4).
  \end{align*}
\end{example}

\section{Properties of the Euler Phi Function}

\begin{theorem}
  The Euler $\varphi$ function is
  multiplicative.
\end{theorem}

\begin{proof}
  Let $m, n \in \Z$, $m, n > 0$ with
  $(m, n) = 1$. We need to show that
  $\varphi(mn) = \varphi(m) \varphi(n)$.
  Consider the array of positive integers
  $\le mn$ organized as follows:
  \[
  \begin{matrix}
    1 & m + 1 & 2m + 1 & \cdots & (n - 1)m + 1 \\
    2 & m + 2 & 2m + 2 & \cdots & (n - 1)m + 2 \\
    \vdots & \vdots & \vdots & \ddots & \vdots \\
    i & m + i & 2m + i & \cdots & (n - 1)m + i \\
    \vdots & \vdots & \vdots & \ddots & \vdots \\
    m & 2m & 3m & \cdots & nm
  \end{matrix}
  \]
  Consider the $i$th row. If $(i, m) > 1$,
  then no element on the $i$th row is
  relatively prime to $m$ (and hence
  cannot be relatively prime to $mn$). Thus
  we may restrict our attention to those
  $i$ that satisfy $(i, m) = 1$. There
  are, by definition, $\varphi(m)$
  such values of $i$. The entries in the
  $i$th row are
  \[
    i, \quad m + i, \quad 2m + i, \quad \dots, \quad (n - 1)m + i.
  \]
  We claim that this is a complete
  residue system modulo $n$. To see this,
  suppose that
  \[
    km + i \equiv jm + i \pmod{n},
    \quad 0 \le k, j \le n - 1.
  \]
  Then $km \equiv jm \Pmod{n}$. Since
  $(m, n) = 1$, this implies
  $k \equiv j \Pmod{n}$. Since
  $0 \le k, j \le n - 1$, we must have
  $k = j$. The claim follows since
  we have $n$ non-congruent elements
  (modulo $n$)
  in the list. Thus, there are
  $\varphi(n)$ elements in the $i$th row
  that are relatively prime to $n$. Also,
  $(km + i, m) = (i, m) = 1$
  by the Euclidean algorithm, so
  they are relatively prime to $m$ as well. Thus
  $\varphi(mn) = \varphi(m) \varphi(n)$.
\end{proof}

\begin{theorem}
  Let $p$ be prime, $a \in \Z$, $a > 0$.
  Then $\varphi(p^a) = p^a - p^{a - 1}$.
\end{theorem}

\begin{proof}
  The total number of integers
  not exceeding $p^a$ is $p^a$. The only
  integers not relatively prime to
  $p^a$ are the multiples of $p$:
  $p, 2p, 3p, \dots, (p^{a - 1}) p$.
  There are $p^{a - 1}$ such integers, so
  $\varphi(p^a) = p^a - p^{a - 1}$.
\end{proof}

\begin{theorem}\label{thm:phi-product}
  Let $n \in \Z$, $n > 0$. Then
  \[
    \varphi(n) = n \prod_{p \mid n} \left(1 - \frac{1}{p}\right).
  \]
\end{theorem}

\begin{proof}
  Write $n = p_1^{a_1} \dots p_r^{a_r}$.
  Then
  \begin{align*}
    \varphi(n)
    = \varphi(p_1^{a_1} \cdots p_r^{a_r})
    &= \varphi(p_1^{a_1}) \cdots \varphi(p_r^{a_r})
    = (p_1^{a_1} - p_1^{a_1 - 1}) \cdots (p_r^{a_r} - p_r^{a_r - 1}) \\
    &= p_1^{a_1} \cdots p_r^{a_r} \left(1 - \frac{1}{p_1}\right) \cdots \left(1 - \frac{1}{p_r}\right)
    = n \prod_{p \mid n} \left(1 - \frac{1}{p}\right).
  \end{align*}
  This proves the desired formula.
\end{proof}

\begin{remark}
  One can interpret
  Theorem \ref{thm:phi-product}
  probabilistically: It says that
  $\varphi(n)$ is $n$ times the ``probability''
  that an integer is not divisible by any
  of the primes dividing $n$.
\end{remark}

\begin{example}
  Consider $n = 504 = 2^3 \cdot 3^2 \cdot 7$.
  Then $\varphi(n)$ is given by
  \[
    \varphi(504)
    = 504 \left(1 - \frac{1}{2}\right)
    \left(1 - \frac{1}{3}\right)
    \left(1 - \frac{1}{7}\right)
    = 504 \cdot \frac{1}{2} \cdot \frac{2}{3} \cdot \frac{6}{7} = 144.
  \]
\end{example}

\begin{theorem}[Gauss]
  Let $n \in \Z$, $n > 0$. Then
  \[
    \sum_{d \mid n} \varphi(d) = n.
  \]
\end{theorem}

\begin{proof}
  Let $d$ be a divisor of $n$. Define
  the set
  \[
    S_d = \{1 \le m \le n : (m, n) = d\}.
  \]
  Note that $(m, n) = d$ if and only if
  $(m / d, n / d) = 1$. Thus
  $|S_d| = \varphi(n / d)$. Note also
  that every integer less than or equal to
  $n$ belongs to exactly one of the
  $S_d$, so
  \[
    n = \sum_{d \mid n} |S_d| = \sum_{d \mid n} \varphi(n / d)
    = \sum_{d \mid n} \varphi(d),
  \]
  which the last equality follows since
  $\{d : d \mid n\} = \{n / d : d \mid n\}$.
\end{proof}

\begin{example}
  Let $n = 12$. We verify that
  $12 = \sum_{d \mid 12} \varphi(d)$.
  Write the table
  \begin{center}
  \begin{tabular}{c|c}
    $d$ & $S_d$ \\
    \hline
    $1$ & $\{1, 5, 7, 11\}$ \\
    $2$ & $\{2, 10\}$ \\
    $3$ & $\{3, 9\}$ \\
    $4$ & $\{4, 8\}$ \\
    $6$ & $\{6\}$ \\
    $12$ & $\{12\}$
  \end{tabular}
  \end{center}
  Summing the $|S_d| = \varphi(12 / d)$, we
  indeed get
  $12 = 4 + 2 + 2 + 2 + 1 + 1$.
\end{example}

\end{document}
